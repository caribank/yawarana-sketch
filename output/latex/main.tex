\documentclass{memoir}
\setsecnumdepth{subsubsection}
\usepackage{tikz}
\usepackage{geometry}
\usepackage{xcolor}
\usepackage[some]{background}

\definecolor{titlepagecolor}{RGB}{208, 84, 0}

\backgroundsetup{
scale=1,
angle=0,
opacity=1,
contents={\begin{tikzpicture}[remember picture,overlay]
 \path [fill=titlepagecolor] (-0.5\paperwidth,5) rectangle (0.5\paperwidth,10);  
\end{tikzpicture}}
}

\font\hugefont="Brill" at 38pt

\usepackage{fontspec}
\setmainfont{Brill}
\usepackage[abbrevs=none,refmode=latex]{expex-acro}
\usepackage{booktabs}
\usepackage[style=authoryear]{biblatex}
\usepackage[textwidth=30mm]{todonotes}
\def\tightlist{}
\usepackage{longtable}
\usepackage{hyperref}
\usepackage[capitalise]{cleveref}

\lingset{everygla=\itshape, belowglpreambleskip=0ex, aboveglftskip=0ex}

\title{A digital sketch grammar of Yawarana}
\author{Florian Matter;Natalia Cáceres Arandia;Spike Gildea}

\newGlossingAbbrev{1}{first person}
\newGlossingAbbrev{2}{second person}
\newGlossingAbbrev{3}{third person}
\newGlossingAbbrev{1+2}{first person inclusive}
\newGlossingAbbrev{1+3}{first person exclusive}
\newGlossingAbbrev{a}{agent-like argument}
\newGlossingAbbrev{abs}{absolutive}
\newGlossingAbbrev{all}{allative}
\newGlossingAbbrev{anim}{animate}
\newGlossingAbbrev{circ}{circumstantive}
\newGlossingAbbrev{cncs}{concessive}
\newGlossingAbbrev{concl}{conclusive}
\newGlossingAbbrev{contrast}{contrastive}
\newGlossingAbbrev{cop}{copula}
\newGlossingAbbrev{dat}{dative}
\newGlossingAbbrev{dem}{demonstrative}
\newGlossingAbbrev{des}{desiderative}
\newGlossingAbbrev{detrz}{detransivizer}
\newGlossingAbbrev{dim}{diminutive}
\newGlossingAbbrev{dist}{distal}
\newGlossingAbbrev{emp}{emphatic}
\newGlossingAbbrev{erg}{ergative}
\newGlossingAbbrev{ess}{essive}
\newGlossingAbbrev{fut}{future}
\newGlossingAbbrev{hsy}{hearsay evidentiality}
\newGlossingAbbrev{imn}{imminent}
\newGlossingAbbrev{imp}{imperative}
\newGlossingAbbrev{inan}{inanimate}
\newGlossingAbbrev{inf}{infinitive}
\newGlossingAbbrev{ins}{instrumental}
\newGlossingAbbrev{intr}{intransitive}
\newGlossingAbbrev{ints}{intensifier}
\newGlossingAbbrev{ipfv}{imperfective}
\newGlossingAbbrev{lk}{linker}
\newGlossingAbbrev{loc}{locative}
\newGlossingAbbrev{med}{medial}
\newGlossingAbbrev{mot}{motion}
\newGlossingAbbrev{motimp}{motion imperative}
\newGlossingAbbrev{neg}{negation}
\newGlossingAbbrev{nmlz}{nominalizer}
\newGlossingAbbrev{npert}{unpossessed}
\newGlossingAbbrev{nposs}{nonpossessed}
\newGlossingAbbrev{p}{patient-like argument}
\newGlossingAbbrev{pert}{pertensive}
\newGlossingAbbrev{pfv}{perfective}
\newGlossingAbbrev{pl}{plural}
\newGlossingAbbrev{plur}{pluractional}
\newGlossingAbbrev{poss}{possession}
\newGlossingAbbrev{priv}{privative}
\newGlossingAbbrev{pro}{pronoun}
\newGlossingAbbrev{prob}{probabilitive}
\newGlossingAbbrev{prog}{progressive}
\newGlossingAbbrev{proh}{prohibitive}
\newGlossingAbbrev{prox}{proximal}
\newGlossingAbbrev{pst}{past}
\newGlossingAbbrev{quot}{quotative}
\newGlossingAbbrev{rst}{restrictive}
\newGlossingAbbrev{s}{intransitive argument}
\newGlossingAbbrev{sg}{singular}
\newGlossingAbbrev{tr}{transitive}
\newGlossingAbbrev{vbz}{verbalizer}
\newGlossingAbbrev{acnnmlz}{action nominalizer}
\newGlossingAbbrev{acnmlz}{action nominalizer}
\newGlossingAbbrev{agtnmlz}{agent nominalizer}
\newGlossingAbbrev{ptcp}{participle}
\newGlossingAbbrev{sup}{supine}
\newGlossingAbbrev{purp}{purposive}
\newGlossingAbbrev{cvb}{converb}
\newGlossingAbbrev{advz}{adverbializer}
\newGlossingAbbrev{nzr}{nominalizer}
\newGlossingAbbrev{gno}{gnomic}
\newGlossingAbbrev{ctmp}{contemporative}
\newGlossingAbbrev{cond}{conditional}
\newGlossingAbbrev{perl}{perlative}
\newGlossingAbbrev{rel}{relativizer}
\newGlossingAbbrev{juss}{jussive}
\newGlossingAbbrev{ctrf}{counterfactive}
\newGlossingAbbrev{cess}{cessative}
\newGlossingAbbrev{ad}{ad-form}
\newGlossingAbbrev{postp}{postposition}
\newGlossingAbbrev{aux}{auxiliary}
\newGlossingAbbrev{adv}{adverb}
\newGlossingAbbrev{prop}{proprietive}
\newGlossingAbbrev{ana}{anaphoric}
\newGlossingAbbrev{mod}{modal}
\newGlossingAbbrev{com}{comitative}
\newGlossingAbbrev{pos}{possessed}
\newGlossingAbbrev{emph}{emphatic}
\newGlossingAbbrev{restr}{restrictive}
\newGlossingAbbrev{cnfrm}{confirmative}
\newGlossingAbbrev{md}{medial}
\newGlossingAbbrev{an}{animate}
\newGlossingAbbrev{px}{proximal}
\newGlossingAbbrev{in}{inanimate}
\newGlossingAbbrev{perf}{perfective}
\newGlossingAbbrev{inm}{immediate}
\newGlossingAbbrev{voc}{vocative}
\newGlossingAbbrev{o}{object}
\newGlossingAbbrev{rm}{remote?}
\newGlossingAbbrev{sit}{situational?}
\newGlossingAbbrev{np}{noun phrase}
\newGlossingAbbrev{pred}{predicative}
\newGlossingAbbrev{subj}{subject}
\newGlossingAbbrev{part}{particle}
\newGlossingAbbrev{psn}{person?}

\begin{document}

\begin{titlingpage}
\BgThispage
\newgeometry{left=1cm,right=1cm}
\vspace*{2cm}
\centering
\textcolor{white}{ \hugefont A digital sketch grammar of Yawarana }
\vspace*{3cm}\par
\noindent
{
\raggedleft
\begin{minipage}{0.90\linewidth}
    \begin{flushright}
        
{\Huge Florian Matter }\\[\baselineskip]

{\Huge Natalia Cáceres Arandia }\\[\baselineskip]

{\Huge Spike Gildea }\\[\baselineskip]

    \end{flushright}
\end{minipage} \hspace{15pt}
}
\centering
\vfill
\rule{0.4\textwidth}{0.4pt}\\
{\Huge 2023 \\ \large pylingdocs }
\end{titlingpage}


\tableofcontents

\chapter{\texorpdfstring{Introduction \label{intro}}{Introduction }}

\section{\texorpdfstring{The Yawarana people and their language
\label{sec:people}}{The Yawarana people and their language }}

\section{\texorpdfstring{Location, historical records
\label{sec:context}}{Location, historical records }}

\section{\texorpdfstring{Current life
\label{sec:currentlife}}{Current life }}

\section{\texorpdfstring{Sociolinguistic vitality
\label{sec:vitality}}{Sociolinguistic vitality }}

\section{\texorpdfstring{Previous studies on the Yawarana language
\label{sec:previous}}{Previous studies on the Yawarana language }}

\section{\texorpdfstring{This project
\label{sec:thisproject}}{This project }}

\chapter{\texorpdfstring{Phonetics and phonology
\label{phono}}{Phonetics and phonology }}

\section{\texorpdfstring{Segmental phonetics and phonemes
\label{sec:segmental}}{Segmental phonetics and phonemes }}

The consonant phonemes of Yawarana are shown in \cref{tab:consonants},
vowel phonemes in \cref{tab:vowels}.

\begin{table}
\caption{Consonant phonemes}
\label{tab:consonants}
\centering
\begin{tabular}{llllll}
\toprule
          & bilabial & alveolar & palatal & velar & glottal \\
\midrule
occlusive &     /p/  &     /t/  &  /t͡ʃ/  &   /k/ &         \\
    nasal &     /m/  &     /n/  &    /ɲ/  &       &         \\
fricative &          &     /s/  &         &       &    /h/  \\
   liquid &          &     /r/  &         &       &         \\
    glide &     /w/  &          &     /j/ &       &         \\
\bottomrule
\end{tabular}

\end{table}

\begin{table}
\caption{Vowel phonemes}
\label{tab:vowels}
\centering
\begin{tabular}{llll}
\toprule
      & front & central & back \\
\midrule
close &  /i/  &    /ɨ/  & /u/  \\
  mid &  /e/  &    /ə/  & /o/  \\
 open &       &    /a/  &      \\
\bottomrule
\end{tabular}

\end{table}

\subsection{\texorpdfstring{Consonants
\label{sec:consonants}}{Consonants }}

\begin{itemize}
\tightlist
\item
  minimal pairs
\end{itemize}

\subsubsection{/h/}

\begin{itemize}
\tightlist
\item
  glottal fricative insertion after dipththongs
\item
  glottal fricative insertion before occlusives
\end{itemize}

\subsection{\texorpdfstring{Vowels \label{sec:vowels}}{Vowels }}

\begin{itemize}
\item
  minimal pairs
\item
  vowel plots
\item
  what about vowel length?
\item
  variation in \emph{ë}/\emph{o}/\emph{e} and \emph{ï}/\emph{i}/\emph{u}
\item
  dipththongs

  \begin{itemize}
  \tightlist
  \item
    /ai/, /aw/, /ei/\ldots{} test combinations
  \end{itemize}
\end{itemize}

\section{\texorpdfstring{Morphophonological Processes
\label{sec:morphophono}}{Morphophonological Processes }}

\subsection{\texorpdfstring{Syllable Reduction
\label{sec:sylred}}{Syllable Reduction }}

\begin{itemize}
\tightlist
\item
  V1rV2 to V1:
\item
  nasal assimilation
\end{itemize}

\subsubsection{Contexts}

\begin{itemize}
\item
  \gl{postp}
\item
  verbal suffixes
\item
  no final nominal reduction
\end{itemize}

\subsubsection{Non-alternating reduced syllables}

e.g.~\obj{wajto} `fire'

\subsection{\texorpdfstring{Vowel harmony
\label{sec:vowelharm}}{Vowel harmony }}

\begin{itemize}
\tightlist
\item
  progressive \obj{-ri} `\gl{pert}'
\item
  regressive /ë/ \textgreater{} /o/
\end{itemize}

\subsection{\texorpdfstring{Palatalization
\label{sec:palatalization}}{Palatalization }}

\begin{itemize}
\tightlist
\item
  \obj{-sapë} `\gl{pfv}'
\item
  \obj{-se} `\gl{pst}'
\end{itemize}

\section{\texorpdfstring{Prosody \label{sec:prosody}}{Prosody }}

\subsection{\texorpdfstring{Lexical stress
\label{sec:stress}}{Lexical stress }}

\subsection{\texorpdfstring{Intonational Phrases
\label{sec:intphrases}}{Intonational Phrases }}

\todo{f0 increase associated w/ pitch reset, clause boundaries?}

\subsection{\texorpdfstring{Intonational Melodies
\label{sec:intmelodies}}{Intonational Melodies }}

\section{\texorpdfstring{Historical Considerations
\label{sec:histphono}}{Historical Considerations }}

\chapter{\texorpdfstring{Parts of speech in Yawarana
\label{POS}}{Parts of speech in Yawarana }}

\section{Distinguishing parts of speech}

\subsection{Adverbs}

\begin{itemize}
\tightlist
\item
  copredicative function
\item
  no person inflection
\end{itemize}

\section{Shared morphology}

\section{Derivation and productivity}

\begin{itemize}
\tightlist
\item
  changing word classes
\item
  semantic variation \& non-compositional meanings
\item
  productive class-changing process w/ lexically conditioned suffixes
\item
  some constructions need a different word class, no meaning change per
  se
\end{itemize}

\chapter{\texorpdfstring{Nouns \label{nouns}}{Nouns }}

\section{\texorpdfstring{Pronouns \label{sec:pronouns}}{Pronouns }}

The personal pronouns of Yawarana are shown in \cref{tab:pronouns}. The
system shows the usual Cariban inclusive/exclusive (\gl{1+2} and
\gl{1+3}) distinction, although \obj{ejnë} `\gl{1+2}\gl{pro}' does not
have the /k/ found elsewhere in the family. Note that \obj{-kontomo}
`\gl{pl}' appears to usually be restricted to verbs, while
\emph{-santomo} is found with third person pronouns and demonstratives.

\begin{table}
\caption{Pronouns}
\label{tab:pronouns}
\centering
\begin{tabular}{lll}
\toprule
         &    \gl{sg} &           \gl{pl} \\
\midrule
  \gl{1} & \obj{wïrë} &                   \\
\gl{1+2} &            &        \obj{ejnë} \\
\gl{1+3} &            &         \obj{ana} \\
  \gl{2} & \obj{mërë} &  \obj{monkontomo} \\
  \gl{3} & \obj{tëwï} & \obj{tëwïsantomo} \\
\bottomrule
\end{tabular}

\end{table}

\todo{tajne, but not attested as an article}

Reduced forms of the first and second person pronouns occur as
proclitics \todo{proclitics or prefixes?} attaching to nouns to indicate
possessor (\cref{sec:nominalperson}), attached to verbs to indicate
subject or object (described in \cref{verbinfl}), or attached to
postpositions to indicate the object of the postposition (described in
\cref{sec:postinfl}):

\ex Yawarana \\
\label{convrisamaj-28}    \begingl
    \glpreamble uyïwïj yawë usenejkari sukuri jwama //
    \gla u-y-ïwïj =yawë u-senejka-ri sukuri jwama//
    \glb \gl{1}\gl{sg}-\gl{rel}-house =\gl{loc} \gl{1}\gl{sg}-remain-\gl{ipfv} quietly //
        \glft ‘‘yo me quedo en mi casa tranquila’’//  
    \endgl 
\xe

\ex Yawarana \\
\label{desccasmaj-25}    \begingl
    \glpreamble mënai wëjkase chijpë wararë //
    \gla më-nai-Ø wëjka-se chi-jpëwara=rë//
    \glb \gl{2}-thing-\gl{poss} fall-\gl{pfv}.\gl{pst} \gl{cop}-NZRlike=\gl{emph}//
        \glft ‘‘se cayó tu cosa’’//  
    \endgl 
\xe

\ex Yawarana \\
\label{convrisamaj-2}    \begingl
    \glpreamble mëyaruwari, mëpëkëpene //
    \gla më-yaruwa-ri më-=pëkëpene//
    \glb \gl{2}-laugh-\gl{ipfv} \gl{2}-=alone//
        \glft ‘‘usted se rie sola’’//  
    \endgl 
\xe

\ex Yawarana \\
\label{ctoaragrme-7}    \begingl
    \glpreamble moyochi //
    \gla moyochi//
    \glb spider//
        \glft ‘la araña’//  
    \endgl 
\xe

The third person demonstrative pronouns or articles are shown in
\cref{tab:pronouns3}.
\todo{is there a 4‑way distinction? [cf. Ye’kwana?]} None of them have
shortened, phonologically bound counterparts.

\begin{table}
\caption{Demonstrative pronouns / articles}
\label{tab:pronouns3}
\centering
\begin{tabular}{lllll}
\toprule
              & \multicolumn{2}{l}{\gl{anim}} & \multicolumn{2}{l}{\gl{inan}} \\
\midrule
              &                   \gl{sg} &                              \gl{pl} &     \gl{sg} &        \gl{pl} \\
    \gl{prox} &                \obj{kërë} &                    \obj{kërësantomo} &   \obj{eni} &   \obj{enijne} \\
medial? near? & \obj{michi} / \obj{misi}  & \obj{michisantomo} / \obj{michitomo} &  \obj{mërë} &                \\
    \gl{dist} &               \obj{mëjkï} &                    \obj{mëkïsantomo} & \obj{mëjnï} & \obj{mëjnijne} \\
\bottomrule
\end{tabular}

\end{table}

\begin{itemize}
\tightlist
\item
  nominal interrogative pronouns:

  \begin{itemize}
  \tightlist
  \item
    \obj{anïkï} `who' (with \emph{-santomo})
  \item
    \obj{ati} `what' (no plural)
  \item
    \emph{ëjkë} `which? (\gl{inan})'
  \end{itemize}
\end{itemize}

\todo{Are there plural forms of any of these?}

\section{\texorpdfstring{Nominal inflection
\label{sec:nouninfl}}{Nominal inflection }}

Nouns in Yawarana may bear suffixes marking their possession status
(\cref{sec:nounposssuf}), number (\cref{sec:nominalnumber}), and nominal
past tense (\cref{sec:nominaltense}). Possessed nouns may bear a person
prefix, or the linker \obj{y-} (\cref{sec:nominalperson}).

\todo{noun classes re: possession}

\subsection{\texorpdfstring{Suffixes for possessed and non-possessed
nouns
\label{sec:nounposssuf}}{Suffixes for possessed and non-possessed nouns }}

In the possession construction in Yawarana, the possessor noun occurs
immediately preceding the possessed noun, which is the head of the
possession phrase. \todo{crossref to phrase structure} Alternatively,
the possessor can appear as a prefix on the possessed noun. The
possessor noun is never marked (for instance, with genitive case), but
the possessed noun (the head) is often marked for being possessed by a
suffix; an unambiguous label for this counterpart of the genitive is
pertensive \parencites{dixon2010basic}. The choice of suffix is
lexically conditioned; while most nouns take \obj{-ri} `\gl{pert}', some
take \obj{-ti}. Unpossessed nouns generally are unmarked, but some 15
nouns \todo{which? list nouns} bear the suffix \obj{-të} `\gl{npert}'
when they appear without a possessor.

Examples \exref[][unsuffixednouns]{onlypossessed} illustrate the
possible patterns of markedness for nouns when possessed and
unpossessed. The vast majority of nouns in our corpus are unmarked when
unpossessed, but when possessed the suffix \obj{-ri} `\gl{pert}' occurs
\exref[]{onlypossessed}. A handful of nouns \todo{i.e., the 15?} is
marked with \obj{-ri}/\obj{-ti} `\gl{pert}' when possessed and with
\obj{-të} `\gl{npert}' when not possessed \exref[]{diffpossessed}.
Another handful is unmarked when possessed and marked with \obj{-të}
`\gl{npert}' when not possessed \exref[]{suffunpossessed}. The fourth
logical category, where neither possession or non-possession is marked,
contains very few members (only one attested so far). For these nouns,
the difference is marked only by the presence or absence of a possessive
prefix or free-form possessor \exref[]{unsuffixednouns}.

\ex\label{onlypossessed} Nouns that take a suffix only when possessed:

\begin{tabular}[t]{llll}

 \emph{akajra-ri} &          ‘X’s bow’ & \emph{akajra} &          ‘bow’ \\

\emph{y-amaka-ri} &        ‘X’s yucca’ &  \emph{amaka} &        ‘yucca’ \\
 \emph{y-ántë-ri} &     ‘X’s fishhook’ &   \emph{antë} &     ‘fishhook’ \\
\emph{y-ateri-ri} & ‘X’s garden/field’ &  \emph{ateri} & ‘garden/field’ \\
    \emph{ënu-ru} &          ‘X’s eye’ &    \emph{ënu} &          ‘eye’ \\
  \emph{y-ëpi-ri} &     ‘X’s medicine’ &    \emph{ëpi} &     ‘medicine’ \\

\end{tabular}
 \xe

\ex\label{diffpossessed} Nouns that take one suffix when possessed and
another when unpossessed:

\begin{tabular}[t]{llll}

   \emph{yë-ri} & ‘X’s tooth’ &   \emph{yë-të} &                                ‘tooth’ \\

 \emph{pata-ri} & ‘X’s place’ & \emph{pata-të} & ‘(part of name) San Juan de Manapiare’ \\
\emph{y-ese-ti} & ‘X’s name’  &  \emph{ese-të} &                                 ‘name’ \\
\emph{y-ase-tï} & ‘X’s cord’  &  \emph{ase-të} &                                 ‘cord’ \\

\end{tabular}
 \xe

\ex\label{suffunpossessed} Nouns that take a suffix only when
unpossessed:

\begin{tabular}[t]{llll}

  \emph{yëjpë} &  ‘X’s bone’ &                \emph{yëjpë-të} &  ‘bone’ \\

   \emph{petï} & ‘X’s thigh’ & \emph{petï-të} / \emph{pej-të} & ‘thigh’ \\
\emph{y-aponi} & ‘X’s stool’ &                 \emph{apon-të} & ‘stool’ \\

\end{tabular}
 \xe

\ex\label{unsuffixednouns} Nouns that never take a suffix, whether
possessed or unpossessed:

\begin{tabular}[t]{llll}

\emph{i-jmëy} & 'his egg’ & \emph{ëjmëy} & 'egg’ \\

\end{tabular}
 \xe

\subsection{\texorpdfstring{Number suffixes
\label{sec:nominalnumber}}{Number suffixes }}

There are three plural suffixes that can occur on nouns, apparently
freely interchangeable. What conditions the choice of suffix is not
clear as of yet.

\begin{itemize}
\tightlist
\item
  \emph{-kontomo}
\end{itemize}

\ex Yawarana \\
\label{ctorat-17}    \begingl
    \glpreamble waijtatomo ëjwenakase //
    \gla waijta-tomo ëj-wenaka-se//
    \glb mouse-\gl{pl} \gl{detrz}-vomit-\gl{pst}//
        \glft ‘The mice vomited.’//  
    \endgl 
\xe

\ex Yawarana \\
\label{ctorat-40}    \begingl
    \glpreamble tipapëjsejne waijtajne //
    \gla tipa-pëj-se-jne waijta-jne//
    \glb go.in.group-\gl{plur}-\gl{pst}-\gl{pl} mouse-\gl{pl}//
        \glft ‘the mice went away.’//  
    \endgl 
\xe

\subsection{\texorpdfstring{Nominal tense
\label{sec:nominaltense}}{Nominal tense }}

\begin{itemize}
\tightlist
\item
  \obj{-jpë} `\gl{pst}'
\end{itemize}

\subsection{\texorpdfstring{Argument prefixes
\label{sec:nominalperson}}{Argument prefixes }}

Person prefixes on nouns are conditioned by the initial segment
(\cref{tab:possprefixes}). C-initial nouns take \obj{i-} `\gl{3}', and
first and second person are bare \obj{u-} `\gl{1}' and \obj{më-}
`\gl{2}'. On V-initial nouns, third person is marked with \obj{t-}
`\gl{3}', and the first and second person prefixes combine with the
linker \obj{y-} `\gl{lk}'. Some examples are shown in
\exref[][lastex]{ctorat-23}.

\begin{table}
\caption{Possessive prefixes on nouns}
\label{tab:possprefixes}
\centering
\begin{tabular}{lll}
\toprule
       &       \_C &               \_V \\
\midrule
\gl{1} &  \obj{u-} &  \obj{u-}\obj{y-} \\
\gl{2} & \obj{më-} & \obj{më-}\obj{y-} \\
\gl{3} &  \obj{i-} &          \obj{t-} \\
\bottomrule
\end{tabular}

\end{table}

\ex Yawarana \\
\label{ctorat-23}    \begingl
    \glpreamble aaa usukuru morone ta wïrë usujta ta ne //
    \gla aaa u-suku-ru morone ta-∅ wïrë u-sujta-∅ ta ne//
    \glb ah \gl{1}-urine-\gl{pert} hurting say-\gl{ipfv} \gl{1}\gl{pro} \gl{1}-urinate-\gl{ipfv} like \gl{ints}//
        \glft ‘My urine hurts, I will urinate.’//  
    \endgl 
\xe

\ex Yawarana \\
\label{convrisamaj-28}    \begingl
    \glpreamble uyïwïj yawë usenejkari sukuri jwama //
    \gla u-y-ïwïj =yawë u-senejka-ri sukuri jwama//
    \glb \gl{1}\gl{sg}-\gl{rel}-house =\gl{loc} \gl{1}\gl{sg}-remain-\gl{ipfv} quietly //
        \glft ‘‘yo me quedo en mi casa tranquila’’//  
    \endgl 
\xe

\ex Yawarana \\
\label{desccasmaj-25}    \begingl
    \glpreamble mënai wëjkase chijpë wararë //
    \gla më-nai-Ø wëjka-se chi-jpëwara=rë//
    \glb \gl{2}-thing-\gl{poss} fall-\gl{pfv}.\gl{pst} \gl{cop}-NZRlike=\gl{emph}//
        \glft ‘‘se cayó tu cosa’’//  
    \endgl 
\xe

\ex Yawarana \\
\label{ctorat-46}    \begingl
    \glpreamble tïwïj yaka waraijtokomo manikijpë //
    \gla t-ïwïj yaka waraijtokomo manikijpë//
    \glb \gl{3}-house \gl{all} man ***//
        \glft ‘He went to his house.’//  
    \endgl 
\xe

\ex Yawarana \\
\label{lastex}    \begingl
    \glpreamble pïrarë ti iwenaru wejsapë //
    \gla pïrarë ti i-wena-ru wej-sapë//
    \glb nothing \gl{hsy} \gl{3}-vomit-\gl{pert} \gl{cop}-\gl{pfv}//
        \glft ‘Their vomit was not there.’//  
    \endgl 
\xe

The linker also occurs with (pro-)nominal possessors:

\ex Yawarana \\
\label{desccasmaj-131}    \begingl
    \glpreamble tarine ma //
    \gla tarine =ma//
    \glb fast =\gl{restr}//
 
    \endgl 
\xe

There are some nouns \todo{presumably kinship terms} that take an
apparently older old second person \obj{a-} `\gl{2}'
(\cref{tab:oldpossprefixes}).

\begin{table}
\caption{Archaic possessive prefixes on nouns}
\label{tab:oldpossprefixes}
\centering
\begin{tabular}{lll}
\toprule
       &      \_C &              \_V \\
\midrule
\gl{1} & \obj{u-} & \obj{u-}\obj{y-} \\
\gl{2} & \obj{a-} & \obj{a-}\obj{y-} \\
\gl{3} & \obj{i-} &         \obj{t-} \\
\bottomrule
\end{tabular}

\end{table}

\todo{find examples for these}

\subsection{\texorpdfstring{Root suppletion in nominal possession
\label{sec:irregnouns}}{Root suppletion in nominal possession }}

\begin{itemize}
\tightlist
\item
  `father':

  \begin{itemize}
  \tightlist
  \item
    1 \emph{papa}
  \item
    2 \emph{ëmë} / \emph{omo} / \emph{ëmo} (?)
  \item
    3 \emph{imu}
  \item
    NP \emph{yïmï}
  \end{itemize}
\end{itemize}

candidates:

\begin{itemize}
\tightlist
\item
  `arrow'
\item
  `house'
\item
  `poop'
\item
  `mother'
\item
  `hammock string'
\end{itemize}

\section{\texorpdfstring{Nominal Derivational Morphology
\label{sec:nounderiv}}{Nominal Derivational Morphology }}

\begin{itemize}
\tightlist
\item
  V → N

  \begin{itemize}
  \tightlist
  \item
    \obj{-ri} `action \gl{nmlz}' \todo{potentially A.NMLZ}
  \item
    \obj{-jpë} `\gl{pst}.\gl{acnmlz}'

    \begin{itemize}
    \tightlist
    \item
      also `\gl{pst}.\gl{abs}.\gl{nmlz}' \todo{convsuenmaj-47}
    \end{itemize}
  \item
    \obj{-në} `\gl{inf}' or `generic action nominalizer'
    \todo{this probably only occurs on intransitive verbs}

    \begin{itemize}
    \tightlist
    \item
      \emph{wënkej-në} from transitive \emph{wënkepï} `forget'
    \end{itemize}
  \item
    \obj{-ni} `\gl{agtnmlz}'
    \todo{ctoyucairdi-4, descokigrme-53 for predicative use}
  \item
    \obj{n-}V\obj{-ri} only with \obj{yeme} `eat (fruits, eggs, soup)'
  \item
    \obj{-sapë} `\gl{abs}.\gl{nmlz}' (contrast with \obj{-jpë}
    `\gl{pst}.\gl{acnmlz}')
  \item
    \obj{-topo} `\gl{circ}.\gl{nmlz}'
  \item
    \obj{‑pïnï} `\gl{priv}.\gl{nmlz}'
    \todo{only found with -se-mï, not attested as nominalizer}
  \end{itemize}
\item
  Adv → N

  \begin{itemize}
  \tightlist
  \item
    \obj{-mï} `\gl{nmlz}'
  \item
    \obj{-ano} `\gl{nmlz}'
  \item
    absence of \emph{-ato} `\gl{nmlz}'
  \end{itemize}
\item
  Postp → N

  \begin{itemize}
  \tightlist
  \item
    \obj{-ano} `\gl{nmlz}'
  \end{itemize}
\item
  N → N

  \begin{itemize}
  \tightlist
  \item
    discuss \emph{pïjkë} and \emph{sere-kë} `manioc-DIM' , reference
    sections
  \item
    \emph{-imë}: e.g., \emph{wara} `woman' \emph{waraimë} `married
    woman' (dictionary)
  \end{itemize}
\item
  What about \obj{-jpë} `\gl{pst}.\gl{acnmlz}' on \gl{ad} forms? Does it
  derive a noun?

  \begin{itemize}
  \item
    \ex Yawarana \\
    \label{histyarirdi-592}    \begingl
      \glpreamble pata penarëjpë mëtë ta, mërë Cerro Muñeca tajtoj mare toto ya //
      \gla pata-Ø penarë-jpë mëtë ta mërë Cerro Muñeca taj-toj mare toto =ya//
      \glb town-\gl{poss} before-\gl{pst}.\gl{psn} there like \gl{2}\gl{sg}  say-\gl{circ} \gl{in}.\gl{rel} non.indian =\gl{erg}//
          \glft ‘‘ahí se ve el sitio donde vivian, donde los criollos llama Cerro Muñeca’’//  
      \endgl 
    \xe
  \end{itemize}
\end{itemize}

\chapter{\texorpdfstring{Verbal roots and stems
\label{verbderiv}}{Verbal roots and stems }}

\section{Classes of verbs}

Yawarana verb roots can be divided into those yielding an intransitive
stem, and those yielding a transitive stem. The only inflectional
criterion distinguishing the two classes is the third person prefix
\obj{ta-}, which can only occur on transitive stems. Thus, transitive
\obj{yawanka} `kill' can take \obj{ta-} \exref[]{convfemgrme-217}, but
intransitive \obj{yaruwa} `laugh' does not \exref[]{convrisamaj-42}.

\pex\label{}    \a Yawarana\\
    \label{convrisamaj-42}        \begingl
        \glpreamble yaruwakontomo yatum ponoko //
        \gla yaruwa-Ø-kontomo yatum//
        \glb laugh-\gl{ipfv}-\gl{pl} day//
            \glft ‘‘se rien cada dia’’//  
        \endgl 
    \a Yawarana\\
    \label{convfemgrme-217}        \begingl
        \glpreamble iyawë chipëkë, tayawankase //
        \gla i-=yawë chi-Ø=pëkë ta-yawanka-se//
        \glb \gl{3}-=\gl{ctmp} \gl{cop}-\gl{ipfv}=because \gl{3}\gl{o}-destroy-\gl{pfv}.\gl{pst}//
            \glft ‘‘por eso, la mató’’//  
        \endgl 
\xe

\todo{subclass of intransitive (?}: -nëpëkë and -tëpëkë)

\begin{itemize}
\item
  detransitive
\item
  ditransitive
\item
  ``n-adding''
\item
  accidental lability
\item
  ijtëri
\item
  Note that all transitive verbs are consonant‑initial, whether
  etymologically or not because \obj{y-} `\gl{lk}' is added to all
  vowel‑initial roots
\item
  the \emph{y‑} disappears when preceded by the detransitivizer
  \todo{examples for detransitivized verbs}
  \todo{what about V-initial intransitive verbs? how are they inflected?}
\end{itemize}

\section{\texorpdfstring{Verbalizing suffixes
\label{sec:vbz}}{Verbalizing suffixes }}

None of these are productive, although there are many lexemes derived
with them.

\subsection{Intransitive}

\subsubsection{-ta / -na}

\obj{-ta} `\gl{vbz}.\gl{intr}' derives intransitive verbs.

\begin{table}
\caption{Lexemes derived with \emph{-ta}}
\label{tab:tavbz}
\centering
\begin{tabular}{ll}
\toprule
Base & Derivation \\
\midrule
     &            \\
     &            \\
     &            \\
     &            \\
     &            \\
     &            \\
     &            \\
     &            \\
     &            \\
     &            \\
     &            \\
     &            \\
     &            \\
     &            \\
     &            \\
     &            \\
     &            \\
     &            \\
     &            \\
     &            \\
     &            \\
     &            \\
     &            \\
     &            \\
     &            \\
     &            \\
     &            \\
     &            \\
     &            \\
     &            \\
\bottomrule
\end{tabular}

\end{table}

\subsubsection{\texorpdfstring{\emph{-pamï} /
\emph{-mamï}}{-pamï / -mamï}}

\todo{check for -pantari}
\todo{check tri and way}

\subsection{Transitive}

\subsubsection{-ka}

\obj{-ka} `\gl{vbz}.\gl{tr}' derives transitive verbs.

\subsubsection{\texorpdfstring{\emph{-jtë} / \emph{-të}}{-jtë / -të}}

\begin{itemize}
\tightlist
\item
  benefactive
\end{itemize}

\subsubsection{\texorpdfstring{\emph{-ma} / \emph{-pa}}{-ma / -pa}}

\begin{itemize}
\tightlist
\item
  causative
\end{itemize}

\section{Valency-changing affixes}

\subsection{\texorpdfstring{Detransitivizing prefixes
\label{sec:detrz}}{Detransitivizing prefixes }}

\todo{distribution? morphemic analysis?}

\begin{enumerate}
\def\labelenumi{\arabic{enumi}.}
\item
  \obj{s-}
\item
  \obj{ëj-}
\item
  \obj{at-}
\end{enumerate}

\subsection{Transitivizing suffixes}

\begin{itemize}
\item
  \emph{-ma} \todo{from verbs or only nouns?}
\item
  \todo{-nïkï / -nïpï / -nëpï}
\item
  does \obj{-ka} `\gl{vbz}.\gl{tr}' go on intransitive verb stems?
\end{itemize}

\subsection{Ditransitivizing suffixes}

\begin{itemize}
\tightlist
\item
  \emph{-po}
\end{itemize}

\section{Meaning-changing suffixes}

\begin{itemize}
\item
  \gl{des} \todo{only occurs with -ri and -jra}
\item
  \gl{plur}
\item
  \gl{cess}
\end{itemize}

\chapter{\texorpdfstring{Verbal inflection
\label{verbinfl}}{Verbal inflection }}

\todo{write an introduction}

\section{\texorpdfstring{Person prefixes
\label{sec:verbperson}}{Person prefixes }}

Verbs are inflected for person with a set of prefixes, shown in
\cref{tab:verbprefixes}. First and second person prefixes show ergative
alignment, expressing \gl{s} and \gl{p}.
\todo{count cases of expressing A} Intransitive verbs are not overtly
inflected for third person, while transitive verbs show an optional
\obj{ta-} in \gl{3}\textgreater{}\gl{3}
scenarios.\todo{mentionworthy that this is not in alternation with preceding NPs}

\begin{table}
\caption{Person marking prefixes on verbs}
\label{tab:verbprefixes}
\centering
\begin{tabular}{lll}
\toprule
         &     \gl{intr} &       \gl{tr} \\
\midrule
  \gl{1} &      \obj{u-} &      \obj{u-} \\
  \gl{2} &     \obj{më-} &     \obj{më-} \\
\gl{1+2} & \emph{ej(n)-} & \emph{ej(n)-} \\
  \gl{3} &             ∅ &     \obj{ta-} \\
\bottomrule
\end{tabular}

\end{table}

\begin{itemize}
\tightlist
\item
  one attested case of \obj{ta-} `\gl{3}\textgreater{}\gl{3}' on the
  lexical verb of a \emph{-pëkë} construction w/ 2nd person \gl{a} on
  \gl{aux} \todo{which one?}

  \begin{itemize}
  \tightlist
  \item
    Ø‑ `\gl{3}\gl{p}' with transitive verbs with \gl{1}\gl{a} or
    \gl{2}\gl{a}
  \end{itemize}
\item
  one example of \obj{më-} `\gl{2}' `\gl{2}\gl{a}' on imperative verb
  \todo{GrMePers.029}
  \todo{FM: u- is sometimes wï-, but usually not transcribed & distribution unclear}
\end{itemize}

\todo{Check: appears that first and second person are reduced forms, with long vowel? wait for leila}

\begin{itemize}
\tightlist
\item
  *\emph{t‑V‑se} is no more --- the \emph{t‑} is gone, except in
  lexicalized items
\end{itemize}

\section{\texorpdfstring{Main clause tense‑aspect‑mood‑polarity suffixes
\label{sec:tam}}{Main clause tense‑aspect‑mood‑polarity suffixes }}

Verbs in main clauses are inflected for TAMP with a set of suffixes,
shown in \cref{tab:verbtam}. They are discussed in
\crefrange{sec:riipfv}{sec:sareimn}.

\begin{table}
\caption{Verbal TAM suffixes}
\label{tab:verbtam}
\centering
\begin{tabular}{ll}
\toprule
             Suffix &            Function \\
\midrule
          \obj{-ri} &        imperfective \\
         \obj{-jpë} &                past \\
          \obj{-se} &     past perfective \\
        \obj{-sapë} &             perfect \\
        \obj{-sarë} &     imminent future \\
      \obj{-nëpëkë} & \gl{prog}.\gl{intr} \\
         \obj{pëkë} &   \gl{prog}.\gl{tr} \\
    \emph{‑tojpano} &            \gl{fut} \\
     (\obj{-tojpe}) &            \gl{fut} \\
          \obj{-ja} &            \gl{neg} \\
\obj{-se}\obj{-mï}  &        ‘obligation’ \\
          \obj{-në} &        impersonal S \\
        \obj{-topo} &                     \\
\bottomrule
\end{tabular}

\end{table}

\begin{table}
\caption{Non-declarative suffixes}
\label{tab:nondecltam}
\centering
\begin{tabular}{ll}
\toprule
                                  Suffix &                                      Function \\
\midrule
                           \emph{‑jrama} &                                     \gl{proh} \\
                               \emph{-i} &                                     \gl{juss} \\
         \obj{-kë} / ‑\emph{të}\obj{-kë} &                   \gl{imp} / \gl{imp}.\gl{pl} \\
\obj{-ta} / \obj{-ta}\emph{ntë}\obj{-kë} & \gl{imp}.\gl{mot} / \gl{imp}.\gl{mot}.\gl{pl} \\
\bottomrule
\end{tabular}

\end{table}

Misc:

\begin{itemize}
\tightlist
\item
  \obj{-se}=\obj{pano} `\gl{pst}=\gl{concl}'
\item
  \obj{-saj}=\obj{pano} `\gl{pfv}=\gl{concl}'
\item
  \obj{-sarë}=\obj{pano} `\gl{imn}=\gl{concl}'
\end{itemize}

\subsection{\texorpdfstring{\obj{-ri} \label{sec:riipfv}}{ }}

\begin{itemize}
\tightlist
\item
  allomorphy:

  \begin{itemize}
  \tightlist
  \item
    \obj{-∅} `\gl{ipfv}', phonetic loss
  \item
    \obj{-ru} `\gl{ipfv}', assimilation
  \item
    what about \obj{-rï} `\gl{ipfv}'? looks like the most conservative
    form
  \end{itemize}
\item
  diachrony: \obj{-ri} `\gl{acnnmlz}'
\item
  pluralization?
\item
  combines with \obj{-jra} `\gl{neg}; \gl{priv}':
\end{itemize}

\ex Yawarana \\
\label{convrisamaj-4}    \begingl
    \glpreamble wïrë yaruwarijra\#\#\# //
    \gla wïrë yaruwa-rijra//
    \glb \gl{1}\gl{sg} laugh-\gl{neg}.\gl{ipfv}//
        \glft ‘!‘yo no me río’’//  
    \endgl 
\xe

\subsubsection{Semantics}

\begin{itemize}
\tightlist
\item
  not specified for tense, just imperfective aspect:

  \begin{itemize}
  \tightlist
  \item
    past \exref[]{ctorat-16}
  \item
    future \exref[]{convrisamaj-6}
  \item
    gnomic/present? \exref[]{gnomicri}
  \end{itemize}
\end{itemize}

\ex Yawarana \\
\label{ctorat-16}    \begingl
    \glpreamble irëjpë tëwï waijtatomo nwajtëri //
    \gla irëjpë tëwï waijta-tomo nwajtë-ri//
    \glb then \gl{3}\gl{pro} mouse-\gl{pl} dance-\gl{ipfv}//
        \glft ‘Then the mice were dancing.’//  
    \endgl 
\xe

\ex Yawarana \\
\label{convrisamaj-6}    \begingl
    \glpreamble ¿ kwase ejnë yaruwari? //
    \gla kwase ejnë yaruwa-ri//
    \glb how \gl{1+2} laugh-\gl{ipfv}//
        \glft ‘‘¿cómo vamos a reir?’’//  
    \endgl 
\xe

\pex\label{gnomicri}    \a Yawarana\\
    \label{convrisamaj-4}        \begingl
        \glpreamble wïrë yaruwarijra\#\#\# //
        \gla wïrë yaruwa-rijra//
        \glb \gl{1}\gl{sg} laugh-\gl{neg}.\gl{ipfv}//
            \glft ‘!‘yo no me río’’//  
        \endgl 
    \a Yawarana\\
    \label{convrisamaj-28}        \begingl
        \glpreamble uyïwïj yawë usenejkari sukuri jwama //
        \gla u-y-ïwïj =yawë u-senejka-ri sukuri jwama//
        \glb \gl{1}\gl{sg}-\gl{rel}-house =\gl{loc} \gl{1}\gl{sg}-remain-\gl{ipfv} quietly //
            \glft ‘‘yo me quedo en mi casa tranquila’’//  
        \endgl 
\xe

\subsection{\texorpdfstring{\obj{-jpë}}{}}

\begin{itemize}
\tightlist
\item
  allomorphy: none?
\item
  diachrony: from \obj{-jpë} `\gl{pst}.\gl{acnmlz}' \todo{negation?}
  \todo{semantics?}
\end{itemize}

\subsection{\texorpdfstring{\obj{-se}}{}}

\begin{itemize}
\tightlist
\item
  allomorphy: \obj{-se}/\obj{-che} `\gl{ptcp}; \gl{sup}'
\item
  diachrony: from \obj{-se} `\gl{ptcp}; \gl{sup}' \todo{negation?}
  \todo{semantics?}
\end{itemize}

\subsection{\texorpdfstring{\obj{-sapë}}{}}

\begin{itemize}
\tightlist
\item
  diachrony: from \obj{-sapë} `\gl{abs}.\gl{nmlz}'
\item
  distribution: only occurs on the copula?
\item
  allomorphy: \obj{-sapë} and \obj{-saj}
\item
  negation: with \obj{-ja} `\gl{neg}' on lexical verb
  \exref[][ctoaragrme-40]{ctoaragrme-38} \todo{semantics?}
\end{itemize}

\ex Yawarana \\
\label{ctoaragrme-38}    \begingl
    \glpreamble irë wejtane mujyampe patakaja wejsapë //
    \gla irë wejtane mujyam=pe pataka-ja wej-sapë//
    \glb \gl{sit}:\gl{dem} though pregnant=\gl{ess} exit-\gl{neg} \gl{cop}-\gl{perf}//
        \glft ‘‘a pesar de eso no salió embarazada’’//  
    \endgl 
\xe

\ex Yawarana \\
\label{ctoaragrme-39}    \begingl
    \glpreamble apatakaja pïnïka wejsapë //
    \gla apataka-ja pïnïka wej-sapë//
    \glb exit-\gl{neg} \gl{prob} \gl{cop}-\gl{perf}//
        \glft ‘tal vez no salió (embarazada)’//  
    \endgl 
\xe

\ex Yawarana \\
\label{ctoaragrme-40}    \begingl
    \glpreamble tayakïjtëja pïnika wejsapë //
    \gla ta-yakïjtë-ja pïnika wej-sapë//
    \glb \gl{3}\gl{o}-impregnate-\gl{neg} \gl{prob} \gl{cop}-\gl{perf}//
        \glft ‘‘tal vez no se acostó con ella’’//  
    \endgl 
\xe

\begin{itemize}
\tightlist
\item
  what about \exref[]{ctorat-19}? is that existential negation?
\end{itemize}

\ex Yawarana \\
\label{ctorat-19}    \begingl
    \glpreamble pïrarë ti iwenaru wejsapë //
    \gla pïrarë ti i-wena-ru wej-sapë//
    \glb nothing \gl{hsy} \gl{3}-vomit-\gl{pert} \gl{cop}-\gl{pfv}//
        \glft ‘Their vomit was not there.’//  
    \endgl 
\xe

\subsection{\texorpdfstring{\obj{-sarë} \label{sec:sareimn}}{ }}

\begin{itemize}
\tightlist
\item
  once a converb, now `imminent future'
\end{itemize}

\ex Yawarana \\
\label{ctorat-25}    \begingl
    \glpreamble irëjpë ta ti ta konopo wejsarë konopo wejsarë //
    \gla irëjpë ta-∅ ti ta konopo wej-sarë konopo wej-sarë//
    \glb then say-\gl{ipfv} \gl{hsy} like rain \gl{cop}-\gl{imn} rain \gl{cop}-\gl{imn}//
        \glft ‘Then they said: “it’s raining, it’s raining”.’//  
    \endgl 
\xe

\ex Yawarana \\
\label{ctoaragrme-25}    \begingl
    \glpreamble moyochi tasarë, moyochi chipokono kojpaye pïnika warotari //
    \gla moyochi ta-sarë moyochi chi-poko-no kojpaye pïnika warota-ri//
    \glb spider say-\gl{inm} spider \gl{cop}-because-\gl{nzr} night:\gl{perl} \gl{prob} work-\gl{ipfv}//
        \glft ‘le dicen araña, tal vez porque la araña trabaja de noche’//  
    \endgl 
\xe

\section{Subordinate Clause markers}

\todo{these should maybe go to another section}

\begin{itemize}
\tightlist
\item
  Nominalizations

  \begin{itemize}
  \tightlist
  \item
    \obj{-ri} `\gl{acnnmlz}'
  \item
    \obj{-jpë} `\gl{pst}.\gl{acnmlz}'
  \item
    \obj{-topo} `\gl{circ}.\gl{nmlz}'
  \end{itemize}
\item
  Adverbial Clauses (S/A)

  \begin{itemize}
  \tightlist
  \item
    \obj{-se} `supine'
  \item
    \obj{-tane} `concessive'
  \item
    \obj{-sarë} `converb'
  \item
    \emph{‑yapo} `neg.purp'
  \item
    others?
  \end{itemize}
\item
  Nominalization + postposition (S/P)

  \begin{itemize}
  \tightlist
  \item
    \obj{-∅} `\gl{ipfv}'\obj{yawë} `simult'
  \item
    \obj{-∅} `\gl{ipfv}' \obj{pe} `\gl{ess}' `when'
  \item
    \obj{-saj} `\gl{abs}.\gl{nmlz}'\obj{yawë} `simult'
  \item
    \obj{-tojpe} `purpose'
  \item
    (‑jpë)=tërë `after'
  \item
    on auxiliary: \obj{-ri} + \obj{po} `\gl{ctrf}'
  \end{itemize}
\item
  not attested:

  \begin{itemize}
  \tightlist
  \item
    \emph{se} `\gl{des}'
  \item
    \emph{-ajtawï} `if when'
  \end{itemize}
\end{itemize}

\section{Number}

\begin{itemize}
\tightlist
\item
  \emph{‑ri=kontomo}
\item
  \emph{saj=kontomo}
\item
  \emph{-pëj‑se=jne}
\item
  \emph{‑se=jne=kontomo}
\item
  \emph{‑se=jne=pano} (\emph{‑se=jne=kontom=pano}?)
\item
  \emph{-të-kë} for the imperative
\item
  what about \emph{-i} `\gl{juss}'?
\end{itemize}

\section{Copula / Auxiliary}

\todo{paradigm}
\todo{did any particles develop from inflected forms?  Man, wai, manai, etc}
\todo{are there irregular pat/perfect participles? nahkë, etc}
\todo{ejnë may come from an inflected form of the copula}

\begin{itemize}
\tightlist
\item
  there is stem allomorphy: \obj{chi}, \obj{wej}
  \todo{should these be the same morpheme?}
\item
  \obj{mare} `\gl{rel}.\gl{inan}' and \obj{manïkï} `\gl{rel}.\gl{anim}'
\end{itemize}

\chapter{\texorpdfstring{Adverbs \label{adverbs}}{Adverbs }}

\begin{itemize}
\item
  copredicative function (resultative, depictive), (S/P pivot)
\item
  also adverbial modification (S/A pivot)
\item
  no person inflection
\end{itemize}

\section{\texorpdfstring{Inflection
\label{sec:adverbinfl}}{Inflection }}

\begin{itemize}
\tightlist
\item
  negation with \obj{-jra} `\gl{neg}; \gl{priv}'
\item
  plural with \obj{-jnë} `\gl{pl}'?
\end{itemize}

\section{\texorpdfstring{Simple adverbs
\label{sec:simpleadv}}{Simple adverbs }}

\todo{list of simple adverbs}
\todo{semantic categories: quantifiers, temporal/locative words, manner, property concepts, ...}

\section{\texorpdfstring{Derived adverbs
\label{sec:derivedadv}}{Derived adverbs }}

Morphemes deriving adverbs:

\begin{itemize}
\item
  \obj{-pe} `\gl{ess}'
\item
  \obj{-re} `\gl{advz}'

  \begin{itemize}
  \tightlist
  \item
    \obj{-ye} `\gl{advz}'
  \item
    vowel change?
  \end{itemize}
\item
  \emph{-rë}
\item
  potentially \emph{-se}, \emph{-ke}, \emph{-ne}
\item
  formatives \& etymology
\end{itemize}

\subsection{\texorpdfstring{\obj{-re}}{}}

\begin{table}
\caption{Adverbs formed with \emph{-re}/\emph{-ye}}
\label{tab:reyeadvz}
\centering
\begin{tabular}{ll}
\toprule
                             \obj{-re} &                 \obj{-ye} \\
\midrule
              \obj{aponore} ‘narrowly’ &      \obj{mëtëjye} ‘thin’ \\
                \obj{chipire} ‘yellow’ &  \obj{chiramujye} ‘mangy’ \\
               \obj{chitënore} ‘whole’ &  \obj{chirimujye} ‘moldy’ \\
\obj{kojpayere} ‘early in the morning’ &  \obj{kojpaye} ‘at night’ \\
                 \obj{korore} ‘always’ &       \obj{pekuye} ‘full’ \\
               \obj{këmure} ‘purulent’ &         \obj{pijye} ‘fat’ \\
                  \obj{këyare} ‘alive’ &    \obj{potijye} ‘smelly’ \\
                \obj{mesujre} ‘bloody’ &        \obj{pëjye} ‘bent’ \\
                 \obj{mesure} ‘bloody’ &        \obj{rajye} ‘sour’ \\
                \obj{nakire} ‘thirsty’ &  \obj{romoye} ‘downriver’ \\
                    \obj{pëjre} ‘bent’ &       \obj{takiye} ‘full’ \\
         \obj{tajchiwëre} ‘tangled up’ & \obj{tënuyaye} ‘sensible’ \\
                 \obj{tajwere} ‘sweet’ &   \obj{wanëmojye} ‘round’ \\
               \obj{tapasajre} ‘muddy’ &   \obj{wanamojye} ‘round’ \\
                    \obj{tapire} ‘red’ &      \obj{yësajye} ‘sour’ \\
                   \obj{tasujre} ‘wet’ &                           \\
          \obj{turupore} ‘voluntarily’ &                           \\
           \obj{tënkëyare} ‘competent’ &                           \\
             \obj{tënsamire} ‘jealous’ &                           \\
                \obj{tëpujre} ‘clingy’ &                           \\
                \obj{tëpîre} ‘flowery’ &                           \\
              \obj{waimure} ‘speaking’ &                           \\
                 \obj{wajyare} ‘happy’ &                           \\
                 \obj{yëmïre} ‘hungry’ &                           \\
\bottomrule
\end{tabular}

\end{table}

\section{Nominalizing adverbs}

\begin{enumerate}
\def\labelenumi{\arabic{enumi}.}
\tightlist
\item
  \obj{-ano} `\gl{nmlz}'
\item
  \obj{-mï} `\gl{nmlz}'
\end{enumerate}

\chapter{\texorpdfstring{Postpositions \label{postp}}{Postpositions }}

\section{Defining the category}

\begin{itemize}
\tightlist
\item
  monomorphemic vs bipartite (vs `stacked')
\end{itemize}

\section{\texorpdfstring{Inflectional morphology
\label{sec:postinfl}}{Inflectional morphology }}

Postpositions take the same inflectional prefixes as nouns
(\cref{sec:nounposssuf}).
\todo{are there postpositions with third person i-?}

\begin{table}
\caption{Person marking prefixes on postpositions}
\label{tab:postpprefixes}
\centering
\begin{tabular}{ll}
\toprule
       \\
\midrule
\gl{1} &    \obj{u-} \\
\gl{2} &   \obj{më-} \\
\gl{3} & \obj{i-/t-} \\
\bottomrule
\end{tabular}

\end{table}

Also:

\begin{itemize}
\tightlist
\item
  \obj{-kontomo} `\gl{pl}'
\item
  \emph{ësë-}
\end{itemize}

\section{Locative Postpositions}

\begin{itemize}
\tightlist
\item
  clear bipartite Ground+Path
\item
  unproductive Bipartite X+Path?
\item
  other forms
\end{itemize}

\begin{table}
\caption{Locative postpositions}
\label{tab:locpost}
\centering
\begin{tabular}{lll}
\toprule
        &   \gl{all} &   \gl{loc} \\
\midrule
 inside & \obj{yaka} & \obj{yawë} \\
aquatic &          ? &          ? \\
\bottomrule
\end{tabular}

\end{table}

\todo{how do these fit in?}

\begin{itemize}
\item
  \emph{poye} `above'
\item
  \emph{po} `locative'
\item
  \emph{yatë} `locative'
\item
  \emph{yapo} `negation'?
\item
  allative:
\end{itemize}

\ex Yawarana \\
\label{histpajirdi-186}    \begingl
    \glpreamble tichikimuru, peti warë patakasapë Yakucho pana //
    \gla ti-chikimu-ru peti warë pataka-sapë Yakucho =pana//
    \glb \gl{3}-knee-\gl{poss} leg thus exit-\gl{perf}  =\gl{all}//
        \glft ‘‘su rodilla, su pierna, salió (llaga) hacia Ayacucho’//  
    \endgl 
\xe

\section{Nonlocative Oblique Postpositions}

\begin{itemize}
\tightlist
\item
  \obj{pana} `\gl{dat}'
\item
  \obj{ke} `\gl{ins}'
\item
  \emph{wanai}
\end{itemize}

\section{Misc}

\begin{itemize}
\tightlist
\item
  \obj{chi} `\gl{cop}' combines with \obj{yawë} `\gl{loc}; \gl{cond};
  \gl{ctmp}', sometimes spelled \emph{chi yawë}, sometimes \emph{chawë}.
\item
  syllable reduction
\item
  postpositions on bare verbs? (e.g.~\emph{wejtawë})
\end{itemize}

\chapter{\texorpdfstring{Particles, ideophones and interjections
\label{partideo}}{Particles, ideophones and interjections }}

\section{Particles}

Three kinds of particles elsewhere in the family:

\begin{enumerate}
\def\labelenumi{\arabic{enumi}.}
\tightlist
\item
  second position (modals, focus)
\item
  phrasal (focus)
\item
  clause boundary
\end{enumerate}

\begin{itemize}
\tightlist
\item
  prosodic effects?
\end{itemize}

\section{Ideophones}

\begin{itemize}
\tightlist
\item
  constructions with \obj{warë} `thus' (example with \emph{pïtï}
  `paint')
\end{itemize}

\section{Interjections}

\begin{itemize}
\tightlist
\item
  kind of particle?
\end{itemize}

\chapter{\texorpdfstring{Negation \label{negation}}{Negation }}

\begin{itemize}
\tightlist
\item
  probably relevant morphemes:

  \begin{itemize}
  \tightlist
  \item
    \obj{-ja} `\gl{neg}'
  \item
    \obj{-jra} `\gl{neg}; \gl{priv}'
  \item
    \obj{-jnari} `\gl{neg}'
  \item
    \obj{-jrama} `\gl{proh}'
  \item
    \obj{-kempïnirë} `\gl{ptcp}.\gl{nzr}.\gl{gno}:\gl{neg}'
  \item
    \obj{pïnirë} `nothing'
  \item
    \obj{pïrarë} `nothing'
  \item
    \emph{‑yapo} `neg.purp'
  \end{itemize}
\end{itemize}

\chapter{Auxiliarized constructions}

claim: everything can take an auxiliary, except \obj{-kë} `\gl{imp}'

\todo{look at frequency and distributional possibilities for various forms with auxiliaries}
\todo{are there limits on what form of AUX can occur?}
\todo{conventionalized meanings of combinations?}
\todo{where is person marking? also alignment}

\section{Defining auxiliaries}

\section{Main clauses}

\begin{itemize}
\tightlist
\item
  multiple auxiliaries
\end{itemize}

\section{Subordinate clauses}

\begin{itemize}
\tightlist
\item
  \emph{chi=pëkë}
\item
  \emph{chi=yawë}/\emph{chawë}
\item
  \emph{chi-ripo}
\item
  \emph{wej-tojpe}
\end{itemize}

\chapter{\texorpdfstring{Phrases \label{phrases}}{Phrases }}

\chapter{\texorpdfstring{Nonverbal predications
\label{nonverbal}}{Nonverbal predications }}

\textcites[366]{gildea2018reconstructing} distinguishes two main formal
types of nonverbal predication in Cariban languages: juxtaposition and
copular constructions.

\begin{itemize}
\tightlist
\item
  juxtaposed NP + ADV or NP + LOC is found in Arara, Ikpeng, Ye'kwana,
  Wayana, and Apalaí
\item
  for PC:

  \begin{itemize}
  \tightlist
  \item
    Nsubj + Npred: nominal (juxtaposition) predication. Limited in
    functional domains.
  \item
    Nsubj + \gl{cop} + Adverbial (adverbs/postpositional phrases).
    Fairly unlimited.
  \end{itemize}
\item
  Innovations:

  \begin{itemize}
  \tightlist
  \item
    Nsubj + \gl{cop} + Npred (S\&M 2009)
  \item
    Nsubj + Adverbial
  \end{itemize}
\end{itemize}

\section{Observations}

\subsection{Patterns}

\begin{itemize}
\tightlist
\item
  copula almost always occurs with adverbs, fairly often with locatives
  and existential particles, the only tokens with nouns are edge cases

  \begin{itemize}
  \tightlist
  \item
    no copula with adverbs when negated
    \exref[]{temp-neg-nsubj-advpred-jra}
  \item
    \gl{np}\gl{pred} \gl{cop} only with `sick' (a noun?)
    \exref[]{temp-aff-npred-cop}, \exref[]{temp-q-npred-cop}
  \item
    \gl{np}\gl{pred} \gl{np}\gl{subj} \gl{cop} only occurs with
    \emph{chi-yawë} `when' \exref[]{cat-aff-npred-nsubj-cop},
    \exref[]{temp-aff-npred-nsubj-cop}
    \todo{does the etymological presence of COP in this construction imply that it once had a copula?}
  \item
    pattern for existential particles?

    \begin{itemize}
    \tightlist
    \item
      copula used to express tense in existential function?
      \exref[]{ex-aff-part-cop-nsubj} vs \exref[]{ex-aff-part-nsubj}
    \item
      no such pattern with locatives\ldots{}
      \exref[]{loc-aff-part-cop-nsubj} vs \exref[]{loc-aff-part-nsubj}
    \end{itemize}
  \end{itemize}
\item
  copula does not combine with \obj{pïrarë} `nothing'

  \begin{itemize}
  \tightlist
  \item
    one counterexample with \gl{part}\gl{pred} + \obj{pïrarë} + \gl{cop}
    + \gl{np}\gl{subj} is with a concessive
    \exref[]{ex-neg-part-pirare-cop-nsubj} -- copula required for
    marking?
  \item
    it does occur with \obj{pïnirë} `nothing', though
    \exref[]{loc-neg-nsubj-cop-pinire-part}
  \end{itemize}
\item
  negation:

  \begin{itemize}
  \tightlist
  \item
    \obj{-jra} `\gl{neg}; \gl{priv}' for adverbs and as privative (?) on
    nouns in existential predicates
  \item
    \obj{pïnirë} `nothing' for nominal predicates and ones with locative
    particles
  \item
    \obj{pïrarë} `nothing' for existential predicates
  \item
    \emph{-ja} or \emph{-jnari} on the copula
  \end{itemize}
\item
  order is fairly flexible; potentially rigid {[}ADVpred (Nsubj) COP{]}
\end{itemize}

\subsection{Open questions}

\begin{itemize}
\tightlist
\item
  unclear role of \obj{manïkï} `\gl{rel}.\gl{anim}' in \gl{np}\gl{pred}
  + \gl{np}\gl{subj} construction
\item
  is there a pattern as to whether existential/locative particles
  combine with the copula?
\end{itemize}

\subsection{Categorization issues}

\begin{itemize}
\tightlist
\item
  how to analyze `close to \gl{loc}' constructions? \obj{tëijpo} `far'
  seems to be an adverb.
\item
  possessives vs properties? (`footed', etc.)
\item
  existentials vs locatives both with
  \emph{mëtë}/\emph{mïntë}/\emph{entë} (largely went by translation)

  \begin{itemize}
  \tightlist
  \item
    sometimes they co-occur \exref[]{convfemgrme-157},
    \exref[]{convfemgrme-99}
  \end{itemize}
\item
  one apparent example of \gl{adv}\gl{pred} without copula
  \exref[]{perm-q-advpred-nsubj}
\item
  weird ``adverbial subject''? \exref[]{poss-neg-advsubj-pirare-locpred}
\item
  construction with two copulas \emph{chi wejsapë}?
  \exref[]{convhistfamsjm-92}, \exref[]{convhistfamsjm-59},
  \exref[]{histgrme-17}, \exref[]{histgrme-107}
\end{itemize}

\section{Overview tables}

\begin{table}
\caption{An overview of nonverbal predication in main clauses}
\label{tab:nvpoverview_main}
\centering
\begin{tabular}{llll}
\toprule
          Function &                                        Affirmative &                                           Negative &                                      Interrogative \\
\midrule
    Identification & \gl{np}~\gl{pred}~ (+ \gl{np}~\gl{subj}~) \exre... & \gl{np}~\gl{pred}~ \obj{pïnirë} (+ \gl{np}~\gl{... & \gl{np}\textsubscript{\gl{pred}} (+ \gl{np}\tex... \\
    Categorization & \gl{np}\textsubscript{\gl{pred}} + \gl{np}\text... & \gl{np}~\gl{pred}~ \obj{pïnirë} (+ \gl{np}~\gl{... & \gl{np}~\gl{pred}~ (+ \gl{np}~\gl{subj}~) \exre... \\
Permanent property & \gl{np}\textsubscript{\gl{pred}} (+ \gl{np}\tex... & \gl{np}~\gl{pred}~ \obj{pïnirë} (+ \gl{np}~\gl{... & \gl{adv}~\gl{pred}~ (\gl{np}~\gl{subj}~) \exref... \\
Temporary property & \gl{np}~\gl{pred}~ \gl{cop} \exref[]{temp-main-... & (\gl{np}~\gl{subj}~) \gl{adv}~\gl{pred}~\obj{-j... & \gl{np}~\gl{pred}~ \gl{cop} \exref[]{temp-main-... \\
          Location & \gl{loc}~\gl{pred}~ + \gl{cop} (+ \gl{np}~\gl{s... & \gl{np}~\gl{subj}~ + \gl{cop} + \obj{pïnirë} + ... & \gl{part}~\gl{pred}~ + \gl{np}~\gl{subj}~ \exre... \\
       Existential & \gl{part}~\gl{pred}~ + \gl{cop} (+ \gl{np}~\gl{... & \gl{part}~\gl{pred}~ \obj{pïrarë} (+\gl{np}~\gl... &                                                  ? \\
        Possession & \gl{np}~\gl{pred}~ (+ \gl{np}~\gl{subj}~) \exre... & ADVsubj \obj{pïrarë} \gl{loc}~\gl{pred}~ \exref... & \gl{loc}~\gl{pred}~ + \gl{cop} (+ \gl{np}~\gl{s... \\
\bottomrule
\end{tabular}

\end{table}

\begin{table}
\caption{An overview of nonverbal predication in subordinate clauses}
\label{tab:nvpoverview_sub}
\centering
\begin{tabular}{lll}
\toprule
          Function &                                        Affirmative &                                           Negative \\
\midrule
    Identification &                                                  ? &                                                  ? \\
    Categorization & \gl{np}~\gl{pred}~ \gl{np}~\gl{subj}~ \gl{cop} ... &                                                  ? \\
Permanent property & \gl{adv}~\gl{pred}~ (\gl{np}~\gl{subj}~) \gl{co... &                                                  ? \\
Temporary property & \gl{np}~\gl{pred}~ \gl{np}~\gl{subj}~ \gl{cop} ... &                                                  ? \\
          Location & \gl{part}~\gl{pred}~ + \gl{np}~\gl{subj}~ \exre... & \gl{loc}~\gl{pred}~ + \gl{cop}-\gl{neg} (+ \gl{... \\
       Existential &                                                  ? &                                                  ? \\
        Possession &                                                  ? &                                                  ? \\
\bottomrule
\end{tabular}

\end{table}

\begin{table}
\caption{Semasiological overview of affirmative nonverbal predications}
\label{tab:nvp_aff_main}
\centering
\begin{tabular}{llllllll}
\toprule
                                                   &                                  id &                                         cat &                                         perm &                                  temp &                                        loc &                                     ex &                                        poss \\
\midrule
         \gl{np}~\gl{pred}~ (+ \gl{np}~\gl{subj}~) & ✓ \exref[]{id-main-aff-npred-nsubj} &        ✓ \exref[]{cat-main-aff-npred-nsubj} &        ✓ \exref[]{perm-main-aff-npred-nsubj} & ✓ \exref[]{temp-main-aff-npred-nsubj} &                                            &                                        &       ✓ \exref[]{poss-main-aff-npred-nsubj} \\
\gl{np}\textsubscript{\gl{pred}} + \gl{np}\text... &                                     & ✓ \exref[]{cat-main-aff-npred-nsubj-maniki} & ✓ \exref[]{perm-main-aff-npred-nsubj-maniki} &                                       &                                            &                                        &                                             \\
 \gl{adv}~\gl{pred}~ (\gl{np}~\gl{subj}~) \gl{cop} &                                     &                                             &                                              &                                       & ✓ \exref[]{loc-main-aff-advpred-nsubj-cop} &                                        & ✓ \exref[]{poss-main-aff-advpred-nsubj-cop} \\
                       \gl{np}~\gl{pred}~ \gl{cop} &                                     &                                             &                                              &   ✓ \exref[]{temp-main-aff-npred-cop} &                                            &                                        &                                             \\
         \gl{part}~\gl{pred}~ + \gl{np}~\gl{subj}~ &                                     &                                             &                                              &                                       &        ✓ \exref[]{loc-main-aff-part-nsubj} &     ✓ \exref[]{ex-main-aff-part-nsubj} &                                             \\
\gl{loc}~\gl{pred}~ + \gl{cop} (+ \gl{np}~\gl{s... &                                     &                                             &                                              &                                       & ✓ \exref[]{loc-main-aff-locpred-cop-nsubj} &                                        & ✓ \exref[]{poss-main-aff-locpred-cop-nsubj} \\
        \gl{loc}~\gl{pred}~ (+ \gl{np}~\gl{subj}~) &                                     &                                             &                                              &                                       &     ✓ \exref[]{loc-main-aff-locpred-nsubj} &                                        &                                             \\
\gl{part}~\gl{pred}~ + \gl{cop} (+ \gl{np}~\gl{... &                                     &                                             &                                              &                                       &    ✓ \exref[]{loc-main-aff-part-cop-nsubj} & ✓ \exref[]{ex-main-aff-part-cop-nsubj} &                                             \\
\bottomrule
\end{tabular}

\end{table}

\begin{table}
\caption{Semasiological overview of affirmative nonverbal predications in
subordinate clauses}
\label{tab:nvp_aff_sub}
\centering
\begin{tabular}{lllll}
\toprule
                                                   &                                     cat &                                       perm &                                     temp &                                       loc \\
\midrule
    \gl{np}~\gl{pred}~ \gl{np}~\gl{subj}~ \gl{cop} & ✓ \exref[]{cat-sub-aff-npred-nsubj-cop} &                                            & ✓ \exref[]{temp-sub-aff-npred-nsubj-cop} &                                           \\
 \gl{adv}~\gl{pred}~ (\gl{np}~\gl{subj}~) \gl{cop} &                                         & ✓ \exref[]{perm-sub-aff-advpred-nsubj-cop} &                                          & ✓ \exref[]{loc-sub-aff-advpred-nsubj-cop} \\
         \gl{part}~\gl{pred}~ + \gl{np}~\gl{subj}~ &                                         &                                            &                                          &        ✓ \exref[]{loc-sub-aff-part-nsubj} \\
\gl{loc}~\gl{pred}~ + \gl{cop} (+ \gl{np}~\gl{s... &                                         &                                            &                                          & ✓ \exref[]{loc-sub-aff-locpred-cop-nsubj} \\
\gl{part}~\gl{pred}~ + \gl{cop} (+ \gl{np}~\gl{... &                                         &                                            &                                          &    ✓ \exref[]{loc-sub-aff-part-cop-nsubj} \\
\bottomrule
\end{tabular}

\end{table}

\begin{table}
\caption{Semasiological overview of negated nonverbal predications}
\label{tab:nvp_neg_main}
\centering
\begin{tabular}{llllllll}
\toprule
                                                   &                                         id &                                         cat &                                         perm &                                            temp &                                            loc &                                            ex &                                             poss \\
\midrule
\gl{np}~\gl{pred}~ \obj{pïnirë} (+ \gl{np}~\gl{... & ✓ \exref[]{id-main-neg-npred-pinire-nsubj} & ✓ \exref[]{cat-main-neg-npred-pinire-nsubj} & ✓ \exref[]{perm-main-neg-npred-pinire-nsubj} &                                                 &                                                &                                               &                                                  \\
(\gl{np}~\gl{subj}~) \gl{adv}~\gl{pred}~\obj{-jra} &                                            &                                             &                                              &     ✓ \exref[]{temp-main-neg-nsubj-advpred-jra} &                                                &                                               &                                                  \\
\gl{adv}~\gl{pred}~ + \gl{cop}-\gl{neg} (+ \gl{... &                                            &                                             &                                              & ✓ \exref[]{temp-main-neg-advpred-cop-neg-nsubj} &                                                &                                               &                                                  \\
\gl{loc}~\gl{pred}~ + \gl{cop}-\gl{neg} (+ \gl{... &                                            &                                             &                                              &                                                 & ✓ \exref[]{loc-main-neg-locpred-cop-neg-nsubj} &                                               &                                                  \\
\gl{np}~\gl{subj}~ + \gl{cop} + \obj{pïnirë} + ... &                                            &                                             &                                              &                                                 & ✓ \exref[]{loc-main-neg-nsubj-cop-pinire-part} &                                               &                                                  \\
\gl{part}~\gl{pred}~ + \obj{pïnirë} (+\gl{np}~\... &                                            &                                             &                                              &                                                 &     ✓ \exref[]{loc-main-neg-part-pinire-nsubj} &                                               &      ✓ \exref[]{poss-main-neg-part-pinire-nsubj} \\
\gl{part}~\gl{pred}~ \obj{pïrarë} (+\gl{np}~\gl... &                                            &                                             &                                              &                                                 &                                                &     ✓ \exref[]{ex-main-neg-part-pirare-nsubj} &                                                  \\
\gl{part}~\gl{pred}~ \obj{pïrarë} + \gl{cop} + ... &                                            &                                             &                                              &                                                 &                                                & ✓ \exref[]{ex-main-neg-part-pirare-cop-nsubj} &                                                  \\
         \obj{pïrarë} \gl{np}~\gl{subj}~\obj{-jra} &                                            &                                             &                                              &                                                 &                                                &      ✓ \exref[]{ex-main-neg-pirare-nsubj-jra} &                                                  \\
                                      \obj{pïrarë} &                                            &                                             &                                              &                                                 &                                                &                ✓ \exref[]{ex-main-neg-pirare} &                                                  \\
          ADVsubj \obj{pïrarë} \gl{loc}~\gl{pred}~ &                                            &                                             &                                              &                                                 &                                                &                                               & ✓ \exref[]{poss-main-neg-advsubj-pirare-locpred} \\
\bottomrule
\end{tabular}

\end{table}

\begin{table}
\caption{Semasiological overview of negated nonverbal predications in subordinate
clauses}
\label{tab:nvp_neg_sub}
\centering
\begin{tabular}{ll}
\toprule
                                                   &                                           loc \\
\midrule
\gl{loc}~\gl{pred}~ + \gl{cop}-\gl{neg} (+ \gl{... & ✓ \exref[]{loc-sub-neg-locpred-cop-neg-nsubj} \\
\bottomrule
\end{tabular}

\end{table}

\begin{table}
\caption{Semasiological overview of interrogative nonverbal predications}
\label{tab:nvp_q_main}
\centering
\begin{tabular}{lllllll}
\toprule
                                                   &                                       id &                                cat &                                  perm &                              temp &                                   loc &                                      poss \\
\midrule
         \gl{np}~\gl{pred}~ (+ \gl{np}~\gl{subj}~) &        ✓ \exref[]{id-main-q-npred-nsubj} & ✓ \exref[]{cat-main-q-npred-nsubj} &                                       &                                   &                                       &                                           \\
\gl{np}\textsubscript{\gl{pred}} + \gl{np}\text... & ✓ \exref[]{id-main-q-npred-nsubj-maniki} &                                    &                                       &                                   &                                       &                                           \\
          \gl{adv}~\gl{pred}~ (\gl{np}~\gl{subj}~) &                                          &                                    & ✓ \exref[]{perm-main-q-advpred-nsubj} &                                   &                                       &                                           \\
                       \gl{np}~\gl{pred}~ \gl{cop} &                                          &                                    &                                       & ✓ \exref[]{temp-main-q-npred-cop} &                                       &                                           \\
         \gl{part}~\gl{pred}~ + \gl{np}~\gl{subj}~ &                                          &                                    &                                       &                                   &     ✓ \exref[]{loc-main-q-part-nsubj} &                                           \\
\gl{loc}~\gl{pred}~ + \gl{cop} (+ \gl{np}~\gl{s... &                                          &                                    &                                       &                                   &                                       & ✓ \exref[]{poss-main-q-locpred-cop-nsubj} \\
        \gl{loc}~\gl{pred}~ (+ \gl{np}~\gl{subj}~) &                                          &                                    &                                       &                                   &  ✓ \exref[]{loc-main-q-locpred-nsubj} &                                           \\
\gl{part}~\gl{pred}~ + \gl{cop} (+ \gl{np}~\gl{... &                                          &                                    &                                       &                                   & ✓ \exref[]{loc-main-q-part-cop-nsubj} &                                           \\
\bottomrule
\end{tabular}

\end{table}

\section{Examples}

\pex\label{id-main-aff-npred-nsubj}    \a Yawarana\\
    \label{convcosnoind-37}        \begingl
        \glpreamble ejnë okïrï tanapï //
        \gla ejnë okï-rï tanapï//
        \glb \gl{1+2} drink-\gl{poss} soft.drink//
            \glft ‘‘nuestra bebida es el yaraki dulce’’//  
        \endgl 
    \a Yawarana\\
    \label{hist2mape-22}        \begingl
        \glpreamble sere yawajni //
        \gla sere yawaj-ni//
        \glb cassava.bread grate-\gl{a}.\gl{nzr}//
            \glft ‘‘esa es la que raya yuca’’//  
        \endgl 
    \a Yawarana\\
    \label{descmensgrme-68}        \begingl
        \glpreamble tëwï Panare tëwïjne //
        \gla tëwï Panare tëwï-jne//
        \glb \gl{3}\gl{sg}  \gl{3}\gl{sg}-\gl{pl}//
            \glft ‘‘esos son los Panare’’//  
        \endgl 
    \a Yawarana\\
    \label{ctotawirdi-91}        \begingl
        \glpreamble orojyamo wacho ti kërë //
        \gla orojyamo wacho-Ø ti kërë//
        \glb soul wife-\gl{poss} like \gl{3}\gl{an}:\gl{md}//
            \glft ‘esa es la esposa del diablo’//  
        \endgl 
    \a Yawarana\\
    \label{histyarirdi-54}        \begingl
        \glpreamble ejnë nonori tëwï //
        \gla ejnë nono-ri tëwï//
        \glb \gl{1+2} soil-\gl{poss} \gl{3}\gl{sg}//
            \glft ‘‘esa es la tierra de nosotros’’//  
        \endgl 
\xe

\pex\label{id-main-neg-npred-pinire-nsubj}    \a Yawarana\\
    \label{convamgu-42}        \begingl
        \glpreamble mokontomo nono pïnirë seni //
        \gla mokontomo nono -pïnirë seni//
        \glb \gl{2}\gl{pl} soil -\gl{neg} \gl{3}\gl{in}:\gl{px}//
            \glft ‘‘esta no es tu tierra’’//  
        \endgl 
    \a Yawarana\\
    \label{histpajirdi-84}        \begingl
        \glpreamble ¿ tëwï pïnirë? //
        \gla tëwï -pïnirë//
        \glb \gl{3}\gl{sg} -\gl{neg}//
            \glft ‘‘¿no es esa?’’//  
        \endgl 
    \a Yawarana\\
    \label{histyarirdi-312}        \begingl
        \glpreamble ejnë yakerej pïnirë //
        \gla ejnë y-akerej -pïnirë//
        \glb \gl{1+2} \gl{rel}-relative -\gl{neg}//
            \glft ‘‘no es nuestra familia’’//  
        \endgl 
\xe

\pex\label{id-main-q-npred-nsubj}    \a Yawarana\\
    \label{convamgu-94}        \begingl
        \glpreamble ayakono ka Petra? //
        \gla a-y-akono ka petra//
        \glb \gl{2}-\gl{rel}-ygr.brother \gl{qp} Petra//
            \glft ‘¿Petra es tu hermana menor?’//  
        \endgl 
    \a Yawarana\\
    \label{convamgu-95}        \begingl
        \glpreamble ¿ ayakono ka, puriri mërë? //
        \gla a-y-akono ka puri-ri mërë//
        \glb \gl{2}-\gl{rel}-ygr.brother \gl{qp} older.bro.\gl{m}-\gl{poss} \gl{2}\gl{sg}//
            \glft ‘¿tu menor, tú eres el mayor?’//  
        \endgl 
\xe

\ex Yawarana \\
\label{id-main-q-npred-nsubj-maniki}    \begingl
    \glpreamble enijpëtërë ma, tëwï maniki //
    \gla enijpëtërë =ma tëwï maniki//
    \glb one =\gl{restr} \gl{3}\gl{sg} \gl{an}.\gl{rel}//
        \glft ‘‘el es el único?’//  
    \endgl 
\xe

\pex\label{cat-main-aff-npred-nsubj}    \a Yawarana\\
    \label{hist2mape-21}        \begingl
        \glpreamble kërë wurijyamo //
        \gla kërë wurijyamo//
        \glb \gl{3}\gl{an}:\gl{md} woman//
            \glft ‘‘esa es una hembra’’//  
        \endgl 
    \a Yawarana\\
    \label{histyarirdi-623}        \begingl
        \glpreamble am, yëye pïnirë, tëpu //
        \gla am yëye -pïnirë tëpu//
        \glb um tree -\gl{neg} rock//
            \glft ‘‘no es palo, es piedra’’//  
        \endgl 
\xe

\ex Yawarana \\
\label{cat-main-aff-npred-nsubj-maniki}    \begingl
    \glpreamble sorori rë mëkï maniki yenepese ta //
    \gla soro-ri =rë mëkï maniki yenepe-se ta//
    \glb lie-\gl{poss} =\gl{emph} \gl{3}\gl{an}:\gl{rm} \gl{an}.\gl{rel} send.away-\gl{pfv}.\gl{pst} like//
        \glft ‘‘eso es un engaño, para espantar’’//  
    \endgl 
\xe

\ex Yawarana \\
\label{cat-sub-aff-npred-nsubj-cop}    \begingl
    \glpreamble tëwïjpë tëwïsantomo takï, papa pano ma tayakarama wejsapë, mïntë wïrë chi wejsapë, muku marë wïrë chi yawë //
    \gla tëwï-jpë tëwïsantomo takï papa =pano =ma ta-yakarama-Ø wej-sapë mïntë wïrë chi-Ø wej-sapë muku marë wïrë chi-Ø =yawë//
    \glb \gl{3}\gl{sg}-\gl{mod} \gl{3}\gl{pl} \gl{cnfrm} father:\gl{voc} =late =\gl{restr} \gl{3}\gl{o}-tell-\gl{ipfv} \gl{cop}-\gl{perf} there \gl{1}\gl{sg} \gl{cop}-\gl{ipfv} \gl{cop}-\gl{perf} child like \gl{1}\gl{sg} \gl{cop}-\gl{ipfv} =\gl{ctmp}//
        \glft ‘‘eso ellos, lo contaba mi papá, yo estaba allá, cuando yo estaba niña’’//  
    \endgl 
\xe

\ex Yawarana \\
\label{cat-main-neg-npred-pinire-nsubj}    \begingl
    \glpreamble am, yëye pïnirë, tëpu //
    \gla am yëye -pïnirë tëpu//
    \glb um tree -\gl{neg} rock//
        \glft ‘‘no es palo, es piedra’’//  
    \endgl 
\xe

\pex\label{cat-main-q-npred-nsubj}    \a Yawarana\\
    \label{convfemgrme-315}        \begingl
        \glpreamble wurichi ken ka imukuru //
        \gla wurichi ka i-muku-ru//
        \glb female \gl{qp} \gl{3}-child-\gl{poss}//
            \glft ‘‘su hija era hembra?’’//  
        \endgl 
    \a Yawarana\\
    \label{histpajirdi-278}        \begingl
        \glpreamble tëwï ne mëyakerej ka? //
        \gla tëwï =ne më-y-akerej ka//
        \glb \gl{3}\gl{sg} =\gl{ints} \gl{2}-\gl{rel}-relative \gl{qp}//
            \glft ‘‘el es tu familia?’’//  
        \endgl 
\xe

\pex\label{perm-main-aff-npred-nsubj}    \a Yawarana\\
    \label{conv1stenc-80}        \begingl
        \glpreamble intompïjkë pïnirë, kamprakemï, kasoremï //
        \gla intompïjkë -pïnirë Ø-kampra-ke-mï kasore-mï//
        \glb small -\gl{neg} -\gl{prop}-big-\gl{prop}-\gl{nzr} fat-\gl{nzr}//
            \glft ‘‘no era pequeño, era grande y era gordo’’//  
        \endgl 
    \a Yawarana\\
    \label{convinsectmaj-118}        \begingl
        \glpreamble mmm, sokowa mukujpë, intompijkë //
        \gla mmm sokowa mukujpë intompijkë//
        \glb yes patawa.palm small small//
            \glft ‘sí, sókowa es el pequeñito’’//  
        \endgl 
\xe

\ex Yawarana \\
\label{perm-main-aff-npred-nsubj-maniki}    \begingl
    \glpreamble sokowa mukujpë manïkï //
    \gla sokowa mukujpë manïkï//
    \glb patawa.palm small \gl{an}.\gl{rel}//
        \glft ‘‘sokowa es el pequeño’’//  
    \endgl 
\xe

\ex Yawarana \\
\label{perm-sub-aff-advpred-nsubj-cop}    \begingl
    \glpreamble iyawë takï kamprake ejnë chi yawëno rë taji rë, //
    \gla i-=yawë takï kampra-ke ejnë chi-Ø =yawë-no =rë taji =rë//
    \glb \gl{3}-=\gl{ctmp} \gl{cnfrm} big-\gl{prop} \gl{1+2} \gl{cop}-\gl{ipfv} =\gl{loc}-\gl{nzr} =\gl{emph} ? =\gl{emph}//
        \glft ‘‘en ese momento ya eramos grandes’’//  
    \endgl 
\xe

\pex\label{perm-main-neg-npred-pinire-nsubj}    \a Yawarana\\
    \label{descmensgrme-67}        \begingl
        \glpreamble makë neke ne Yawarana na pïnirë //
        \gla makë neke =ne Yawarana na -pïnirë//
        \glb mother:\gl{voc} \gl{contrast} =\gl{ints} Yawarana more -\gl{neg}//
            \glft ‘‘mi mamá no era muy Yawarana’’//  
        \endgl 
    \a Yawarana\\
    \label{ctovarmafl-64}        \begingl
        \glpreamble karasakem pïnirë mëkï //
        \gla karasake-m -pïnirë mëkï//
        \glb white-\gl{nzr} -\gl{neg} \gl{3}\gl{an}:\gl{rm}//
            \glft ‘‘ese no es blanco’’//  
        \endgl 
\xe

\ex Yawarana \\
\label{perm-main-q-advpred-nsubj}    \begingl
    \glpreamble ¿ tëwï ka mojne? //
    \gla tëwï ka mojne//
    \glb \gl{3}\gl{sg} \gl{qp} salty//
        \glft ‘‘¿esos son salados?’’//  
    \endgl 
\xe

\ex Yawarana \\
\label{temp-main-aff-npred-cop}    \begingl
    \glpreamble terepun, terepun wejsapë, nwa ijtëse, Yakucho pana ijtëjpë, irëjpë ta ijtë rë sëmase, tawara taji //
    \gla t-erepun t-erepun wej-sapë nwa ij-të-se Yakucho =pana ij-të-jpë irëjpë ta ijtë =rë sëma-se tawara taji//
    \glb \gl{3}-sick \gl{3}-sick \gl{cop}-\gl{perf} thus \gl{3}-go-\gl{pfv}.\gl{pst} Ayacucho =\gl{all} \gl{3}-go-\gl{pst} then like there =\gl{emph} die-\gl{pfv}.\gl{pst} thus ?//
        \glft ‘‘estaba enfermo, así se fue, se fue a Ayacucho y allá mismo murió’’//  
    \endgl 
\xe

\pex\label{temp-main-aff-npred-nsubj}    \a Yawarana\\
    \label{desccasmaj-85}        \begingl
        \glpreamble irë pana rë warë yapijtom terepun rërë //
        \gla irë pana =rë warë yapijtom t-erepun rërë//
        \glb and.now =\gl{emph} thus old \gl{3}-sick //
            \glft ‘‘pa completar el viejo el viejo está enfermo’’//  
        \endgl 
    \a Yawarana\\
    \label{conv1stenc-99}        \begingl
        \glpreamble terepun, terepun wejsapë, nwa ijtëse, Yakucho pana ijtëjpë, irëjpë ta ijtë rë sëmase, tawara taji //
        \gla t-erepun t-erepun wej-sapë nwa ij-të-se Yakucho =pana ij-të-jpë irëjpë ta ijtë =rë sëma-se tawara taji//
        \glb \gl{3}-sick \gl{3}-sick \gl{cop}-\gl{perf} thus \gl{3}-go-\gl{pfv}.\gl{pst} Ayacucho =\gl{all} \gl{3}-go-\gl{pst} then like there =\gl{emph} die-\gl{pfv}.\gl{pst} thus ?//
            \glft ‘‘estaba enfermo, así se fue, se fue a Ayacucho y allá mismo murió’’//  
        \endgl 
\xe

\ex Yawarana \\
\label{temp-sub-aff-npred-nsubj-cop}    \begingl
    \glpreamble pïrajrarë wïrë chi yawë pïnika, tajne ya wïrë yarëse warë, mïntëno Majawa të wïrë këyëtase //
    \gla pïrajrarë wïrë chi-Ø =yawë pïnika tajne =ya wïrë yarë-se warë mïntë-no Majawa =të wïrë këyëta-se//
    \glb small \gl{1}\gl{sg} \gl{cop}-\gl{ipfv} =\gl{ctmp} \gl{prob} \gl{3}\gl{pl} =\gl{erg} \gl{1}\gl{sg} carry-\gl{pfv}.\gl{pst} thus there-\gl{nzr} Majagua =\gl{loc} \gl{1}\gl{sg} grow.up-\gl{pfv}.\gl{pst}//
        \glft ‘‘tal vez cuando yo estaba pequeña, ellos me llevaron, allá en Majagua crecí’’//  
    \endgl 
\xe

\pex\label{temp-main-neg-advpred-cop-neg-nsubj}    \a Yawarana\\
    \label{descmensgrme-43}        \begingl
        \glpreamble tawara uchija //
        \gla tawara u-chi-ja//
        \glb thus \gl{1}\gl{sg}-\gl{cop}-\gl{neg}//
            \glft ‘‘yo no era así’’//  
        \endgl 
    \a Yawarana\\
    \label{histyarirdi-248}        \begingl
        \glpreamble tawara ana chija ana //
        \gla tawara ana chi-ja ana//
        \glb thus \gl{1+3} \gl{cop}-\gl{neg} \gl{1+3}//
            \glft ‘‘así no eramos nosotros’’//  
        \endgl 
    \a Yawarana\\
    \label{histyarirdi-249}        \begingl
        \glpreamble chija tawara penarë //
        \gla chi-ja tawara penarë//
        \glb \gl{cop}-\gl{neg} thus before//
            \glft ‘‘así no era antes’’//  
        \endgl 
\xe

\pex\label{temp-main-neg-nsubj-advpred-jra}    \a Yawarana\\
    \label{convestsjm-34}        \begingl
        \glpreamble enirë takï tawarajra //
        \gla enirë takï tawara-jra//
        \glb now \gl{cnfrm} thus-\gl{neg}//
            \glft ‘‘no es así como ahora’’//  
        \endgl 
    \a Yawarana\\
    \label{histaccigrme-2}        \begingl
        \glpreamble wïrë jtari nopejra //
        \gla wïrë jta-ri nope-jra//
        \glb \gl{1}\gl{sg} foot-\gl{poss} good-\gl{neg}//
            \glft ‘‘mi pie está malo’’//  
        \endgl 
\xe

\ex Yawarana \\
\label{temp-main-q-npred-cop}    \begingl
    \glpreamble ¿ terepun ka wejsapë? //
    \gla t-erepun ka wej-sapë//
    \glb \gl{3}-sick \gl{qp} \gl{cop}-\gl{perf}//
        \glft ‘¿estaba enfermo?’//  
    \endgl 
\xe

\ex Yawarana \\
\label{loc-main-aff-advpred-nsubj-cop}    \begingl
    \glpreamble entë ma teijpojra wejtane //
    \gla entë =ma tëijpo-jra wej-tane//
    \glb here =\gl{restr} far-\gl{neg} \gl{cop}-\gl{cncs}//
        \glft ‘‘aunque está cerca de aquí’’//  
    \endgl 
\xe

\pex\label{loc-main-aff-locpred-cop-nsubj}    \a Yawarana\\
    \label{convfemgrme-292}        \begingl
        \glpreamble ijtë wejsapë, wïrë yakërë wejsapë //
        \gla ijtë wej-sapë wïrë y-akërë wej-sapë//
        \glb there \gl{cop}-\gl{perf} \gl{1}\gl{sg} \gl{rel}-\gl{com} \gl{cop}-\gl{perf}//
            \glft ‘‘ahí estaba, estaba conmigo’’//  
        \endgl 
    \a Yawarana\\
    \label{histgrme-107}        \begingl
        \glpreamble ana chi wejsapë mëtë sawë të //
        \gla ana chi-Ø wej-sapë mëtë sawë =të//
        \glb \gl{1+3} \gl{cop}-\gl{ipfv} \gl{cop}-\gl{perf} there rapid =\gl{loc}//
            \glft ‘‘nosotros estuvimos allí en el raudal’’//  
        \endgl 
    \a Yawarana\\
    \label{convamgu-127}        \begingl
        \glpreamble mïntë La Esperanza të ana chi yawë //
        \gla mïntë La Esperanza =të ana chi-Ø =yawë//
        \glb there La Esperanza =\gl{loc} \gl{1+3} \gl{cop}-\gl{ipfv} =\gl{ctmp}//
            \glft ‘‘allá cuando estabamos en la Esperanza’’//  
        \endgl 
\xe

\pex\label{loc-main-aff-locpred-nsubj}    \a Yawarana\\
    \label{convamgu-80}        \begingl
        \glpreamble Wayawarë të rë pïnika tëwï //
        \gla wayawarë =të =rë pïnika tëwï//
        \glb Guayabal =\gl{loc} =\gl{emph} \gl{prob} \gl{3}\gl{sg}//
            \glft ‘sí tal vez está en Guayabal’’//  
        \endgl 
    \a Yawarana\\
    \label{histyarirdi-339}        \begingl
        \glpreamble wïrë ntawë sukase //
        \gla wïrë ntawë sukase//
        \glb \gl{1}\gl{sg} in.mouth all//
            \glft ‘‘todo está en mi boca’’//  
        \endgl 
\xe

\pex\label{loc-main-aff-part-cop-nsubj}    \a Yawarana\\
    \label{convfemgrme-292}        \begingl
        \glpreamble ijtë wejsapë, wïrë yakërë wejsapë //
        \gla ijtë wej-sapë wïrë y-akërë wej-sapë//
        \glb there \gl{cop}-\gl{perf} \gl{1}\gl{sg} \gl{rel}-\gl{com} \gl{cop}-\gl{perf}//
            \glft ‘‘ahí estaba, estaba conmigo’’//  
        \endgl 
    \a Yawarana\\
    \label{histgrme-17}        \begingl
        \glpreamble takërë, ana chi wejsapë mëntë, wasai të //
        \gla t-akërë ana chi-Ø wej-sapë mïntë wasai =të//
        \glb \gl{3}-\gl{com} \gl{1+3} \gl{cop}-\gl{ipfv} \gl{cop}-\gl{perf} there palm.sp =\gl{loc}//
            \glft ‘‘con él estuvimos allá, en Cucurito’’//  
        \endgl 
    \a Yawarana\\
    \label{convamgu-99}        \begingl
        \glpreamble mïjna wïrë chija chipokono, entë ma wïrë chiri //
        \gla mïjna wïrë chi-ja chi-poko-no entë =ma wïrë chi-ri//
        \glb there \gl{1}\gl{sg} \gl{cop}-\gl{neg} \gl{cop}-because-\gl{nzr} here =\gl{restr} \gl{1}\gl{sg} \gl{cop}-\gl{ipfv}//
            \glft ‘‘como yo no estaba allá, estoy aquí no más’’//  
        \endgl 
    \a Yawarana\\
    \label{convamgu-138}        \begingl
        \glpreamble entë ma ejnë chi rë //
        \gla entë =ma ejnë chi-Ø =rë//
        \glb here =\gl{restr} \gl{1+2} \gl{cop}-\gl{ipfv} =\gl{emph}//
            \glft ‘‘vamos a estar aquí’’//  
        \endgl 
    \a Yawarana\\
    \label{convhistfamsjm-92}        \begingl
        \glpreamble tëwïjpë tëwïsantomo takï, papa pano ma tayakarama wejsapë, mïntë wïrë chi wejsapë, muku marë wïrë chi yawë //
        \gla tëwï-jpë tëwïsantomo takï papa =pano =ma ta-yakarama-Ø wej-sapë mïntë wïrë chi-Ø wej-sapë muku marë wïrë chi-Ø =yawë//
        \glb \gl{3}\gl{sg}-\gl{mod} \gl{3}\gl{pl} \gl{cnfrm} father:\gl{voc} =late =\gl{restr} \gl{3}\gl{o}-tell-\gl{ipfv} \gl{cop}-\gl{perf} there \gl{1}\gl{sg} \gl{cop}-\gl{ipfv} \gl{cop}-\gl{perf} child like \gl{1}\gl{sg} \gl{cop}-\gl{ipfv} =\gl{ctmp}//
            \glft ‘‘eso ellos, lo contaba mi papá, yo estaba allá, cuando yo estaba niña’’//  
        \endgl 
    \a Yawarana\\
    \label{ctorosq-17}        \begingl
        \glpreamble tëwï totope wejsapë mïntë pare //
        \gla tëwï toto=pe wej-sapë mïntë pare//
        \glb \gl{3}\gl{sg} non.indian=\gl{ess} \gl{cop}-\gl{perf} there and//
            \glft ‘‘él estaba allá como persona’’//  
        \endgl 
    \a Yawarana\\
    \label{ctovarmafl-40}        \begingl
        \glpreamble entë ana chiri petomyakërë yatunu //
        \gla entë ana chi-ri petomyakërë yatunu//
        \glb here \gl{1+3} \gl{cop}-\gl{ipfv} three sun//
            \glft ‘‘aquí vamos a estar 3 días’’//  
        \endgl 
    \a Yawarana\\
    \label{histyarirdi-674}        \begingl
        \glpreamble entë ma ta ejnë chiri, mïntë ti toto yamïjpë sukase //
        \gla entë =ma ta ejnë chi-ri mïntë ti toto yamï-jpë sukase//
        \glb here =\gl{restr} like \gl{1+2} \gl{cop}-\gl{ipfv} there like non.indian pick.up-\gl{pst} all//
            \glft ‘‘aquí no más vamos a estar, allá todo lo agarraron los criollos’’//  
        \endgl 
\xe

\pex\label{loc-main-aff-part-nsubj}    \a Yawarana\\
    \label{histanfo-1}        \begingl
        \glpreamble wïrë tawara rë entë //
        \gla wïrë tawara =rë entë//
        \glb \gl{1}\gl{sg} too =\gl{emph} here//
            \glft ‘‘yo estaba aquí (en San Juan Manapiare)’’//  
        \endgl 
    \a Yawarana\\
    \label{histgrme-3}        \begingl
        \glpreamble mëntë torori //
        \gla mïntë toro-ri//
        \glb there cementery-\gl{poss}//
            \glft ‘‘su tumba está allá’’//  
        \endgl 
    \a Yawarana\\
    \label{histgrme-67}        \begingl
        \glpreamble mïjna mëtëri ejnë pata yaka wïrë patari mëtë Kaño Awïyama tënojpë wïrë wepïjpë wïrë //
        \gla mïjna më-të-ri ejnë pata-Ø =yaka wïrë pata-ri mëtë Kaño Awïyama =të-nojpë wïrë wepï-jpë wïrë//
        \glb there \gl{2}-go-\gl{ipfv} \gl{1+2} town-\gl{poss} =into \gl{1}\gl{sg} town-\gl{poss} there   =\gl{loc}-\gl{abla} \gl{1}\gl{sg} come-\gl{pst} \gl{1}\gl{sg}//
            \glft ‘‘vaya para allá, a nuestro pueblo, mi pueblo es allá, yo me vine desde Caño Auyama’’//  
        \endgl 
    \a Yawarana\\
    \label{convinsectmaj-33}        \begingl
        \glpreamble entë tëmuru //
        \gla entë t-ëmu-ru//
        \glb here \gl{3}-testicle-\gl{poss}//
            \glft ‘‘aquí están sus testículos’’//  
        \endgl 
    \a Yawarana\\
    \label{convamgu-35}        \begingl
        \glpreamble mëtë ma mëmukutomo //
        \gla mëtë =ma më-muku-tomo//
        \glb there =\gl{restr} \gl{2}-child-\gl{pl}//
            \glft ‘‘ahí mismo están tus hijos’’//  
        \endgl 
    \a Yawarana\\
    \label{convamgu-76}        \begingl
        \glpreamble mëtë ëmë //
        \gla mëtë ëmë//
        \glb there \gl{2}:father:\gl{pos}//
            \glft ‘‘ahí está tu papá’’//  
        \endgl 
    \a Yawarana\\
    \label{convamgu-89}        \begingl
        \glpreamble tëwïsantomo neke ne mëtë //
        \gla tëwïsantomo neke =ne mëtë//
        \glb \gl{3}\gl{pl} \gl{contrast} =\gl{ints} there//
            \glft ‘‘ellos sí están ahí’’//  
        \endgl 
    \a Yawarana\\
    \label{ctorosq-28}        \begingl
        \glpreamble entë wïrë, ta ti ta //
        \gla entë wïrë ta-Ø ti ta//
        \glb here \gl{1}\gl{sg} say-\gl{ipfv} like like//
            \glft ‘‘aquí estoy, dijo’’//  
        \endgl 
    \a Yawarana\\
    \label{ctovarmafl-366}        \begingl
        \glpreamble mëkï takï mïntë, kawë ti //
        \gla mëkï takï mïntë kawë ti//
        \glb \gl{3}\gl{an}:\gl{rm} \gl{cnfrm} there high like//
            \glft ‘‘aquel estaba allá arriba’’//  
        \endgl 
\xe

\ex Yawarana \\
\label{loc-sub-aff-advpred-nsubj-cop}    \begingl
    \glpreamble teijpojra ëjtë pëkë teijpojra, yëye chipëkë mïntë //
    \gla tëijpo-jra ëjtë =pëkë tëijpo-jra yëye chi-Ø=pëkë mïntë//
    \glb far-\gl{neg} house =about far-\gl{neg} tree \gl{cop}-\gl{ipfv}=because there//
        \glft ‘‘porque estaba un palo cerca de la casa’’//  
    \endgl 
\xe

\pex\label{loc-sub-aff-locpred-cop-nsubj}    \a Yawarana\\
    \label{convfemgrme-99}        \begingl
        \glpreamble tawara chijpë ta mïntë sawë të ana chi yawë, mïntë //
        \gla tawara chi-jpë ta mïntë sawë =të ana chi-Ø =yawë mïntë//
        \glb thus \gl{cop}-\gl{pst} like there rapid =\gl{loc} \gl{1+3} \gl{cop}-\gl{ipfv} =\gl{ctmp} there//
            \glft ‘‘así también fue allá cuando estabamos en el raudal, allá’’//  
        \endgl 
    \a Yawarana\\
    \label{convfemgrme-157}        \begingl
        \glpreamble mïntë, Manapiare ana chi yawë //
        \gla mïntë Manapiare ana chi-Ø =yawë//
        \glb there Manapiare \gl{1+3} \gl{cop}-\gl{ipfv} =\gl{ctmp}//
            \glft ‘‘cuando estuvimos en Manapiare’’//  
        \endgl 
\xe

\ex Yawarana \\
\label{loc-sub-aff-part-cop-nsubj}    \begingl
    \glpreamble entë ma wïrë chi chipokono //
    \gla entë =ma wïrë chi-Ø chi-poko-no//
    \glb here =\gl{restr} \gl{1}\gl{sg} \gl{cop}-\gl{ipfv} \gl{cop}-because-\gl{nzr}//
        \glft ‘‘por eso yo estoy aquí mismo’’//  
    \endgl 
\xe

\ex Yawarana \\
\label{loc-sub-aff-part-nsubj}    \begingl
    \glpreamble asanë mëtë chipokono //
    \gla asanë mëtë chi-poko-no//
    \glb \gl{2}:mother:\gl{pos} there \gl{cop}-because-\gl{nzr}//
        \glft ‘‘porque tienen su mamá ahí’’//  
    \endgl 
\xe

\pex\label{loc-main-neg-locpred-cop-neg-nsubj}    \a Yawarana\\
    \label{ctorosq-8}        \begingl
        \glpreamble tawara tajne yëmpëkapëtïrï, mëwaraijtori yakërë ta wïrë chija, tata wejsaj ta //
        \gla tawara tajne yëmpëka-pëtï-rï më-waraijto-ri y-akërë ta wïrë chi-ja ta-ta-Ø wej-saj ta//
        \glb too \gl{3}\gl{pl} offend-\gl{plac}-\gl{ipfv} \gl{2}-husband-\gl{poss} \gl{rel}-\gl{com} like \gl{1}\gl{sg} \gl{cop}-\gl{neg} \gl{3}\gl{o}-say-\gl{ipfv} \gl{cop}-\gl{perf} like//
            \glft ‘‘así la insultaban, con tu esposo no estuve, decía’’//  
        \endgl 
    \a Yawarana\\
    \label{histpajirdi-81}        \begingl
        \glpreamble wïrë yakërë wïchirijnari wïrëjpë waraijtori //
        \gla wïrë y-akërë wï-chi-ri-jnari wïrë-jpë waraijto-ri//
        \glb \gl{1}\gl{sg} \gl{rel}-\gl{com} \gl{1}\gl{sg}-\gl{cop}-\gl{ipfv}-\gl{neg} \gl{1}\gl{sg}-\gl{mod} husband-\gl{poss}//
            \glft ‘‘conmigo no está mi marido’’//  
        \endgl 
\xe

\ex Yawarana \\
\label{loc-main-neg-nsubj-cop-pinire-part}    \begingl
    \glpreamble kërë chijpë pïnirë entë, pïrarë //
    \gla kërë chi-jpë =pïnirë entë pïrarë//
    \glb \gl{3}\gl{an}:\gl{md} \gl{cop}-\gl{pst} =\gl{neg} here nothing//
        \glft ‘‘ese no estaba aquí, no había’’//  
    \endgl 
\xe

\ex Yawarana \\
\label{loc-main-neg-part-pinire-nsubj}    \begingl
    \glpreamble asanë mëtë pïnirë //
    \gla asanë mëtë =pïnirë//
    \glb \gl{2}:mother:\gl{pos} there =\gl{neg}//
        \glft ‘‘¿tu mamá no está ahí?’’//  
    \endgl 
\xe

\ex Yawarana \\
\label{loc-sub-neg-locpred-cop-neg-nsubj}    \begingl
    \glpreamble mïjna wïrë chija chipokono, entë ma wïrë chiri //
    \gla mïjna wïrë chi-ja chi-poko-no entë =ma wïrë chi-ri//
    \glb there \gl{1}\gl{sg} \gl{cop}-\gl{neg} \gl{cop}-because-\gl{nzr} here =\gl{restr} \gl{1}\gl{sg} \gl{cop}-\gl{ipfv}//
        \glft ‘‘como yo no estaba allá, estoy aquí no más’’//  
    \endgl 
\xe

\ex Yawarana \\
\label{loc-main-q-locpred-nsubj}    \begingl
    \glpreamble ¿ ëkë yaye chima? //
    \gla ëkë =yaye chima//
    \glb which =\gl{perl} path//
        \glft ‘¿por donde es ese camino?’//  
    \endgl 
\xe

\pex\label{loc-main-q-part-cop-nsubj}    \a Yawarana\\
    \label{convhistfamsjm-49}        \begingl
        \glpreamble kwajsapene mokontomo wejsapë mëtë //
        \gla kwajsapene mokontomo wej-sapë mëtë//
        \glb how.much.time \gl{2}\gl{pl} \gl{cop}-\gl{perf} there//
            \glft ‘‘cuanto tiempo estaban ahí?’’//  
        \endgl 
    \a Yawarana\\
    \label{convhistfamsjm-59}        \begingl
        \glpreamble petomyakërë yatunë ana chi wejsapë mëtë, wanene pano yakërë //
        \gla petomyakërë yatunë ana chi-Ø wej-sapë mëtë wanene =pano y-akërë//
        \glb three sun \gl{1+3} \gl{cop}-\gl{ipfv} \gl{cop}-\gl{perf} there aunt:\gl{voc} =late \gl{rel}-\gl{com}//
            \glft ‘‘estabamos ahí tres días, con mi tía’’//  
        \endgl 
\xe

\pex\label{loc-main-q-part-nsubj}    \a Yawarana\\
    \label{convamgu-77}        \begingl
        \glpreamble ¿ ëmë neke ne mëtë? //
        \gla ëmë neke =ne mëtë//
        \glb \gl{2}:father:\gl{pos} \gl{contrast} =\gl{ints} there//
            \glft ‘¿tu papá está ahí, no?’//  
        \endgl 
    \a Yawarana\\
    \label{ctowaru-66}        \begingl
        \glpreamble mëtë ka mokontomo ta //
        \gla mëtë ka mokontomo ta//
        \glb there \gl{qp} \gl{2}\gl{pl} like//
            \glft ‘‘¿ahí están ustedes?’’//  
        \endgl 
\xe

\pex\label{ex-main-aff-part-cop-nsubj}    \a Yawarana\\
    \label{convcosnoind-49}        \begingl
        \glpreamble mëtë wejtane, intipijkë ma wejsapë, kamicha waraijne //
        \gla mëtë wej-tane intipijkë =ma wej-sapë kamicha warai-jne//
        \glb there \gl{cop}-\gl{cncs} a.little =\gl{restr} \gl{cop}-\gl{perf} clothing like-\gl{pl}//
            \glft ‘‘aunque existía, había poquito’//  
        \endgl 
    \a Yawarana\\
    \label{convfemgrme-284}        \begingl
        \glpreamble mïntë tujnaka yakasempejra tuna rë wejsapë rërë mïntë //
        \gla mïntë tujnaka yaka-sempejra tuna =rë wej-sapë rërë mïntë//
        \glb there deep.in excavate-\gl{neg}.\gl{abil} water =\gl{emph} \gl{cop}-\gl{perf}  there//
            \glft ‘‘allá no podían cobar hondo, había mucha agua allá’’//  
        \endgl 
    \a Yawarana\\
    \label{conv1stenc-123}        \begingl
        \glpreamble wurijyantomo rë wejsapë ijtë //
        \gla wurijyan-tomo =rë wej-sapë ijtë//
        \glb woman-\gl{pl} =\gl{emph} \gl{cop}-\gl{perf} there//
            \glft ‘‘hay había puras mujeres no más’’//  
        \endgl 
    \a Yawarana\\
    \label{ctovarmafl-354}        \begingl
        \glpreamble mëtë yïpï wejsaj ti ta //
        \gla mëtë yïpï wej-saj ti ta//
        \glb there mount \gl{cop}-\gl{perf} like like//
            \glft ‘‘ahí había un cerro (alto)’’//  
        \endgl 
    \a Yawarana\\
    \label{ctovarmafl-355}        \begingl
        \glpreamble yïpï wejsapë mëtë //
        \gla yïpï wej-sapë mëtë//
        \glb mount \gl{cop}-\gl{perf} there//
            \glft ‘‘había un cerrote ahí’’//  
        \endgl 
    \a Yawarana\\
    \label{ctovarmafl-453}        \begingl
        \glpreamble kamprari chiri entë //
        \gla kampra-ri chi-ri entë//
        \glb big-\gl{poss} \gl{cop}-\gl{ipfv} here//
            \glft ‘‘hay grandes aquí’’//  
        \endgl 
    \a Yawarana\\
    \label{histpajirdi-120}        \begingl
        \glpreamble ¡ taa \%¡ mïntë tëpu pïnika wejsapë warë //
        \gla taa mïntë tëpu pïnika wej-sapë warë//
        \glb crack there rock \gl{prob} \gl{cop}-\gl{perf} thus//
            \glft ‘‘ta! allá será que había una piedra’’//  
        \endgl 
\xe

\pex\label{ex-main-aff-part-nsubj}    \a Yawarana\\
    \label{convcosnoind-48}        \begingl
        \glpreamble eni takï tajtojpe ne kamicha mëtë //
        \gla eni takï tajtojpe =ne kamicha mëtë//
        \glb \gl{3}\gl{in}:\gl{px} \gl{cnfrm} \gl{hesit} =\gl{ints} clothing there//
            \glft ‘‘ahora ahí está ropa’’//  
        \endgl 
    \a Yawarana\\
    \label{histgrme-76}        \begingl
        \glpreamble tuna paijche mïntë tapasajre //
        \gla tuna paijche mïntë tapasajre//
        \glb water deep there muddy//
            \glft ‘‘alla hay agua hondo, charcos’’//  
        \endgl 
    \a Yawarana\\
    \label{histgrme-86}        \begingl
        \glpreamble entë tënësemï //
        \gla entë tënësemï//
        \glb here fish//
            \glft ‘‘aquí hay comida’’//  
        \endgl 
    \a Yawarana\\
    \label{histgrme-89}        \begingl
        \glpreamble aaa ëjpïna mëtë tënësemï seremakë //
        \gla aaa ëjpïna mëtë tënësemï serema-kë//
        \glb ah a.lot there fish eat-\gl{imp}//
            \glft ‘‘ah, alla hay bastante comida, come’’//  
        \endgl 
    \a Yawarana\\
    \label{ctorosq-116}        \begingl
        \glpreamble mëntë takï wasai punemï //
        \gla mïntë takï wasai pune-mï//
        \glb there \gl{cnfrm} palm.sp meaty-\gl{nzr}//
            \glft ‘ahí hay un cucurito que tiene carne (fruta)’//  
        \endgl 
\xe

\ex Yawarana \\
\label{ex-main-neg-part-pirare-cop-nsubj}    \begingl
    \glpreamble irë wejtane entë pïrarë //
    \gla irë wej-tane entë pïrarë//
    \glb \gl{tmp}:\gl{dem} \gl{cop}-\gl{cncs} here nothing//
        \glft ‘‘sin embargo aquí no hay’’//  
    \endgl 
\xe

\pex\label{ex-main-neg-part-pirare-nsubj}    \a Yawarana\\
    \label{convinsectmaj-18}        \begingl
        \glpreamble entë pïrarë, mïjna //
        \gla entë pïrarë mïjna//
        \glb here nothing there//
            \glft ‘‘aquí no hay, allá’’//  
        \endgl 
    \a Yawarana\\
    \label{histyarirdi-823}        \begingl
        \glpreamble nono entë pïrarë tëwïjna neke n ti yamïrï //
        \gla nono entë pïrarë tëwïjna neke =n ti yamï-rï//
        \glb soil here nothing there.\gl{ana} \gl{contrast} =\gl{ints} like pick.up-\gl{ipfv}//
            \glft ‘‘aquí no hay barro, allá si se agarra’’//  
        \endgl 
    \a Yawarana\\
    \label{histyarirdi-824}        \begingl
        \glpreamble entë pïrarë ta wïrë ya //
        \gla entë pïrarë ta wïrë =ya//
        \glb here nothing like \gl{1}\gl{sg} =\gl{erg}//
            \glft ‘‘aquí no hay digo yo’’//  
        \endgl 
\xe

\ex Yawarana \\
\label{ex-main-neg-pirare}    \begingl
    \glpreamble kërë chijpë pïnirë entë, pïrarë //
    \gla kërë chi-jpë =pïnirë entë pïrarë//
    \glb \gl{3}\gl{an}:\gl{md} \gl{cop}-\gl{pst} =\gl{neg} here nothing//
        \glft ‘‘ese no estaba aquí, no había’’//  
    \endgl 
\xe

\pex\label{ex-main-neg-pirare-nsubj-jra}    \a Yawarana\\
    \label{desccasmaj-64}        \begingl
        \glpreamble serere yawakïja chi yawë, pïrarë serejra //
        \gla serere yawakï-ja chi-Ø =yawë pïrarë sere-jra//
        \glb  grate-\gl{neg} \gl{cop}-\gl{ipfv} =\gl{cond} nothing cassava.bread-\gl{priv}//
            \glft ‘‘si no se ralla yuca, no hay yuca’’//  
        \endgl 
    \a Yawarana\\
    \label{desccasmaj-65}        \begingl
        \glpreamble mañuku yamanëja chi yawë, pïrarë mañukujra //
        \gla mañuku yamanë-ja chi-Ø =yawë pïrarë mañuku-jra//
        \glb manioc.meal make-\gl{neg} \gl{cop}-\gl{ipfv} =\gl{cond} nothing manioc.meal-\gl{priv}//
            \glft ‘‘si no hace mañoco, no hay mañoco’’//  
        \endgl 
\xe

\ex Yawarana \\
\label{poss-main-aff-advpred-nsubj-cop}    \begingl
    \glpreamble tajrominke chikontomo //
    \gla t-tajromin-ke chi-Ø-kontomo//
    \glb -\gl{prop}-bad.news-\gl{prop} \gl{cop}-\gl{ipfv}-\gl{pl}//
        \glft ‘‘esos tienen mala seña’’//  
    \endgl 
\xe

\ex Yawarana \\
\label{poss-main-aff-locpred-cop-nsubj}    \begingl
    \glpreamble warë tëwïsantomo tinawë wejsapë puwakontomojne //
    \gla warë tëwïsantomo t-inawë wej-sapë puwa-kontomo-jne//
    \glb thus \gl{3}\gl{pl} \gl{3}-in.possesion \gl{cop}-\gl{perf} point-\gl{pl}-\gl{pl}//
        \glft ‘‘así ellos tenía púas’’//  
    \endgl 
\xe

\pex\label{poss-main-aff-npred-nsubj}    \a Yawarana\\
    \label{convinsectmaj-24}        \begingl
        \glpreamble ijtakemï, warë //
        \gla i-jta-ke-mï warë//
        \glb \gl{3}-foot-\gl{prop}-\gl{nzr} thus//
            \glft ‘‘tiene patas, así’’//  
        \endgl 
    \a Yawarana\\
    \label{ctorosq-47}        \begingl
        \glpreamble tesekemï tëwï //
        \gla t-ese-ke-mï tëwï//
        \glb -\gl{prop}-name-\gl{prop}-\gl{nzr} \gl{3}\gl{sg}//
            \glft ‘tiene nombre’//  
        \endgl 
\xe

\ex Yawarana \\
\label{poss-main-neg-advsubj-pirare-locpred}    \begingl
    \glpreamble uyepematojpe pïrarë wïrë inawë //
    \gla u-yepema-toj=pe pïrarë wïrë inawë//
    \glb \gl{1}\gl{sg}-pay-\gl{circ}=\gl{ess} nothing \gl{1}\gl{sg} in.possesion//
        \glft ‘‘yo no tengo para pagar’’//  
    \endgl 
\xe

\ex Yawarana \\
\label{poss-main-neg-part-pinire-nsubj}    \begingl
    \glpreamble entë ejnë pata pïnirë, entë //
    \gla entë ejnë pata-Ø -pïnirë entë//
    \glb here \gl{1+2} town-\gl{poss} -\gl{neg} here//
        \glft ‘‘aqui no es nuestra tierra, aquí’’//  
    \endgl 
\xe

\pex\label{poss-main-q-locpred-cop-nsubj}    \a Yawarana\\
    \label{convamgu-325}        \begingl
        \glpreamble ¿ asanë inawë ka wejsapë, tëwï? //
        \gla asanë inawë ka wej-sapë tëwï//
        \glb \gl{2}:mother:\gl{pos} in.possesion \gl{qp} \gl{cop}-\gl{perf} \gl{3}\gl{sg}//
            \glft ‘‘¿tu mamá tenía eso?’’//  
        \endgl 
    \a Yawarana\\
    \label{histyarirdi-827}        \begingl
        \glpreamble më inawë wejsapë enijpëtërë, ¿ ëjka makë? //
        \gla më- inawë wej-sapë enijpëtërë ëjka makë//
        \glb \gl{2}- in.possesion \gl{cop}-\gl{perf} one ¿right? mother:\gl{voc}//
            \glft ‘‘¿tu tenías uno, verdad, mamá?’//  
        \endgl 
\xe

\chapter{\texorpdfstring{Simple verbal clauses
\label{simpleverb}}{Simple verbal clauses }}

\todo{examples of [non]declarative w/ [in]transitive verbs}

\begin{itemize}
\tightlist
\item
  order of arguments re: the verb (and each other)
\item
  case marking patterns
\item
  indexation
\item
  clausal particles
\end{itemize}

\chapter{\texorpdfstring{Questions \label{questions}}{Questions }}

\chapter{\texorpdfstring{Multiclausal
\label{multiclausal}}{Multiclausal }}

\begin{itemize}
\item
  argument of the matrix clause
\item
  adverbial adjunct
\item
  relative clause
\item
  differences \& similarities to simple verb clauses?
\item
  order of arguments re: the verb (and each other)
\item
  case marking patterns
\item
  indexation
\item
  clausal particles
\item
  +mapping between matrix and subordinate
\end{itemize}

\chapter{\texorpdfstring{Word order variation
\label{wordorder}}{Word order variation }}

\chapter{\texorpdfstring{Pragmatically marked constructions
\label{pragmarked}}{Pragmatically marked constructions }}

\begin{itemize}
\tightlist
\item
  participant nominalizations for pseudo-clefts
\end{itemize}

\chapter{\texorpdfstring{Detransitive voice
\label{voice}}{Detransitive voice }}

\begin{itemize}
\tightlist
\item
  functions of \gl{detrz}:

  \begin{itemize}
  \tightlist
  \item
    antipassive
  \item
    passive
  \item
    reflexive
  \item
    reciprocal
  \item
    anticausative
  \end{itemize}
\item
  other strategies for removing participant:

  \begin{itemize}
  \tightlist
  \item
    \emph{-se-mï} `gnomic'
  \item
    \obj{-në} `\gl{inf}'
  \end{itemize}
\item
  what is not used for voice?

  \begin{itemize}
  \tightlist
  \item
    \emph{-sapë}
  \item
    participle
  \end{itemize}
\end{itemize}

\printbibliography

\end{document}