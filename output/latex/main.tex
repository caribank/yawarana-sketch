\documentclass{memoir}
\setsecnumdepth{subsubsection}
\usepackage{tikz}
\usepackage{geometry}
\usepackage{xcolor}
\usepackage[some]{background}

\definecolor{titlepagecolor}{RGB}{208, 84, 0}

\backgroundsetup{
scale=1,
angle=0,
opacity=1,
contents={\begin{tikzpicture}[remember picture,overlay]
 \path [fill=titlepagecolor] (-0.5\paperwidth,5) rectangle (0.5\paperwidth,10);  
\end{tikzpicture}}
}

\font\hugefont="Brill" at 38pt

\usepackage{fontspec}
\setmainfont{Brill}
\usepackage[abbrevs=none,refmode=latex]{expex-acro}
\usepackage{booktabs}
\usepackage[style=authoryear]{biblatex}
\usepackage[textwidth=30mm]{todonotes}
\def\tightlist{}
\usepackage{longtable}
\usepackage{hyperref}
\usepackage[capitalise]{cleveref}

\lingset{everygla=\itshape, belowglpreambleskip=0ex, aboveglftskip=0ex}

\title{A digital sketch grammar of Yawarana}
\author{Florian Matter;Natalia Cáceres Arandia;Spike Gildea}

\newGlossingAbbrev{pl}{plural}
\newGlossingAbbrev{sg}{singular}
\newGlossingAbbrev{detrz}{detransivizer}
\newGlossingAbbrev{pst}{past}
\newGlossingAbbrev{1}{first person}
\newGlossingAbbrev{2}{2 second person}
\newGlossingAbbrev{3}{3 third person}
\newGlossingAbbrev{1+2}{first person inclusive}
\newGlossingAbbrev{1+3}{first person exclusive}
\newGlossingAbbrev{lk}{linker}
\newGlossingAbbrev{loc}{locative}
\newGlossingAbbrev{ipfv}{imperfective}
\newGlossingAbbrev{anim}{animate}
\newGlossingAbbrev{inan}{inanimate}
\newGlossingAbbrev{prox}{proximal}
\newGlossingAbbrev{dist}{distal}
\newGlossingAbbrev{poss}{possession}
\newGlossingAbbrev{nposs}{nonpossessed}
\newGlossingAbbrev{in}{inanimate}
\newGlossingAbbrev{md}{medial}
\newGlossingAbbrev{emp}{emphatic}
\newGlossingAbbrev{pert}{pertensive}
\newGlossingAbbrev{quot}{quotative}
\newGlossingAbbrev{cop}{copula}
\newGlossingAbbrev{pfv}{perfective}
\newGlossingAbbrev{vbz}{verbalizer}
\newGlossingAbbrev{ints}{intensifier}
\newGlossingAbbrev{all}{allative}
\newGlossingAbbrev{plur}{pluractional}
\newGlossingAbbrev{npert}{unpossessed}
\newGlossingAbbrev{intr}{intransitive}
\newGlossingAbbrev{tr}{transitive}
\newGlossingAbbrev{neg}{negation}
\newGlossingAbbrev{proh}{prohibitive}
\newGlossingAbbrev{fut}{future}
\newGlossingAbbrev{concl}{conclusive}
\newGlossingAbbrev{imp}{imperative}
\newGlossingAbbrev{mot}{motion}

\begin{document}

\begin{titlingpage}
\BgThispage
\newgeometry{left=1cm,right=1cm}
\vspace*{2cm}
\centering
\textcolor{white}{ \hugefont A digital sketch grammar of Yawarana }
\vspace*{3cm}\par
\noindent
{
\raggedleft
\begin{minipage}{0.90\linewidth}
    \begin{flushright}
        
{\Huge Florian Matter }\\[\baselineskip]

{\Huge Natalia Cáceres Arandia }\\[\baselineskip]

{\Huge Spike Gildea }\\[\baselineskip]

    \end{flushright}
\end{minipage} \hspace{15pt}
}
\centering
\vfill
\rule{0.4\textwidth}{0.4pt}\\
{\Huge 2023 \\ \large pylingdocs }
\end{titlingpage}


\tableofcontents

\chapter{\texorpdfstring{Introduction \label{intro}}{Introduction }}

\section{\texorpdfstring{The Yawarana people and their language
\label{sec:people}}{The Yawarana people and their language }}

\section{\texorpdfstring{Location, historical records
\label{sec:context}}{Location, historical records }}

\section{\texorpdfstring{Current life
\label{sec:currentlife}}{Current life }}

\section{\texorpdfstring{Sociolinguistic vitality
\label{sec:vitality}}{Sociolinguistic vitality }}

\section{\texorpdfstring{Previous studies on the Yawarana language
\label{sec:previous}}{Previous studies on the Yawarana language }}

\section{\texorpdfstring{This project
\label{sec:thisproject}}{This project }}

\chapter{\texorpdfstring{Phonetics and phonology
\label{phono}}{Phonetics and phonology }}

\section{\texorpdfstring{Segmental phonetics and phonemes
\label{sec:segmental}}{Segmental phonetics and phonemes }}

The consonant phonemes of Yawarana are shown in \cref{tab:consonants},
and the vowels in \cref{tab:vowels}. This is a fairly standard Cariban
phoneme inventory, only departing from the mainstream by the addition of
/t͡ʃ/.

\begin{table}
\caption{Consonant phonemes}
\label{tab:consonants}
\centering
\begin{tabular}{llllll}
\toprule
          & bilabial & alveolar & palatal & velar & glottal \\
\midrule
occlusive &     /p/  &     /t/  &  /t͡ʃ/  &   /k/ &         \\
    nasal &     /m/  &     /n/  &    /ɲ/  &       &         \\
fricative &          &     /s/  &         &       &    /h/  \\
   liquid &          &     /r/  &         &       &         \\
    glide &     /w/  &          &     /j/ &       &         \\
\bottomrule
\end{tabular}

\end{table}

\begin{table}
\caption{Vowel phonemes}
\label{tab:vowels}
\centering
\begin{tabular}{llll}
\toprule
      & front & central & back \\
\midrule
close &  /i/  &    /ɨ/  & /u/  \\
  mid &  /e/  &    /ə/  & /o/  \\
 open &       &    /a/  &      \\
\bottomrule
\end{tabular}

\end{table}

\subsection{\texorpdfstring{Consonants
\label{sec:consonants}}{Consonants }}

\subsection{\texorpdfstring{Vowels \label{sec:vowels}}{Vowels }}

\section{\texorpdfstring{Morphophonological Processes
\label{sec:morphophono}}{Morphophonological Processes }}

\subsection{\texorpdfstring{Syllable Reduction
\label{sec:sylred}}{Syllable Reduction }}

\subsection{\texorpdfstring{Vowel harmony?
\label{sec:vowelharm}}{Vowel harmony? }}

\section{\texorpdfstring{Prosody \label{sec:prosody}}{Prosody }}

\subsection{\texorpdfstring{Lexical stress
\label{sec:stress}}{Lexical stress }}

\subsection{\texorpdfstring{Intonational Phrases
\label{sec:intphrases}}{Intonational Phrases }}

\subsection{\texorpdfstring{Intonational Melodies
\label{sec:intmelodies}}{Intonational Melodies }}

\section{\texorpdfstring{Historical Considerations
\label{sec:histphono}}{Historical Considerations }}

\chapter{\texorpdfstring{Distinguishing parts of speech in Yawarana
\label{POS}}{Distinguishing parts of speech in Yawarana }}

\chapter{\texorpdfstring{Nouns \label{nouns}}{Nouns }}

\section{\texorpdfstring{Pronouns \label{sec:pronouns}}{Pronouns }}

The personal pronouns of Yawarana are shown in \cref{tab:pronouns}. It
shows the usual Cariban inclusive/exclusive (\gl{1+2} and \gl{1+3})
distinction. Note that the plural marker \obj{-kontomo} appears to
usually be restricted to verbs, while \emph{-santomo} is found with
third person pronouns and demonstratives.

\begin{table}
\caption{Pronouns}
\label{tab:pronouns}
\centering
\begin{tabular}{lll}
\toprule
         &         sg &                pl \\
\midrule
  \gl{1} & \obj{wïrë} &                   \\
\gl{1+2} &            &        \obj{ejnë} \\
\gl{1+3} &            &         \obj{ana} \\
  \gl{2} & \obj{mërë} &   \obj{mokontomo} \\
  \gl{3} & \obj{tëwï} & \obj{tëwïsantomo} \\
\bottomrule
\end{tabular}

\end{table}

Short forms of the first and second person pronouns can occur as
proclitics attaching to nouns to indicate possessor
(\cref{sec:nominalperson}), attached to verbs to indicate subject or
object (described in \cref{verbinfl}), or attached to postpositions to
indicate object of the postposition (described in \cref{sec:postinfl}):

\ex Yawarana \\
\label{convrisamaj-28}    \textit{uyïwïj yawë usenejkari sukuri jwama }\\
        ‘yo me quedo en mi casa tranquila’ \xe

\ex Yawarana \\
\label{desccasmaj-025}    \textit{mënai wëjkase chijpë wararë }\\
        ‘se cayó tu cosa’ \xe

\ex Yawarana \\
\label{convrisamaj-02}    \begingl
    \glpreamble  mëyaruwari, mëpëkëpene //
    \gla më-yaruwa-ri më-pëkëpene//
    \glb \gl{2}-laugh-\gl{ipfv} \gl{2}-alone//
        \glft ‘You just laugh.’//  
    \endgl 
\xe

An open question is whether \obj{ta-} on verbs is a reduction of
\obj{tëwï}.

The third person demonstrative pronouns or articles are shown in
\cref{tab:pronouns3}.

\begin{table}
\caption{Demonstrative pronouns / articles}
\label{tab:pronouns3}
\centering
\begin{tabular}{lllll}
\toprule
              & \multicolumn{2}{l}{anim} & \multicolumn{2}{l}{inan} \\
\midrule
              &          sg &                 pl &         sg &                        pl \\
         prox &  \obj{kërë} & \emph{kërësantomo} &  \obj{eni} &  \obj{eni}-\emph{santomo} \\
medial? near? & \obj{michi} &                    & \obj{mërë} & \obj{mërë}-\emph{santomo} \\
         dist &  \obj{mëkï} &  \obj{mëkïsantomo} & \obj{mënï} & \obj{mënï}-\emph{santomo} \\
\bottomrule
\end{tabular}

\end{table}

None of the demonstrative pronouns have shortened, phonologically bound
counterparts.

\begin{itemize}
\tightlist
\item
  Nominal Interrogative pronouns:

  \begin{itemize}
  \tightlist
  \item
    \obj{aniki} `who? anim'
  \item
    \obj{ati} `what? inan'
  \item
    \emph{ëjkë} `which? inan'
  \end{itemize}
\end{itemize}

\section{\texorpdfstring{Nominal inflection
\label{sec:nouninfl}}{Nominal inflection }}

Nouns in Yawarana may bear suffixes for possession
(\cref{sec:nounposssuf}) and number (\cref{sec:nominalnumber}), and
possessed nouns may bear a third person prefix, indexing a third person
possessor, or a first or second proclitic (a reduced form of the free
pronoun), indexing a first or second person possessor
(\cref{sec:nominalperson}).

\subsection{\texorpdfstring{Suffixes for possessed (poss) and
non-possessed (nposs) nouns
\label{sec:nounposssuf}}{Suffixes for possessed (poss) and non-possessed (nposs) nouns }}

In the possession construction in Yawarana, the possessor noun occurs
immediately preceding the possessed noun, which is the head of the
possession phrase. Alternatively, the possessor can appear as a bound
pronominal clitic (first \& second person) or a prefix (third person) on
the possessed noun. The possessor noun is never marked (for instance,
with genitive case), but the possessed noun (the head) is often marked
by a lexically specified `possessed' suffix, either \obj{-ru} `pert' or
\obj{-ti} `pos'. Unpossessed nouns generally are unmarked, but some 15
nouns bear the suffix \obj{-të} `npert' when they appear without a
possessor. Examples \exref[][unsuffixednouns]{onlypossessed} illustrate
the possible patterns of markedness for nouns when possessed and
unpossessed. In the first category, which contains the vast majority of
nouns in our corpus, the unpossessed noun is unmarked, but when
possessed the suffix -ri `pos' occurs \exref[]{onlypossessed}. A handful
of nouns is marked with -ri/-ti `pos' when possessed and with -të `npos'
when not possessed \exref[]{diffpossessed}. Another handful is unmarked
when possessed and marked with -të when not possessed
\exref[]{suffunpossessed}. The fourth logical possibility, in which the
noun bears no marker of possession (or non-possession) whether possessed
or unpossessed, contains very few members (only one attested so far); in
this case, the difference is marked only by the presence or absence of a
possessive prefix or free-form possessor \exref[]{unsuffixednouns}.

\ex\label{onlypossessed} Nouns that take a suffix only when possessed:

\begin{tabular}[t]{llll}

 \emph{akajra-ri} &          ‘X’s bow’ & \emph{akajra} &          ‘bow’ \\

\emph{y-amaka-ri} &        ‘X’s yucca’ &  \emph{amaka} &        ‘yucca’ \\
 \emph{y-ántë-ri} &     ‘X’s fishhook’ &   \emph{antë} &     ‘fishhook’ \\
\emph{y-ateri-ri} & ‘X’s garden/field’ &  \emph{ateri} & ‘garden/field’ \\
    \emph{ënu-ru} &          ‘X’s eye’ &    \emph{ënu} &          ‘eye’ \\
  \emph{y-ëpi-ri} &     ‘X’s medicine’ &    \emph{ëpi} &     ‘medicine’ \\

\end{tabular}
 \xe

\ex\label{diffpossessed} Nouns that take one suffix when possessed and
another when unpossessed

\begin{tabular}[t]{llll}

   \emph{yë-ri} & ‘X’s tooth’ &   \emph{yë-të} &                                ‘tooth’ \\

 \emph{pata-ri} & ‘X’s place’ & \emph{pata-të} & ‘(part of name) San Juan de Manapiare’ \\
\emph{y-ese-ti} & ‘X’s name’  &  \emph{ese-të} &                                 ‘name’ \\
\emph{y-ase-tï} & ‘X’s cord’  &  \emph{ase-të} &                                 ‘cord’ \\

\end{tabular}
 \xe

\ex\label{suffunpossessed} Nouns that take a suffix only when
unpossessed:

\begin{tabular}[t]{llll}

  \emph{yëjpë} &  ‘X’s bone’ &                \emph{yëjpë-të} &  ‘bone’ \\

   \emph{petï} & ‘X’s thigh’ & \emph{petï-të} / \emph{pej-të} & ‘thigh’ \\
\emph{y-aponi} & ‘X’s stool’ &                 \emph{apon-të} & ‘stool’ \\

\end{tabular}
 \xe

\ex\label{unsuffixednouns} Nouns that never take a suffix, whether
possessed or unpossessed:

\begin{tabular}[t]{llll}

\emph{i-jmëy} & 'his egg’ & \emph{ëjmëy} & 'egg’ \\

\end{tabular}
 \xe

\subsection{\texorpdfstring{Number suffixes
\label{sec:nominalnumber}}{Number suffixes }}

There are two plural suffixes that can occur on nouns, apparently freely
interchangeable. What conditions the choice of suffix is not clear as of
yet.

\ex Yawarana \\
\label{ctorat-17}    \textit{waijtatomo ëjwenakase }\\
        ‘se vomitaron las ratas’ \xe

\ex Yawarana \\
\label{ctorat-40}    \textit{tipapëjsejne waijtajne }\\
        ‘las ratas se fueron’ \xe

\subsection{\texorpdfstring{Argument prefixes
\label{sec:nominalperson}}{Argument prefixes }}

Person prefixes on nouns are conditioned by the initial segment
(\cref{tab:possprefixes}). C-initial nouns take third person \obj{i-},
and first and second person are bare \obj{u-} and \obj{më-}. On
V-initial nouns, third person is marked by \obj{t-}, and first and
second person combine with the linker \obj{y-}. Some examples are shown
in \exref[][ctorat-19]{ctorat-23}.

\begin{table}
\caption{Possessive prefixes on nouns}
\label{tab:possprefixes}
\centering
\begin{tabular}{lll}
\toprule
       &       \_C &               \_V \\
\midrule
\gl{1} &  \obj{u-} &  \obj{u-}\obj{y-} \\
\gl{2} & \obj{më-} & \obj{më-}\obj{y-} \\
\gl{3} &  \obj{i-} &          \obj{t-} \\
\bottomrule
\end{tabular}

\end{table}

\ex Yawarana \\
\label{ctorat-23}    \textit{aaa usukuru morone ta wïrë usujta ta ne }\\
        ‘me duele mis orines, voy a orinar’ \xe

\ex Yawarana \\
\label{convrisamaj-28}    \textit{uyïwïj yawë usenejkari sukuri jwama }\\
        ‘yo me quedo en mi casa tranquila’ \xe

\ex Yawarana \\
\label{desccasmaj-025}    \textit{mënai wëjkase chijpë wararë }\\
        ‘se cayó tu cosa’ \xe

\ex Yawarana \\
\label{ctorat-46}    \textit{tïwïj yaka waraijtokomo manikijpë }\\
        ‘se fue el hombre para su casa (porque ya amaneció)’ \xe

\ex Yawarana \\
\label{ctorat-19}    \textit{pïrarë ti iwenaru wejsapë }\\
        ‘no había su vómito’ \xe

The linker also occurs with (pro-)nominal possessors:

\ex Yawarana \\
\label{desccasmaj-131}    \textit{ejnë yemekunu }\\
        ‘la mano de uno’ \xe

There are some nouns (kinship terms?) that take an apparently older old
second person \obj{a-} (\cref{tab:oldpossprefixes}).

\begin{table}
\caption{Archaic possessive prefixes on nouns}
\label{tab:oldpossprefixes}
\centering
\begin{tabular}{lll}
\toprule
       &      \_C &              \_V \\
\midrule
\gl{1} & \obj{u-} & \obj{u-}\obj{y-} \\
\gl{2} & \obj{a-} & \obj{a-}\obj{y-} \\
\gl{3} & \obj{i-} &         \obj{t-} \\
\bottomrule
\end{tabular}

\end{table}

\section{\texorpdfstring{Nominal Derivational Morphology
\label{sec:nounderiv}}{Nominal Derivational Morphology }}

\begin{itemize}
\tightlist
\item
  V → N

  \begin{itemize}
  \item
    \obj{-ri} `act.nzr'
  \item
    \obj{-jpë}

    \begin{itemize}
    \tightlist
    \item
      `past.abs.nzr'

      \begin{itemize}
      \tightlist
      \item
        `past.act.nzr?'
      \end{itemize}
    \end{itemize}
  \item
    ?\obj{-në} `infinitive / generic action nominalizer'

    \begin{itemize}
    \tightlist
    \item
      Only intransitive verbs? no also \emph{wanumanë} `gossip, lie' and
      \emph{wajtënë} `dance'
    \end{itemize}
  \item
    \obj{-ni} `a.nzr'
  \item
    \obj{n-} `o.nzr'

    \begin{itemize}
    \tightlist
    \item
      \obj{n-}V\obj{-ri} `nonpast?'
    \item
      ?? \obj{n-}V\obj{-jpë} `past?'
    \end{itemize}
  \item
    \obj{-sapë} `abs.nzr' (contrast with ‑jpë )
  \item
    \obj{-topo} `circ.nzr'
  \item
    \obj{‑pïnï} `privative.nzr' ?
  \end{itemize}
\item
  Adv → N

  \begin{itemize}
  \tightlist
  \item
    \obj{-mï} `nzr'
  \end{itemize}
\item
  Postp → N

  \begin{itemize}
  \tightlist
  \item
    \obj{-ano} `nzr'
  \end{itemize}
\item
  What about \obj{-jpë} on AD forms? Does it derive a noun?
\end{itemize}

\subsection{Misc}

predicative negation of nominalized verb:

\ex Yawarana \\
\label{convrisamaj-52}    \textit{tari yarikasemïjra }\\
        ‘uy! no hay como para reir’ \xe

\chapter{\texorpdfstring{Verbal inflection
\label{verbinfl}}{Verbal inflection }}

\section{\texorpdfstring{Person prefixes
\label{sec:verbperson}}{Person prefixes }}

\begin{table}
\caption{Person marking prefixes on verbs}
\label{tab:verbprefixes}
\centering
\begin{tabular}{lll}
\toprule
       &      intr &        tr \\
\midrule
\gl{1} &  \obj{u-} &  \obj{u-} \\
\gl{2} & \obj{më-} & \obj{më-} \\
\gl{3} &         ∅ & \obj{ta-} \\
\bottomrule
\end{tabular}

\end{table}

\begin{itemize}
\tightlist
\item
  Absolutive proclitics

  \begin{itemize}
  \tightlist
  \item
  \item
    \obj{u-} `1S/O'
  \item
    \obj{më-} `2S/O'

    \begin{itemize}
    \tightlist
    \item
      one example of (më=) `2A' on imperative verb
    \end{itemize}
  \end{itemize}
\item
  Third person

  \begin{itemize}
  \tightlist
  \item
    Ø‑ `3S' with intransitive verbs
  \item
    • exception: \emph{ij‑të‑ri} `he goes' plus 2 more
  \item
    Ø‑ `3O' with transitive verbs with 1A or 2A; also sometimes 3A
  \item
    \obj{ta-} `3A3O'

    \begin{itemize}
    \tightlist
    \item
      Not required, but possible

      \begin{itemize}
      \tightlist
      \item
        Check: not in alternation with preceding O NP?
      \end{itemize}
    \item
      \obj{ta-} `3O' attested on one V in the pan‑Cariban
      ``progressive'' construction w/ 2nd person A
    \end{itemize}
  \end{itemize}
\item
  Note that all transitive verbs are consonant‑initial, whether
  etymologically or not because \obj{y-} `rel' is added to all
  vowel‑initial roots

  \begin{itemize}
  \tightlist
  \item
    the \emph{y‑} disappears when preceded by the detransitivizer
  \end{itemize}
\end{itemize}

\section{Non‑personal inflectional prefixation --- is there any?
Probably not?}

\begin{itemize}
\tightlist
\item
  t‑V‑se is no more --- the t‑ is gone (except with tënëse `eat' and
  tenise `drink')

  \begin{itemize}
  \tightlist
  \item
    Any divergent forms with the negative?

    \begin{itemize}
    \tightlist
    \item
      \emph{i‑} with intransitive negatives?
    \item
      \emph{an‑} with transitive negatives?
    \end{itemize}
  \end{itemize}
\end{itemize}

\section{Tense‑aspect‑mood‑polarity suffixes}

\begin{table}
\caption{Verbal TAM suffixes}
\label{tab:verbtam}
\centering
\begin{tabular}{ll}
\toprule
       Suffix &        Function \\
\midrule
    \obj{-ri} &    imperfective \\
   \obj{-jpë} &            past \\
    \obj{-se} &         past 2? \\
  \obj{-sapë} &     perfective? \\
  \obj{-sarë} & imminent future \\
\obj{-tëpëkë} &       prog.intr \\
   \obj{pëkë} &         prog.tr \\
  \obj{-sarë} & imminent future \\
\bottomrule
\end{tabular}

\end{table}

\begin{itemize}
\item
  \obj{-ja} `neg'
\item
  \emph{‑jrama} `proh'
\item
  \emph{‑tojpano} `fut'
\item
  \obj{-se}=\obj{pano} `pst=concl'
\item
  \obj{-saj}=\obj{pano} `pfv=concl'
\item
  imperatives:

  \begin{itemize}
  \tightlist
  \item
    \obj{-kë} / ‑\emph{të}\obj{-kë} `imp / imp.pl'
  \item
    \obj{-ta} / \obj{-ta}\emph{ntë}\obj{-kë} `imp.mot / imp.mot.pl'
  \end{itemize}
\end{itemize}

\subsection{\texorpdfstring{\obj{-ri}}{}}

\begin{itemize}
\tightlist
\item
  allomorphy:

  \begin{itemize}
  \tightlist
  \item
    \obj{-∅}, phonetic loss
  \item
    \obj{-ru}, assimilation
  \item
    what about \obj{-rï}? Looks like the original one\ldots{}
  \end{itemize}
\item
  diachrony: related to other \emph{-ri}
\item
  combines with \obj{-jra}:
\end{itemize}

\ex Yawarana \\
\label{convrisamaj-04}    \textit{wïrë yaruwarijra }\\
        ‘!‘yo no me río’ \xe

\subsubsection{Semantics}

not specified for tense, just imperfective aspect:

\ex Yawarana \\
\label{ctorat-16}    \textit{irëjpë tëwï waijtatomo nwajtëri }\\
        ‘después las ratas estaban bailando’ \xe

\ex Yawarana \\
\label{convrisamaj-06}    \textit{kwase ejnë yaruwari }\\
        ‘¿cómo vamos a reir?’ \xe

\ex Yawarana \\
\label{convrisamaj-04}    \textit{wïrë yaruwarijra }\\
        ‘!‘yo no me río’ \xe

\ex Yawarana \\
\label{convrisamaj-28}    \textit{uyïwïj yawë usenejkari sukuri jwama }\\
        ‘yo me quedo en mi casa tranquila’ \xe

\subsection{\texorpdfstring{\obj{-jpë}}{}}

\begin{itemize}
\tightlist
\item
  allomorphy: none?
\item
  diachrony: from other \emph{-jpë}
\end{itemize}

\subsection{\texorpdfstring{\obj{-se}}{}}

\begin{itemize}
\tightlist
\item
  allomorphy: \obj{-se}/\obj{-che}
\item
  diachrony: from participle
\end{itemize}

\subsection{\texorpdfstring{\obj{-sapë}}{}}

\begin{itemize}
\tightlist
\item
  diachrony and distribution: only occurs on the copula?
\item
  allomorphy: \obj{-sapë} and \obj{-saj}
\item
  negation: with \obj{-ja} on lexical verb
  \exref[][ctoaragrme-40]{ctoaragrme-38}
\end{itemize}

\ex Yawarana \\
\label{ctoaragrme-38}    \textit{irë wejtane mujyampe patakaja wejsapë }\\
        ‘a pesar de eso no salió embarazada’ \xe

\ex Yawarana \\
\label{ctoaragrme-39}    \textit{apatakaja pïnïka wejsapë }\\
        ‘tal vez no salió (embarazada)’ \xe

\ex Yawarana \\
\label{ctoaragrme-40}    \textit{tayakïjtëja pïnika wejsapë }\\
        ‘tal vez no se acostó con ella’ \xe

\begin{itemize}
\tightlist
\item
  what about \exref[]{ctorat-19}? is that existential negation?
\end{itemize}

\ex Yawarana \\
\label{ctorat-19}    \textit{pïrarë ti iwenaru wejsapë }\\
        ‘no había su vómito’ \xe

\subsection{\texorpdfstring{\obj{-sarë}}{}}

\ex Yawarana \\
\label{ctorat-25}    \textit{irëjpë ta ti ta konopo wejsarë konopo wejsarë }\\
        ‘después dijo, está lloviendo, está lloviendo!’ \xe

\ex Yawarana \\
\label{ctoaragrme-25}    \textit{moyochi tasarë moyochi chipokono kojpaye pïnika warotari }\\
        ‘le dicen araña, tal vez porque la araña trabaja de noche’ \xe

\section{Subordinate Clause markers}

\begin{itemize}
\tightlist
\item
  Nominalizations
\item
  Adverbial Clauses

  \begin{itemize}
  \tightlist
  \item
    \obj{-se} `supine'
  \item
    \obj{-tojpe} `purpose'
  \item
    (‑jpë)=tërë `after'
  \item
    \obj{-tane} `concessive'
  \item
    \obj{-sarë} `converb'
  \item
    \obj{yawë} `simult'
  \item
    \emph{‑yapo} `neg.purp'
  \item
    others?
  \end{itemize}
\end{itemize}

\section{Number}

\begin{itemize}
\tightlist
\item
  \emph{‑rï=kontomo}
\item
  \emph{‑se=jne=kontomo}
\item
  \emph{‑se=jne=pano} (\emph{‑se=jne=kontom=pano}?)
\end{itemize}

\section{Copula / Auxiliary}

\begin{itemize}
\tightlist
\item
  Paradigm
\item
  Any particles? Man, wai, manai, etc?
\item
  Past/Perfect particles? nahkë, etc.
\item
  chijpë, wejsapë
\end{itemize}

\chapter{\texorpdfstring{Verbal roots and stems
\label{derbderiv}}{Verbal roots and stems }}

\chapter{\texorpdfstring{Adverbs \label{adverbs}}{Adverbs }}

\section{Inflection}

\begin{itemize}
\tightlist
\item
  presumably no prefixation
\item
  negation:
\end{itemize}

\ex Yawarana \\
\label{convrisamaj-52}    \textit{tari yarikasemïjra }\\
        ‘uy! no hay como para reir’ \xe

\section{\texorpdfstring{Simple adverbs
\label{sec:simpleadv}}{Simple adverbs }}

\section{\texorpdfstring{Derived adverbs
\label{sec:derivedadv}}{Derived adverbs }}

\chapter{\texorpdfstring{Postpositions \label{postp}}{Postpositions }}

\section{Defining the category}

\section{\texorpdfstring{Inflectional morphology
\label{sec:postinfl}}{Inflectional morphology }}

Postpositions take the same inflectional prefixes as nouns
(\cref{sec:nounposssuf}).

\begin{tabular}[t]{ll}

       \\

\gl{1} &     \obj{u-} \\
\gl{2} &    \obj{më-} \\
\gl{3} & \obj{i-/t-}? \\

\end{tabular}

\section{Locative Postpositions}

\begin{itemize}
\tightlist
\item
  Clear bipartite Ground+Path
\item
  Unproductive Bipartite X+Path?
\item
  Other forms
\end{itemize}

\begin{table}
\caption{Locative postpositions}
\label{tab:locpost}
\centering
\begin{tabular}{lll}
\toprule
        &         all &         loc \\
\midrule
 inside &  \obj{yaka} &  \obj{yawë} \\
aquatic & \obj{jwaka} & \obj{jwawë} \\
\bottomrule
\end{tabular}

\end{table}

\begin{itemize}
\tightlist
\item
  \emph{poye}
\item
  \emph{po}
\item
  \emph{yatë}
\item
  \emph{yapo}
\end{itemize}

\ex Yawarana \\
\label{histpajirdi-186}    \textit{tichikimuru peti warë patakasapë yakucho pana }\\
        ‘su rodilla, su pierna, salió (llaga) hacia Ayacucho’ \xe

\section{Nonlocative Oblique Postpositions}

\begin{itemize}
\tightlist
\item
  ya `erg'
\item
  ke `instr'
\item
  wanai
\item
  etc.
\end{itemize}

\section{Propositional Postpositions}

\begin{itemize}
\tightlist
\item
  =se `desiderative'
\item
  others?
\end{itemize}

\chapter{\texorpdfstring{Particles and Ideophones
\label{partideo}}{Particles and Ideophones }}

\chapter{\texorpdfstring{Phrases \label{phrases}}{Phrases }}

\chapter{\texorpdfstring{Nonverbal predications
\label{nonverbal}}{Nonverbal predications }}

\chapter{\texorpdfstring{Simple verbal clauses
\label{simpleverb}}{Simple verbal clauses }}

\chapter{\texorpdfstring{Negation \label{negation}}{Negation }}

\chapter{\texorpdfstring{Questions \label{questions}}{Questions }}

\chapter{\texorpdfstring{Multiclausal
\label{multiclausal}}{Multiclausal }}

\chapter{\texorpdfstring{Word order variation
\label{wordorder}}{Word order variation }}

\chapter{\texorpdfstring{Pragmatically marked constructions
\label{marked}}{Pragmatically marked constructions }}

\chapter{For testing purposes}

\section{Inline linked entities}

\subsection{Single}

\begin{enumerate}
\def\labelenumi{\arabic{enumi}.}
\tightlist
\item
  morph: \obj{-jne}
\item
  morpheme: \obj{-jnë}
\item
  wordform: (asamo-0-ri-jraneg)
\item
  text: ``Historia personal por AnFo''
\end{enumerate}

\subsection{Multiple}

\begin{enumerate}
\def\labelenumi{\arabic{enumi}.}
\tightlist
\item
  morph: \obj{-jne}, and \obj{-i}
\item
  morpheme: \obj{-jnë}, and \obj{-ru}
\item
  wordform: (asamo-0-ri-jraneg,asamo-0-ri-zero)
\item
  text:
\end{enumerate}

\section{Examples}

\subsection{Interlinear}

\ex Yawarana \\
\label{ctorat-34}    \begingl
    \glpreamble  ëkëtë mërë ëkï //
    \gla ëkëtë më-rë ëkï//
    \glb where \gl{2}-\gl{emp} manioc.beer//
        \glft ‘Where is the chicha?’//  
    \endgl 
\xe

\pex\label{}    \a Yawarana\\
    \label{ctorat-35}        \textit{ëkï ta rë pïrarë wenarujpë ta rë pïrarë }\\
            ‘el yaraki no había y tampoco el vómito’     \a Yawarana\\
    \label{ctorat-36}        \textit{ta ti wejsaj ti tëwï }\\
            ‘dijo él’ \xe

\subsection{Other}

\ex\label{test1}\begin{tabular}[t]{ll}

\obj{konopo} &   root \\

   \obj{-se} & suffix \\

\end{tabular}
 \xe

\pex\label{multiparttest} \a\label{test2} Hello \obj{-se} `PST'
\a\label{test3} World

\emph{and} more \xe

\subsection{Example references}

\exref[]{ctorat-34}

\exref[]{multiigt} or \exref[]{ctorat-36} or even \exref[a-b]{multiigt}

\exref[]{test1}

\exref[]{multiparttest}

\exref[][multiparttest]{test1}

\printbibliography

\end{document}