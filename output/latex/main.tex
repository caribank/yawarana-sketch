\documentclass{memoir}
\setsecnumdepth{subsubsection}
\usepackage{tikz}
\usepackage{geometry}
\usepackage{xcolor}
\usepackage[some]{background}

\definecolor{titlepagecolor}{RGB}{208, 84, 0}

\backgroundsetup{
scale=1,
angle=0,
opacity=1,
contents={\begin{tikzpicture}[remember picture,overlay]
 \path [fill=titlepagecolor] (-0.5\paperwidth,5) rectangle (0.5\paperwidth,10);  
\end{tikzpicture}}
}

\font\hugefont="Brill" at 38pt

\usepackage{fontspec}
\setmainfont{Brill}
\usepackage[abbrevs=none,refmode=latex]{expex-acro}
\usepackage{booktabs}
\usepackage[style=authoryear]{biblatex}
\usepackage[textwidth=30mm]{todonotes}
\def\tightlist{}
\usepackage{longtable}
\usepackage{hyperref}
\usepackage[capitalise]{cleveref}

\addbibresource{sources.bib}
\lingset{everygla=\itshape, belowglpreambleskip=0ex, aboveglftskip=0ex}

\title{A digital sketch grammar of Yawarana}
\author{Florian Matter;Natalia Cáceres Arandia;Spike Gildea}

\newGlossingAbbrev{1}{first person}
\newGlossingAbbrev{2}{second person}
\newGlossingAbbrev{3}{third person}
\newGlossingAbbrev{1+2}{first person inclusive}
\newGlossingAbbrev{1+3}{first person exclusive}
\newGlossingAbbrev{a}{agent-like argument}
\newGlossingAbbrev{abs}{absolutive}
\newGlossingAbbrev{all}{allative}
\newGlossingAbbrev{anim}{animate}
\newGlossingAbbrev{circ}{circumstantive}
\newGlossingAbbrev{cncs}{concessive}
\newGlossingAbbrev{concl}{conclusive}
\newGlossingAbbrev{contrast}{contrastive}
\newGlossingAbbrev{cop}{copula}
\newGlossingAbbrev{dat}{dative}
\newGlossingAbbrev{dem}{demonstrative}
\newGlossingAbbrev{des}{desiderative}
\newGlossingAbbrev{detrz}{detransivizer}
\newGlossingAbbrev{dim}{diminutive}
\newGlossingAbbrev{dist}{distal}
\newGlossingAbbrev{emp}{emphatic}
\newGlossingAbbrev{erg}{ergative}
\newGlossingAbbrev{ess}{essive}
\newGlossingAbbrev{fut}{future}
\newGlossingAbbrev{hsy}{hearsay evidentiality}
\newGlossingAbbrev{imn}{imminent}
\newGlossingAbbrev{imp}{imperative}
\newGlossingAbbrev{inan}{inanimate}
\newGlossingAbbrev{inf}{infinitive}
\newGlossingAbbrev{ins}{instrumental}
\newGlossingAbbrev{intr}{intransitive}
\newGlossingAbbrev{ints}{intensifier}
\newGlossingAbbrev{ipfv}{imperfective}
\newGlossingAbbrev{lk}{linker}
\newGlossingAbbrev{loc}{locative}
\newGlossingAbbrev{med}{medial}
\newGlossingAbbrev{mot}{motion}
\newGlossingAbbrev{motimp}{motion imperative}
\newGlossingAbbrev{neg}{negation}
\newGlossingAbbrev{nmlz}{nominalizer}
\newGlossingAbbrev{npert}{unpossessed}
\newGlossingAbbrev{nposs}{nonpossessed}
\newGlossingAbbrev{p}{patient-like argument}
\newGlossingAbbrev{pert}{pertensive}
\newGlossingAbbrev{pfv}{perfective}
\newGlossingAbbrev{pl}{plural}
\newGlossingAbbrev{plur}{pluractional}
\newGlossingAbbrev{poss}{possession}
\newGlossingAbbrev{priv}{privative}
\newGlossingAbbrev{pro}{pronoun}
\newGlossingAbbrev{prob}{probabilitive}
\newGlossingAbbrev{prog}{progressive}
\newGlossingAbbrev{proh}{prohibitive}
\newGlossingAbbrev{prox}{proximal}
\newGlossingAbbrev{pst}{past}
\newGlossingAbbrev{quot}{quotative}
\newGlossingAbbrev{rst}{restrictive}
\newGlossingAbbrev{s}{intransitive argument}
\newGlossingAbbrev{sg}{singular}
\newGlossingAbbrev{tr}{transitive}
\newGlossingAbbrev{vbz}{verbalizer}
\newGlossingAbbrev{acnnmlz}{action nominalizer}
\newGlossingAbbrev{acnmlz}{action nominalizer}
\newGlossingAbbrev{agtnmlz}{agent nominalizer}
\newGlossingAbbrev{ptcp}{participle}
\newGlossingAbbrev{sup}{supine}
\newGlossingAbbrev{purp}{purposive}
\newGlossingAbbrev{cvb}{converb}
\newGlossingAbbrev{advz}{adverbializer}
\newGlossingAbbrev{nzr}{nominalizer}
\newGlossingAbbrev{gno}{gnomic}
\newGlossingAbbrev{ctmp}{contemporative}
\newGlossingAbbrev{cond}{conditional}
\newGlossingAbbrev{perl}{perlative}
\newGlossingAbbrev{rel}{relativizer}
\newGlossingAbbrev{juss}{jussive}
\newGlossingAbbrev{ctrf}{counterfactive}
\newGlossingAbbrev{cess}{cessative}
\newGlossingAbbrev{ad}{ad-form}
\newGlossingAbbrev{postp}{postposition}
\newGlossingAbbrev{aux}{auxiliary}
\newGlossingAbbrev{adv}{adverb}
\newGlossingAbbrev{prop}{proprietive}
\newGlossingAbbrev{ana}{anaphoric}
\newGlossingAbbrev{mod}{modal}
\newGlossingAbbrev{com}{comitative}
\newGlossingAbbrev{pos}{possessed}
\newGlossingAbbrev{emph}{emphatic}
\newGlossingAbbrev{restr}{restrictive}
\newGlossingAbbrev{cnfrm}{confirmative}
\newGlossingAbbrev{md}{medial}
\newGlossingAbbrev{an}{animate}
\newGlossingAbbrev{px}{proximal}
\newGlossingAbbrev{in}{inanimate}
\newGlossingAbbrev{perf}{perfective}
\newGlossingAbbrev{inm}{immediate}
\newGlossingAbbrev{voc}{vocative}
\newGlossingAbbrev{o}{object}
\newGlossingAbbrev{rm}{remote?}
\newGlossingAbbrev{sit}{situational?}
\newGlossingAbbrev{np}{noun phrase}
\newGlossingAbbrev{pred}{predicative}
\newGlossingAbbrev{subj}{subject}
\newGlossingAbbrev{part}{particle}
\newGlossingAbbrev{psn}{person?}
\newGlossingAbbrev{caus}{causative}
\newGlossingAbbrev{cntr}{unknown abbreviation}
\newGlossingAbbrev{q}{question particle/marker}
\newGlossingAbbrev{exist}{unknown abbreviation}
\newGlossingAbbrev{eval}{unknown abbreviation}
\newGlossingAbbrev{obl}{oblique}
\newGlossingAbbrev{fmr}{unknown abbreviation}
\newGlossingAbbrev{post}{unknown abbreviation}
\newGlossingAbbrev{pol}{unknown abbreviation}
\newGlossingAbbrev{foc}{focus}
\newGlossingAbbrev{n}{neuter}
\newGlossingAbbrev{fin}{unknown abbreviation}
\newGlossingAbbrev{qp}{unknown abbreviation}
\newGlossingAbbrev{instr}{unknown abbreviation}
\newGlossingAbbrev{pst.abs.nmlz}{unknown abbreviation}
\newGlossingAbbrev{3p}{unknown abbreviation}
\newGlossingAbbrev{1a}{unknown abbreviation}
\newGlossingAbbrev{2a}{unknown abbreviation}
\newGlossingAbbrev{np~subj~}{unknown abbreviation}

\begin{document}

\begin{titlingpage}
\BgThispage
\newgeometry{left=1cm,right=1cm}
\vspace*{2cm}
\centering
\textcolor{white}{ \hugefont A digital sketch grammar of Yawarana }
\vspace*{3cm}\par
\noindent
{
\raggedleft
\begin{minipage}{0.90\linewidth}
    \begin{flushright}
        
{\Huge Florian Matter }\\[\baselineskip]

{\Huge Natalia Cáceres Arandia }\\[\baselineskip]

{\Huge Spike Gildea }\\[\baselineskip]

    \end{flushright}
\end{minipage} \hspace{15pt}
}
\centering
\vfill
\rule{0.4\textwidth}{0.4pt}\\
{\Huge 2023 \\ \large pylingdocs }
\end{titlingpage}


\tableofcontents

\chapter{\texorpdfstring{Introduction \label{intro}}{Introduction }}

\section{\texorpdfstring{The Yawarana people and their language
\label{sec:people}}{The Yawarana people and their language }}

\section{\texorpdfstring{Location, historical records
\label{sec:context}}{Location, historical records }}

\section{\texorpdfstring{Current life
\label{sec:currentlife}}{Current life }}

\section{\texorpdfstring{Sociolinguistic vitality
\label{sec:vitality}}{Sociolinguistic vitality }}

\section{\texorpdfstring{Previous studies on the Yawarana language
\label{sec:previous}}{Previous studies on the Yawarana language }}

\section{\texorpdfstring{This project
\label{sec:thisproject}}{This project }}

\section{\texorpdfstring{Variation \label{sec:variation}}{Variation }}

There are a number of corners of Yawarana grammar that are subject to
variation:

\begin{itemize}
\tightlist
\item
  plural marking (\cref{sec:nominalnumber}, \cref{sec:verbalnumber},
  \cref{postp})
\item
  constituent order (\cref{wordorder})
\item
  expression of verbal arguments (\cref{simpleverb},
  \cref{sec:nominalperson}, \cref{sec:verbperson})
\item
  presence or absence of the ergative marker \obj{ya} `\gl{erg}'
  (\cref{simpleverb})
\end{itemize}

\chapter{\texorpdfstring{Phonetics and phonology
\label{phono}}{Phonetics and phonology }}

\section{\texorpdfstring{Segmental phonetics and phonemes
\label{sec:segmental}}{Segmental phonetics and phonemes }}

The consonant phonemes of Yawarana are shown in \cref{tab:consonants},
vowel phonemes in \cref{tab:vowels}.

\begin{table}
\caption{Consonant phonemes}
\label{tab:consonants}
\centering
\begin{tabular}{llllll}
\toprule
          & bilabial & alveolar & palatal & velar & glottal \\
\midrule
occlusive &     /p/  &     /t/  &  /t͡ʃ/  &   /k/ &         \\
    nasal &     /m/  &     /n/  &    /ɲ/  &       &         \\
fricative &          &     /s/  &         &       &    /h/  \\
   liquid &          &     /r/  &         &       &         \\
    glide &     /w/  &          &     /j/ &       &         \\
\bottomrule
\end{tabular}

\end{table}

\begin{table}
\caption{Vowel phonemes}
\label{tab:vowels}
\centering
\begin{tabular}{llll}
\toprule
      & front & central & back \\
\midrule
close &  /i/  &    /ɨ/  & /u/  \\
  mid &  /e/  &    /ə/  & /o/  \\
 open &       &    /a/  &      \\
\bottomrule
\end{tabular}

\end{table}

\subsection{\texorpdfstring{Consonants
\label{sec:consonants}}{Consonants }}

\begin{itemize}
\tightlist
\item
  minimal pairs
\end{itemize}

\subsubsection{/h/}

\begin{itemize}
\tightlist
\item
  glottal fricative insertion after diphthongs
\item
  glottal fricative insertion before occlusives
\end{itemize}

\subsection{\texorpdfstring{Vowels \label{sec:vowels}}{Vowels }}

\begin{itemize}
\item
  minimal pairs
\item
  vowel plots
\item
  what about vowel length?
\item
  variation in \emph{ë}/\emph{o}/\emph{e} and \emph{ï}/\emph{i}/\emph{u}
\item
  dipththongs

  \begin{itemize}
  \tightlist
  \item
    /ai/, /aw/, /ei/\ldots{} test combinations
  \item
    {[}todo: au or aw? ai or ay?{]}
  \end{itemize}
\end{itemize}

\section{\texorpdfstring{Morphophonological Processes
\label{sec:morphophono}}{Morphophonological Processes }}

\subsection{\texorpdfstring{Syllable Reduction
\label{sec:sylred}}{Syllable Reduction }}

\begin{itemize}
\tightlist
\item
  V1rV2 to V1:
\item
  nasal assimilation
\end{itemize}

\subsubsection{Contexts}

\begin{itemize}
\item
  \gl{postp}
\item
  verbal suffixes
\item
  no final nominal reduction
\end{itemize}

\subsubsection{Non-alternating reduced syllables}

e.g.~\obj{wajto} `fire'

\subsection{\texorpdfstring{Vowel harmony
\label{sec:vowelharm}}{Vowel harmony }}

\begin{itemize}
\tightlist
\item
  progressive \obj{-ri} `\gl{pert}'
\item
  regressive /ë/ \textgreater{} /o/
\end{itemize}

\subsection{\texorpdfstring{Palatalization
\label{sec:palatalization}}{Palatalization }}

\begin{itemize}
\tightlist
\item
  \obj{-sapë} `\gl{pfv}'
\item
  \obj{-se} `\gl{pst}'
\end{itemize}

\section{\texorpdfstring{Prosody \label{sec:prosody}}{Prosody }}

\subsection{\texorpdfstring{Lexical stress
\label{sec:stress}}{Lexical stress }}

\subsection{\texorpdfstring{Intonational Phrases
\label{sec:intphrases}}{Intonational Phrases }}

{[}todo: f0 increase associated w/ pitch reset, clause boundaries?{]}

\subsection{\texorpdfstring{Intonational Melodies
\label{sec:intmelodies}}{Intonational Melodies }}

\section{\texorpdfstring{Historical Considerations
\label{sec:histphono}}{Historical Considerations }}

\chapter{\texorpdfstring{Parts of speech in Yawarana
\label{POS}}{Parts of speech in Yawarana }}

\section{Distinguishing parts of speech}

\begin{itemize}
\tightlist
\item
  \textcites[111]{koehn1986apalai}: ``Particles follow words of any
  class other than the ideophone, and never occur as free forms or in
  isolation.''
\end{itemize}

\subsection{Adverbs}

\begin{itemize}
\tightlist
\item
  copredicative function
\item
  no person inflection {[}todo: except some deverbalized ones?{]}
\item
  deriving aderbs: \obj{-ke} `\gl{prop}' {[}todo: -ke is negated with
  -jra only when on noun roots?{]}
\end{itemize}

\section{Shared morphology}

\section{Derivation and productivity}

\begin{itemize}
\tightlist
\item
  changing word classes
\item
  semantic variation \& non-compositional meanings
\item
  productive class-changing process w/ lexically conditioned suffixes
\item
  some constructions need a different word class, no meaning change per
  se
\end{itemize}

\chapter{\texorpdfstring{Nouns \label{nouns}}{Nouns }}

\section{\texorpdfstring{Pronouns \label{sec:pronouns}}{Pronouns }}

The personal pronouns of Yawarana are shown in \cref{tab:pronouns}. The
system shows the usual Cariban inclusive/exclusive (\gl{1+2} and
\gl{1+3}) distinction, although \obj{ejnë} `\gl{1+2}\gl{pro}' does not
have the /k/ found elsewhere in the family. It is like a reflex of an
old copula + infinitive \emph{*eti-në}. {[}todo: do we have parallel
cases elsewhere?{]} Regarding plural marking, it should be noted that
\obj{-kontomo} `\gl{pl}' appears to usually be restricted to verbs,
while \emph{-santomo} is found with third person pronouns and
demonstratives.

\begin{table}
\caption{Pronouns}
\label{tab:pronouns}
\centering
\begin{tabular}{lll}
\toprule
         &                     \gl{sg} &                                    \gl{pl} \\
\midrule
  \gl{1} & \obj{wïrë} ‘\gl{1}\gl{pro}’ &                                            \\
\gl{1+2} &                             &              \obj{ejnë} ‘\gl{1+2}\gl{pro}’ \\
\gl{1+3} &                             &               \obj{ana} ‘\gl{1+3}\gl{pro}’ \\
  \gl{2} & \obj{mërë} ‘\gl{2}\gl{pro}’ &  \obj{monkontomo} ‘\gl{2}\gl{pl}.\gl{pro}’ \\
  \gl{3} & \obj{tëwï} ‘\gl{3}\gl{pro}’ & \obj{tëwïsantomo} ‘\gl{3}\gl{pro}.\gl{pl}’ \\
\bottomrule
\end{tabular}

\end{table}

{[}todo: tajne, but not attested as an article{]}

Reduced forms of the first and second person pronouns occur as
proclitics {[}todo: proclitics or prefixes?{]} attaching to nouns to
indicate possessor (\cref{sec:nominalperson}), attached to verbs to
indicate subject or object (described in \cref{verbinfl}), or attached
to postpositions to indicate the object of the postposition (described
in \cref{sec:postinfl}):

\ex \label{convrisamaj-28}
\begingl \glpreamble uyïwïj yawë usenejkari sukuri jwama //
\gla u-y-ïwïj-∅ yawë u-senejka-ri sukuri jwama//
\glb \gl{1}-\gl{lk}-house-\gl{pert} \gl{loc} \gl{1}-stay-\gl{ipfv} silently ***//
\glft ‘I silently stay in my house.’ (personal knowledge
)//
\endgl
\xe

\ex \label{desccasmaj-25}
\begingl \glpreamble mënai wëjkase chijpë wararë //
\gla më-nai-Ø wëjka-se chi-jpëwara rë//
\glb \gl{2}-thing-\gl{poss} fall-\gl{pfv}.\gl{pst} \gl{cop}-NZRlike \gl{emph}//
\glft ‘‘se cayó tu cosa’’ (personal knowledge
)//
\endgl
\xe

\ex \label{convrisamaj-2}
\begingl \glpreamble mëyaruwari, mëpëkëpene //
\gla më-yaruwa-ri më-pëkëpene//
\glb \gl{2}-laugh-\gl{ipfv} \gl{2}-alone//
\glft ‘You just laugh.’ (personal knowledge
)//
\endgl
\xe

\ex \label{ctoaragrme-7}
\begingl \glpreamble moyochi //
\gla moyochi//
\glb spider//
\glft ‘la araña’ (personal knowledge
)//
\endgl
\xe

The third person demonstrative pronouns or articles are shown in
\cref{tab:pronouns3}. {[}todo: is there a 4‑way distinction?
{[}cf.~Ye'kwana?{]}{]} None of them have shortened, phonologically bound
counterparts.

\begin{table}
\caption{Demonstrative pronouns / articles}
\label{tab:pronouns3}
\centering
\begin{tabular}{lllll}
\toprule
              & \multicolumn{2}{l}{\gl{anim}} & \multicolumn{2}{l}{\gl{inan}} \\
\midrule
              &                                            \gl{sg} &                                            \gl{pl} &                           \gl{sg} &                                      \gl{pl} \\
    \gl{prox} &                   \obj{kërë} ‘\gl{prox}.\gl{anim}’ &    \obj{kërësantomo} ‘\gl{prox}.\gl{anim}.\gl{pl}’ &   \obj{eni} ‘\gl{prox}.\gl{inan}’ &   \obj{enijne} ‘\gl{prox}.\gl{inan}.\gl{pl}’ \\
medial? near? & \obj{michi} ‘\gl{med}.\gl{anim}’ / \obj{misi} ‘... & \obj{michisantomo} ‘\gl{med}.\gl{inan}.\gl{pl}’... &   \obj{mërë} ‘\gl{med}.\gl{inan}’ &                                              \\
    \gl{dist} &                  \obj{mëjkï} ‘\gl{dist}.\gl{anim}’ &    \obj{mëkïsantomo} ‘\gl{dist}.\gl{anim}.\gl{pl}’ & \obj{mëjnï} ‘\gl{dist}.\gl{inan}’ & \obj{mëjnijne} ‘\gl{dist}.\gl{inan}.\gl{pl}’ \\
\bottomrule
\end{tabular}

\end{table}

\begin{itemize}
\tightlist
\item
  nominal interrogative pronouns:

  \begin{itemize}
  \tightlist
  \item
    \obj{anïkï} `who' (with \emph{-santomo})
  \item
    \obj{ati} `what' (no plural)
  \item
    \emph{ëjkë} `which? (\gl{inan})'
  \end{itemize}
\end{itemize}

{[}todo: Are there plural forms of any of these?{]}

\section{\texorpdfstring{Nominal inflection
\label{sec:nouninfl}}{Nominal inflection }}

Nouns in Yawarana may bear suffixes marking their possession status
(\cref{sec:nounposssuf}), number (\cref{sec:nominalnumber}), and nominal
past tense (\cref{sec:nominaltense}). Possessed nouns may bear a person
prefix, or the linker \obj{y-} (\cref{sec:nominalperson}).

{[}todo: noun classes re: possession{]}

\subsection{\texorpdfstring{Suffixes for possessed and non-possessed
nouns
\label{sec:nounposssuf}}{Suffixes for possessed and non-possessed nouns }}

In the possession construction in Yawarana, the possessor noun occurs
immediately preceding the possessed noun, which is the head of the
possession phrase. {[}todo: crossref to phrase structure{]}
Alternatively, the possessor can appear as a prefix on the possessed
noun. The possessor noun is never marked (for instance, with genitive
case), but the possessed noun (the head) is often marked for being
possessed by a suffix; an unambiguous label for this counterpart of the
genitive is pertensive \parencites{dixon2010basic}. The choice of suffix
is lexically conditioned; while most nouns take \obj{-ri} `\gl{pert}',
some take \obj{-ti}. Unpossessed nouns generally are unmarked, but some
15 nouns {[}todo: which? list nouns{]} bear the suffix \obj{-të}
`\gl{npert}' when they appear without a possessor.

Examples \exref[][unsuffixednouns]{onlypossessed} illustrate the
possible patterns of markedness for nouns when possessed and
unpossessed. The vast majority of nouns in our corpus are unmarked when
unpossessed, but when possessed the suffix \obj{-ri} `\gl{pert}' occurs
\exref[]{onlypossessed}. A handful of nouns {[}todo: i.e., the 15?{]} is
marked with \obj{-ri}/\obj{-ti} `\gl{pert}' when possessed and with
\obj{-të} `\gl{npert}' when not possessed \exref[]{diffpossessed}.
Another handful is unmarked when possessed and marked with \obj{-të}
`\gl{npert}' when not possessed \exref[]{suffunpossessed}. The fourth
logical category, where neither possession or non-possession is marked,
contains very few members (only one attested so far). For these nouns,
the difference is marked only by the presence or absence of a possessive
prefix or free-form possessor \exref[]{unsuffixednouns}.

\ex\label{onlypossessed} Nouns that take a suffix only when possessed:

\begin{tabular}[t]{llll}

 \emph{akajra-ri} &          ‘X’s bow’ & \emph{akajra} &          ‘bow’ \\

\emph{y-amaka-ri} &        ‘X’s yucca’ &  \emph{amaka} &        ‘yucca’ \\
 \emph{y-ántë-ri} &     ‘X’s fishhook’ &   \emph{antë} &     ‘fishhook’ \\
\emph{y-ateri-ri} & ‘X’s garden/field’ &  \emph{ateri} & ‘garden/field’ \\
    \emph{ënu-ru} &          ‘X’s eye’ &    \emph{ënu} &          ‘eye’ \\
  \emph{y-ëpi-ri} &     ‘X’s medicine’ &    \emph{ëpi} &     ‘medicine’ \\

\end{tabular}
 \xe

\ex\label{diffpossessed} Nouns that take one suffix when possessed and
another when unpossessed:

\begin{tabular}[t]{llll}

   \emph{yë-ri} & ‘X’s tooth’ &   \emph{yë-të} &                                ‘tooth’ \\

 \emph{pata-ri} & ‘X’s place’ & \emph{pata-të} & ‘(part of name) San Juan de Manapiare’ \\
\emph{y-ese-ti} & ‘X’s name’  &  \emph{ese-të} &                                 ‘name’ \\
\emph{y-ase-tï} & ‘X’s cord’  &  \emph{ase-të} &                                 ‘cord’ \\

\end{tabular}
 \xe

\ex\label{suffunpossessed} Nouns that take a suffix only when
unpossessed:

\begin{tabular}[t]{llll}

  \emph{yëjpë} &  ‘X’s bone’ &                \emph{yëjpë-të} &  ‘bone’ \\

   \emph{petï} & ‘X’s thigh’ & \emph{petï-të} / \emph{pej-të} & ‘thigh’ \\
\emph{y-aponi} & ‘X’s stool’ &                 \emph{apon-të} & ‘stool’ \\

\end{tabular}
 \xe

\ex\label{unsuffixednouns} Nouns that never take a suffix, whether
possessed or unpossessed:

\begin{tabular}[t]{llll}

\emph{i-jmëy} & 'his egg’ & \emph{ëjmëy} & 'egg’ \\

\end{tabular}
 \xe

\subsection{\texorpdfstring{Number suffixes
\label{sec:nominalnumber}}{Number suffixes }}

There are three plural suffixes that can occur on nouns, apparently
freely interchangeable. What conditions the choice of suffix is not
clear as of yet.

\begin{itemize}
\tightlist
\item
  \emph{-kontomo}
\end{itemize}

\ex \label{ctorat-17}
\begingl \glpreamble waijtatomo ëjwenakase //
\gla waijta-tomo ëjwenaka-se//
\glb mouse-\gl{pl} vomit-\gl{pst}//
\glft ‘The mice vomited.’ (personal knowledge
)//
\endgl
\xe

\ex \label{ctorat-40}
\begingl \glpreamble tipapëjsejne waijtajne //
\gla tipa-pëj-se-jne waijta-jne//
\glb go\_in\_group-\gl{plur}-\gl{pst}-\gl{pl} mouse-\gl{pl}//
\glft ‘the mice went away.’ (personal knowledge
)//
\endgl
\xe

\subsection{\texorpdfstring{Nominal tense
\label{sec:nominaltense}}{Nominal tense }}

\begin{itemize}
\tightlist
\item
  \obj{-jpë} `\gl{pst}'
\end{itemize}

\subsection{\texorpdfstring{Argument prefixes
\label{sec:nominalperson}}{Argument prefixes }}

Person prefixes on nouns are conditioned by the initial segment
(\cref{tab:possprefixes}). C-initial nouns take \obj{i-} `\gl{3}', and
first and second person are bare \obj{u-} `\gl{1}' and \obj{më-}
`\gl{2}'. On V-initial nouns, third person is marked with \obj{t-}
`\gl{3}', and the first and second person prefixes combine with the
linker \obj{y-}. Some examples are shown in
\exref[][lastex]{convrisamaj-28}.

\begin{table}
\caption{Possessive prefixes on nouns}
\label{tab:possprefixes}
\centering
\begin{tabular}{lll}
\toprule
       &                \_C &                                  \_V \\
\midrule
\gl{1} &  \obj{u-} ‘\gl{1}’ &  \obj{u-} ‘\gl{1}’\obj{y-} ‘\gl{lk}’ \\
\gl{2} & \obj{më-} ‘\gl{2}’ & \obj{më-} ‘\gl{2}’\obj{y-} ‘\gl{lk}’ \\
\gl{3} &  \obj{i-} ‘\gl{3}’ &                             \obj{t-} \\
\bottomrule
\end{tabular}

\end{table}

\ex \label{convrisamaj-28}
\begingl \glpreamble uyïwïj yawë usenejkari sukuri jwama //
\gla u-y-ïwïj-∅ yawë u-senejka-ri sukuri jwama//
\glb \gl{1}-\gl{lk}-house-\gl{pert} \gl{loc} \gl{1}-stay-\gl{ipfv} silently ***//
\glft ‘I silently stay in my house.’ (personal knowledge
)//
\endgl
\xe

\ex \label{ctorat-46}
\begingl \glpreamble tïwïj yaka waraijtokomo manikijpë //
\gla t-ïwïj-∅ yaka waraijtokomo manikijpë//
\glb \gl{3}-house-\gl{pert} \gl{all} man ***//
\glft ‘He went to his house.’ (personal knowledge
)//
\endgl
\xe

\ex \label{lastex}
\begingl \glpreamble pïrarë ti iwenaru wejsapë //
\gla pïrarë ti i-wena-ru wej-sapë//
\glb \gl{neg}.\gl{exist} \gl{hsy} \gl{3}-vomit-\gl{pert} \gl{cop}-\gl{pfv}//
\glft ‘their vomit was not there.’ (personal knowledge
)//
\endgl
\xe

The linker also occurs with (pro-)nominal possessors:

\ex \label{desccasmaj-131}
\begingl \glpreamble tarine ma //
\gla tarine ma//
\glb fast \gl{restr}//
\glft ‘None’ (personal knowledge
)//
\endgl
\xe

There are some nouns {[}todo: presumably kinship terms{]} that take an
apparently older old second person \obj{a-} `\gl{2}'
(\cref{tab:oldpossprefixes}).

\begin{table}
\caption{Archaic possessive prefixes on nouns}
\label{tab:oldpossprefixes}
\centering
\begin{tabular}{lll}
\toprule
       &               \_C &                                 \_V \\
\midrule
\gl{1} & \obj{u-} ‘\gl{1}’ & \obj{u-} ‘\gl{1}’\obj{y-} ‘\gl{lk}’ \\
\gl{2} & \obj{a-} ‘\gl{2}’ & \obj{a-} ‘\gl{2}’\obj{y-} ‘\gl{lk}’ \\
\gl{3} & \obj{i-} ‘\gl{3}’ &                   \obj{t-} ‘\gl{3}’ \\
\bottomrule
\end{tabular}

\end{table}

{[}todo: find more examples of these{]}

\subsection{\texorpdfstring{Root suppletion in nominal possession
\label{sec:irregnouns}}{Root suppletion in nominal possession }}

\begin{itemize}
\tightlist
\item
  `father':

  \begin{itemize}
  \tightlist
  \item
    1 \emph{papa}
  \item
    2 \emph{ëmë} / \emph{omo} / \emph{ëmo} (?)
  \item
    3 \emph{imu}
  \item
    NP \emph{yïmï}
  \end{itemize}
\end{itemize}

candidates:

\begin{itemize}
\tightlist
\item
  `arrow'
\item
  `house'
\item
  `poop'
\item
  `mother'
\item
  `hammock string'
\end{itemize}

\section{\texorpdfstring{Nominal Derivational Morphology
\label{sec:nounderiv}}{Nominal Derivational Morphology }}

\begin{itemize}
\tightlist
\item
  V → N

  \begin{itemize}
  \tightlist
  \item
    \obj{-ri} `action \gl{nmlz}' {[}todo: potentially A.NMLZ{]}
  \item
    \obj{-jpë} `\gl{pst}.\gl{acnnmlz}'

    \begin{itemize}
    \tightlist
    \item
      also `\gl{pst}.\gl{abs}.\gl{nmlz}' {[}todo: convsuenmaj-47{]}
    \end{itemize}
  \item
    \obj{-në} `\gl{inf}' or `generic action nominalizer' {[}todo: this
    probably only occurs on intransitive verbs{]}

    \begin{itemize}
    \tightlist
    \item
      \emph{wënkej-në} from transitive \emph{wënkepï} `forget'
    \end{itemize}
  \item
    \obj{-ni} `\gl{agtnmlz}' {[}todo: ctoyucairdi-4, descokigrme-53 for
    predicative use{]}
  \item
    \obj{n-}V\obj{-ri} only with \obj{yeme} 'eat(fruits,\_eggs,\_soup)'
  \item
    \obj{-sapë} `\gl{abs}.\gl{nmlz}' (contrast with \obj{-jpë}
    `\gl{pst}.\gl{acnnmlz}')
  \item
    \obj{-topo} `\gl{circ}.\gl{nmlz}'
  \item
    \obj{‑pïnï} `\gl{priv}.\gl{nmlz}' {[}todo: only found with -se-mï,
    not attested as nominalizer{]}
  \end{itemize}
\item
  Adv → N

  \begin{itemize}
  \tightlist
  \item
    \obj{-mï} `\gl{nmlz}'
  \item
    \obj{-ano} `\gl{nmlz}'
  \item
    absence of \emph{-ato} `\gl{nmlz}'
  \end{itemize}
\item
  Postp → N

  \begin{itemize}
  \tightlist
  \item
    \obj{-ano} `\gl{nmlz}'
  \end{itemize}
\item
  N → N

  \begin{itemize}
  \tightlist
  \item
    discuss \emph{pïjkë} and \emph{sere-kë} `manioc-DIM' , reference
    sections
  \item
    \emph{-imë}: e.g., \emph{wara} `woman' \emph{waraimë} `married
    woman' (dictionary)
  \end{itemize}
\item
  What about \obj{-jpë} `\gl{pst}.\gl{acnnmlz}' on \gl{ad} forms? Does
  it derive a noun?

  \begin{itemize}
  \item
    \ex \label{histyarirdi-592}
    \begingl \glpreamble pata penarëjpë mëtë ta, mërë Cerro Muñeca tajtoj mare toto ya //
    \gla pata-Ø penarë-jpë mëtë ta mërë Cerro Muñeca taj-toj mare toto ya//
    \glb town-\gl{poss} before-\gl{pst}.\gl{psn} there like \gl{2}\gl{sg}  say-\gl{circ} \gl{in}.\gl{rel} non.indian \gl{erg}//
    \glft ‘‘ahí se ve el sitio donde vivian, donde los criollos llama Cerro Muñeca’’ (personal knowledge
    )//
    \endgl
    \xe
  \end{itemize}
\end{itemize}

\chapter{\texorpdfstring{Verbal roots and stems
\label{verbderiv}}{Verbal roots and stems }}

\section{Classes of verbs}

Yawarana verb roots can be divided into those yielding an intransitive
stem, and those yielding a transitive stem. The only inflectional
criterion distinguishing the two classes is the third person prefix
\obj{ta-}, which can only occur on transitive stems. Thus, transitive
\obj{yawanka} `kill' can take \obj{ta-} \exref[]{convfemgrme-217}, but
intransitive \obj{yaruwa} `laugh' does not \exref[]{convrisamaj-42}.

\pex\label{}    \a Yawarana\\
    \label{convrisamaj-42}        \begingl
        \glpreamble yaruwakontomo yatum ponoko //
        \gla yaruwa-∅-kontomo yatum ponoko//
        \glb laugh-\gl{ipfv}-\gl{pl} day ***//
            \glft ‘They laugh every day.’//  
        \endgl 
    \a Yawarana\\
    \label{convfemgrme-217}        \begingl
        \glpreamble iyawë chipëkë, tayawankase //
        \gla i- yawë chi-Ø pëkë ta-yawanka-se//
        \glb \gl{3}- \gl{ctmp} \gl{cop}-\gl{ipfv} because \gl{3}\gl{o}-destroy-\gl{pfv}.\gl{pst}//
            \glft ‘‘por eso, la mató’’//  
        \endgl 
\xe

\begin{itemize}
\item
  detransitive
\item
  ditransitive
\item
  ``n-adding''
\item
  accidental lability
\item
  ijtëri
\item
  Note that all transitive verbs are consonant‑initial, whether
  etymologically or not because \obj{y-} `\gl{lk}' is added to all
  vowel‑initial roots
\item
  the \emph{y‑} disappears when preceded by the detransitivizer {[}todo:
  examples for detransitivized verbs{]} {[}todo: what about V-initial
  intransitive verbs? how are they inflected?{]}
\end{itemize}

\section{\texorpdfstring{Verbalizing suffixes
\label{sec:vbz}}{Verbalizing suffixes }}

None of these are productive, although there are many lexemes derived
with them.

\subsection{Intransitive}

\subsubsection{\texorpdfstring{-ta / -na \label{sec:tavbz}}{-ta / -na }}

\obj{-ta} `\gl{vbz}.\gl{intr}' derives intransitive verbs.

\begin{table}
\caption{Lexemes derived with \emph{-ta}}
\label{tab:tavbz}
\centering
\begin{tabular}{ll}
\toprule
Base & Derivation \\
\midrule
     &            \\
     &            \\
     &            \\
     &            \\
     &            \\
     &            \\
     &            \\
     &            \\
     &            \\
     &            \\
     &            \\
     &            \\
     &            \\
     &            \\
     &            \\
     &            \\
     &            \\
     &            \\
     &            \\
     &            \\
     &            \\
     &            \\
     &            \\
     &            \\
     &            \\
     &            \\
\bottomrule
\end{tabular}

\end{table}

\subsubsection{\texorpdfstring{\emph{-pamï} /
\emph{-mamï}}{-pamï / -mamï}}

{[}todo: check for -pantari{]} {[}todo: check tri and way{]}

\subsection{Transitive}

\subsubsection{\texorpdfstring{-ka \label{sec:kavbz}}{-ka }}

\obj{-ka} `\gl{vbz}.\gl{tr}' derives transitive verbs.

\subsubsection{\texorpdfstring{\emph{-jtë} / \emph{-të}
\label{sec:jtevbz}}{-jtë / -të }}

\begin{itemize}
\tightlist
\item
  \obj{-jtë} `\gl{vbz}.\gl{tr}'
\end{itemize}

\subsubsection{\texorpdfstring{\emph{-ma} / \emph{-pa}
\label{sec:macaus}}{-ma / -pa }}

\begin{itemize}
\tightlist
\item
  \obj{-ma} `\gl{caus}'
\end{itemize}

\section{Valency-changing affixes}

\subsection{\texorpdfstring{Detransitivizing prefixes
\label{sec:detrz}}{Detransitivizing prefixes }}

{[}todo: distribution? morphemic analysis?{]}

\begin{enumerate}
\def\labelenumi{\arabic{enumi}.}
\item
  \obj{s-}
\item
  \obj{ëj-}
\item
  \obj{at-}
\end{enumerate}

\subsection{Transitivizing suffixes}

\begin{itemize}
\tightlist
\item
  \obj{-ma} `\gl{caus}' {[}todo: from verbs or only nouns?{]}
\item
  {[}todo: -nïkï / -nïpï / -nëpï{]}
\item
  does \obj{-ka} `\gl{vbz}.\gl{tr}' go on intransitive verb stems?
\end{itemize}

\subsection{\texorpdfstring{Ditransitivizing suffixes
\label{sec:ditrz}}{Ditransitivizing suffixes }}

\begin{itemize}
\tightlist
\item
  \obj{-po} `\gl{caus}'
\end{itemize}

\section{\texorpdfstring{Meaning-changing suffixes
\label{sec:meaningderiv}}{Meaning-changing suffixes }}

\begin{itemize}
\item
  \obj{-po} `\gl{des}' (only occurs with \obj{-ri} `\gl{ipfv}' and
  \obj{-jra} `\gl{neg}')
\item
  \gl{plur}
\item
  \gl{cess}
\end{itemize}

\chapter{\texorpdfstring{Verbal inflection
\label{verbinfl}}{Verbal inflection }}

{[}todo: write an introduction{]}

\section{\texorpdfstring{Person prefixes
\label{sec:verbperson}}{Person prefixes }}

Verbs are inflected for person with a set of prefixes, shown in
\cref{tab:verbprefixes}. First and second person prefixes show ergative
alignment, expressing \gl{s} and \gl{p}. Intransitive verbs are not
overtly inflected for third person, while transitive verbs show an
optional \obj{ta-} in \gl{3}\textgreater{}\gl{3} scenarios.{[}todo:
mentionworthy that this is not in alternation with preceding NPs{]}

\begin{table}
\caption{Person marking prefixes on verbs}
\label{tab:verbprefixes}
\centering
\begin{tabular}{lll}
\toprule
         &          \gl{intr} &                           \gl{tr} \\
\midrule
  \gl{1} &  \obj{u-} ‘\gl{1}’ &                 \obj{u-} ‘\gl{1}’ \\
  \gl{2} & \obj{më-} ‘\gl{2}’ &                \obj{më-} ‘\gl{2}’ \\
\gl{1+2} &      \emph{ej(n)-} &                     \emph{ej(n)-} \\
  \gl{3} &                  ∅ & \obj{ta-} ‘\gl{3}\gl{p} / \gl{3}’ \\
\bottomrule
\end{tabular}

\end{table}

\begin{itemize}
\tightlist
\item
  \obj{më-} `\gl{2}' marking A in \exref[]{2a}
\item
  is the verb in \exref[]{learn} transitive?
\item
  those in \exref[]{2sub} and \exref[]{1sub} are subordinate
\end{itemize}

\pex\label{2a}    \a Yawarana\\
    \label{convhistfamsjm-13}        \begingl
        \glpreamble tëwï ma takï mëyakarama chijpë wararë kwa ta sënkatoj mëtë //
        \gla tëwï ma takï më-yakarama-Ø chi-jpëwara rë kwa ta sënka-toj mëtë//
        \glb \gl{3}\gl{sg} \gl{restr} \gl{cnfrm} \gl{2}-tell-\gl{ipfv} \gl{cop}-NZRlike \gl{emph} thus get.finished-\gl{circ} there//
            \glft ‘‘cuenta otra vez como terminaron ahí’’//  
        \endgl 
    \a Yawarana\\
    \label{ctovarmafl-324}        \begingl
        \glpreamble michi ma mëyapëjjrama //
        \gla michi ma më-yapëj-jrama//
        \glb \gl{3}\gl{an}:\gl{px} \gl{restr} \gl{2}-seize-\gl{proh}//
            \glft ‘‘esto no lo toques’’//  
        \endgl 
    \a Yawarana\\
    \label{histyarirdi-633}        \begingl
        \glpreamble mëinija ka, aniki pinchi, tënësem warai yichapë, okonotojpe //
        \gla më-ini-ja ka anikipinchi tënësem warai yi-chapë okono-tojpe//
        \glb \gl{2}-see-\gl{neg} \gl{qp} whoblank fish like put-\gl{abs}.\gl{nzr} get.dry-\gl{purp}//
            \glft ‘‘usted no ha visto un pescado que se pone, para que se seque’’//  
        \endgl 
\xe

\pex\label{learn}    \a Yawarana\\
    \label{convhistfamsjm-15}        \begingl
        \glpreamble kwase mëëmpamïjpë ejnë waimu yaye //
        \gla kwase më-ëmpamï-jpë ejnë waimu-Ø yaye//
        \glb how \gl{2}-learn-\gl{pst} \gl{1+2} speak-\gl{nmlz} \gl{perl}//
            \glft ‘‘como aprendiste nuestro idioma’//  
        \endgl 
    \a Yawarana\\
    \label{convhistfamsjm-238}        \begingl
        \glpreamble irëjpë, kwaraijyawë rë mëëmpamïjpë tajto marë ti? //
        \gla irëjpë kwaraijyawë rë më-ëmpamï-jpë tajto marë ti//
        \glb then when \gl{emph} \gl{2}-learn-\gl{pst}  keeps.going like//
            \glft ‘‘después, cuando aprediste lo que dices?’’//  
        \endgl 
\xe

\pex\label{2sub}    \a Yawarana\\
    \label{convfemgrme-43}        \begingl
        \glpreamble mërë warë mëyënëtojpano ka uyakërë mërë wepïrï //
        \gla mërë warë më-yënë-tojpano ka u-y-akërë mërë wepï-rï//
        \glb \gl{2}\gl{sg} thus \gl{2}-eat.meat-\gl{fin}:\gl{concl} \gl{qp} \gl{1}\gl{sg}-\gl{rel}-\gl{com} \gl{2}\gl{sg} come-\gl{ipfv}//
            \glft ‘‘así para tú comer tu viniste conmigo?’’//  
        \endgl 
    \a Yawarana\\
    \label{ctoaragrme-5}        \begingl
        \glpreamble mëyakarama mare //
        \gla më-yakarama-Ø mare//
        \glb \gl{2}-tell-\gl{ipfv} \gl{in}.\gl{rel}//
            \glft ‘ese que tu estás diciendo’//  
        \endgl 
\xe

\ex \label{1sub}
\begingl \glpreamble uyepematojpe pïrarë wïrë inawë //
\gla u-yepema-toj pe pïrarë wïrë inawë//
\glb \gl{1}\gl{sg}-pay-\gl{circ} \gl{ess} nothing \gl{1}\gl{sg} in.possesion//
\glft ‘‘yo no tengo para pagar’’ (personal knowledge
)//
\endgl
\xe

\begin{itemize}
\tightlist
\item
  1\textgreater2:
\end{itemize}

\pex\label{}    \a Yawarana\\
   \label{convfemgrme-231}        \begingl
       \glpreamble entë mëinpojra wïrë ya //
       \gla entë më-in-pojra wïrë ya//
       \glb here \gl{2}-see-\gl{neg}.\gl{des} \gl{1}\gl{sg} \gl{erg}//
           \glft ‘‘yo no los quiero ver aquí’’//  
       \endgl 
   \a Yawarana\\
   \label{convfemgrme-232}        \begingl
       \glpreamble mëini wïrë ya //
       \gla më-ini-Ø wïrë ya//
       \glb \gl{2}-see-\gl{ipfv} \gl{1}\gl{sg} \gl{erg}//
           \glft ‘‘te voy a ver’’//  
       \endgl 
   \a Yawarana\\
   \label{ctovarmafl-283}        \begingl
       \glpreamble tëwï ke ma mëyepema wïrë ya, ta ti ta //
       \gla tëwï ke ma më-yepema-Ø wïrë ya ta-Ø ti ta//
       \glb \gl{3}\gl{sg} \gl{instr} \gl{restr} \gl{2}-pay-\gl{ipfv} \gl{1}\gl{sg} \gl{erg} say-\gl{ipfv} like like//
           \glft ‘‘te pago con esto no más, dijo’’//  
       \endgl 
\xe

\begin{itemize}
\tightlist
\item
  one attested case of \obj{ta-} `\gl{3}\textgreater{}\gl{3}' on the
  lexical verb of a \emph{-pëkë} construction w/ 2nd person \gl{a} on
  \gl{aux} {[}todo: which one?{]}

  \begin{itemize}
  \tightlist
  \item
    Ø‑ `\gl{3}\gl{p}' with transitive verbs with \gl{1}\gl{a} or
    \gl{2}\gl{a}
  \end{itemize}
\item
  one example of \obj{më-} `\gl{2}' `\gl{2}\gl{a}' on imperative verb
  {[}todo: GrMePers.029{]} {[}todo: FM: u- is sometimes wï-, but usually
  not transcribed, and distribution unclear{]}
\end{itemize}

{[}todo: Check: appears that first and second person are reduced forms,
with long vowel? wait for leila{]}

\begin{itemize}
\tightlist
\item
  *\emph{t‑V‑se} is no more --- the \emph{t‑} is gone, except in
  lexicalized items
\end{itemize}

\section{\texorpdfstring{Main clause tense‑aspect‑mood‑polarity suffixes
\label{sec:tam}}{Main clause tense‑aspect‑mood‑polarity suffixes }}

Verbs in main clauses are inflected for TAMP with a set of suffixes,
shown in \cref{tab:verbtam}. They are discussed in
\crefrange{sec:riipfv}{sec:sareimn}.

\begin{table}
\caption{Verbal TAM suffixes}
\label{tab:verbtam}
\centering
\begin{tabular}{ll}
\toprule
                             Suffix &            Function \\
\midrule
              \obj{-ri} ‘\gl{ipfv}’ &        imperfective \\
              \obj{-jpë} ‘\gl{pst}’ &                past \\
               \obj{-se} ‘\gl{pst}’ &     past perfective \\
             \obj{-sapë} ‘\gl{pfv}’ &             perfect \\
             \obj{-sarë} ‘\gl{imn}’ &     imminent future \\
\obj{-nëpëkë} ‘\gl{prog}.\gl{intr}’ & \gl{prog}.\gl{intr} \\
                 \obj{pëkë} ‘about’ &   \gl{prog}.\gl{tr} \\
                    \emph{‑tojpano} &            \gl{fut} \\
                     (\obj{-tojpe}) &            \gl{fut} \\
                          \obj{-ja} &            \gl{neg} \\
                \obj{-se}\obj{-mï}  &        ‘obligation’ \\
                          \obj{-në} &        impersonal S \\
                        \obj{-topo} &                     \\
\bottomrule
\end{tabular}

\end{table}

\begin{table}
\caption{Non-declarative suffixes}
\label{tab:nondecltam}
\centering
\begin{tabular}{ll}
\toprule
                                  Suffix &                                      Function \\
\midrule
                           \emph{‑jrama} &                                     \gl{proh} \\
                               \emph{-i} &                                     \gl{juss} \\
         \obj{-kë} / ‑\emph{të}\obj{-kë} &                   \gl{imp} / \gl{imp}.\gl{pl} \\
\obj{-ta} / \obj{-ta}\emph{ntë}\obj{-kë} & \gl{imp}.\gl{mot} / \gl{imp}.\gl{mot}.\gl{pl} \\
\bottomrule
\end{tabular}

\end{table}

Misc:

\begin{itemize}
\tightlist
\item
  \obj{-se}=\obj{pano} `\gl{pst}=\gl{concl}'
\item
  \obj{-saj}=\obj{pano} `\gl{pfv}=\gl{concl}'
\item
  \obj{-sarë}=\obj{pano} `\gl{imn}=\gl{concl}'
\end{itemize}

\subsection{\texorpdfstring{\obj{-ri} \label{sec:riipfv}}{ }}

\begin{itemize}
\tightlist
\item
  allomorphy:

  \begin{itemize}
  \tightlist
  \item
    \obj{-∅} `\gl{ipfv}', phonetic loss
  \item
    \obj{-ru} `\gl{ipfv}', assimilation
  \item
    what about \obj{-rï} `\gl{ipfv}'? looks like the most conservative
    form
  \end{itemize}
\item
  diachrony: \obj{-ri} `\gl{acnnmlz}'
\item
  pluralization?
\item
  combines with \obj{-jra} `\gl{neg}':
\end{itemize}

\ex \label{convrisamaj-4}
\begingl \glpreamble wïrë yaruwarijra //
\gla wïrë yaruwa-ri-jra//
\glb \gl{1}\gl{pro} laugh-\gl{ipfv}-\gl{neg}//
\glft ‘I don’t laugh.’ (personal knowledge
)//
\endgl
\xe

\subsubsection{Semantics}

\begin{itemize}
\tightlist
\item
  not specified for tense, just imperfective aspect:

  \begin{itemize}
  \tightlist
  \item
    past \exref[]{ctorat-16}
  \item
    future \exref[]{convrisamaj-6}
  \item
    gnomic/present? \exref[]{gnomicri}
  \end{itemize}
\end{itemize}

\ex \label{ctorat-16}
\begingl \glpreamble irëjpë tëwï waijtatomo nwajtëri //
\gla irëjpë tëwï waijta-tomo nwajtë-ri//
\glb then \gl{3}\gl{pro} mouse-\gl{pl} dance-\gl{ipfv}//
\glft ‘Then the mice were dancing.’ (personal knowledge
)//
\endgl
\xe

\ex \label{convrisamaj-6}
\begingl \glpreamble ¿ kwase ejnë yaruwari? //
\gla kwase ejnë yaruwa-ri//
\glb how \gl{1+2}\gl{pro} laugh-\gl{ipfv}//
\glft ‘How will we laugh?’ (personal knowledge
)//
\endgl
\xe

\pex\label{gnomicri}    \a Yawarana\\
    \label{convrisamaj-4}        \begingl
        \glpreamble wïrë yaruwarijra //
        \gla wïrë yaruwa-ri-jra//
        \glb \gl{1}\gl{pro} laugh-\gl{ipfv}-\gl{neg}//
            \glft ‘I don’t laugh.’//  
        \endgl 
    \a Yawarana\\
    \label{convrisamaj-28}        \begingl
        \glpreamble uyïwïj yawë usenejkari sukuri jwama //
        \gla u-y-ïwïj-∅ yawë u-senejka-ri sukuri jwama//
        \glb \gl{1}-\gl{lk}-house-\gl{pert} \gl{loc} \gl{1}-stay-\gl{ipfv} silently ***//
            \glft ‘I silently stay in my house.’//  
        \endgl 
\xe

\subsection{\texorpdfstring{\obj{-jpë}}{}}

\begin{itemize}
\tightlist
\item
  allomorphy: none?
\item
  diachrony: from \obj{-jpë} `\gl{pst}.\gl{acnnmlz}' {[}todo:
  negation?{]} {[}todo: semantics?{]}
\end{itemize}

\subsection{\texorpdfstring{\obj{-se}}{}}

\begin{itemize}
\tightlist
\item
  allomorphy: \obj{-se}/\obj{-che} `\gl{ptcp} / \gl{sup}'
\item
  diachrony: from \obj{-se} `\gl{ptcp} / \gl{sup}'
\item
  negation: replaced with \obj{-ja} `\gl{neg}' \exref[]{conv1stenc-28}
\end{itemize}

\ex \label{conv1stenc-28}
\begingl \glpreamble wëjkaja, ana tëse neke ne //
\gla wëjka-ja ana të-se neke ne//
\glb fall-\gl{neg} \gl{1+3} go-\gl{pfv}.\gl{pst} \gl{contrast} \gl{ints}//
\glft ‘‘no nos caimos, nosotros nos fuimos’’ (personal knowledge
)//
\endgl
\xe

{[}todo: semantics?{]}

\subsection{\texorpdfstring{\obj{-sapë}}{}}

\begin{itemize}
\tightlist
\item
  diachrony: from \obj{-sapë} `\gl{abs}.\gl{nmlz}'
\item
  distribution: only occurs on the copula?
\item
  allomorphy: \obj{-sapë} and \obj{-saj}
\item
  negation: with \obj{-ja} `\gl{neg}' on lexical verb
  \exref[][ctoaragrme-40]{ctoaragrme-38} {[}todo: semantics?{]}
\end{itemize}

\ex \label{ctoaragrme-38}
\begingl \glpreamble irë wejtane mujyampe patakaja wejsapë //
\gla irë wejtane mujyam pe pataka-ja wej-sapë//
\glb \gl{sit}:\gl{dem} though pregnant \gl{ess} exit-\gl{neg} \gl{cop}-\gl{perf}//
\glft ‘‘a pesar de eso no salió embarazada’’ (personal knowledge
)//
\endgl
\xe

\ex \label{ctoaragrme-39}
\begingl \glpreamble apatakaja pïnïka wejsapë //
\gla apataka-ja pïnïka wej-sapë//
\glb exit-\gl{neg} \gl{prob} \gl{cop}-\gl{perf}//
\glft ‘tal vez no salió (embarazada)’ (personal knowledge
)//
\endgl
\xe

\ex \label{ctoaragrme-40}
\begingl \glpreamble tayakïjtëja pïnika wejsapë //
\gla ta-yakïjtë-ja pïnika wej-sapë//
\glb \gl{3}\gl{o}-impregnate-\gl{neg} \gl{prob} \gl{cop}-\gl{perf}//
\glft ‘‘tal vez no se acostó con ella’’ (personal knowledge
)//
\endgl
\xe

\begin{itemize}
\tightlist
\item
  what about \exref[]{ctorat-19}? is that existential negation?
\end{itemize}

\ex \label{ctorat-19}
\begingl \glpreamble pïrarë ti iwenaru wejsapë //
\gla pïrarë ti i-wena-ru wej-sapë//
\glb \gl{neg}.\gl{exist} \gl{hsy} \gl{3}-vomit-\gl{pert} \gl{cop}-\gl{pfv}//
\glft ‘their vomit was not there.’ (personal knowledge
)//
\endgl
\xe

\subsection{\texorpdfstring{\obj{-sarë} \label{sec:sareimn}}{ }}

\begin{itemize}
\tightlist
\item
  once a converb, now `imminent future'
\end{itemize}

\ex \label{ctorat-25}
\begingl \glpreamble irëjpë ta ti ta konopo wejsarë konopo wejsarë //
\gla irëjpë ta-∅ ti ta konopo wej-sarë konopo wej-sarë//
\glb then say-\gl{ipfv} \gl{hsy} like rain come-\gl{imn} rain come-\gl{imn}//
\glft ‘Then they said: “it’s raining, it’s raining”.’ (personal knowledge
)//
\endgl
\xe

\ex \label{ctoaragrme-25}
\begingl \glpreamble moyochi tasarë, moyochi chipokono kojpaye pïnika warotari //
\gla moyochi ta-sarë moyochi chi-poko-no kojpaye pïnika warota-ri//
\glb spider say-\gl{inm} spider \gl{cop}-because-\gl{nzr} night:\gl{perl} \gl{prob} work-\gl{ipfv}//
\glft ‘le dicen araña, tal vez porque la araña trabaja de noche’ (personal knowledge
)//
\endgl
\xe

\section{Subordinate Clause markers}

{[}todo: these should maybe go to another section{]}

\begin{itemize}
\tightlist
\item
  Nominalizations

  \begin{itemize}
  \tightlist
  \item
    \obj{-ri} `\gl{acnnmlz}'
  \item
    \obj{-jpë} `\gl{pst}.\gl{acnnmlz}'
  \item
    \obj{-topo} `\gl{circ}.\gl{nmlz}'
  \end{itemize}
\item
  Adverbial Clauses (S/A)

  \begin{itemize}
  \tightlist
  \item
    \obj{-se} `supine'
  \item
    \obj{-tane} `concessive'
  \item
    \obj{-sarë} `converb'
  \item
    \emph{‑yapo} `neg.purp'
  \item
    others?
  \end{itemize}
\item
  Nominalization + postposition (S/P)

  \begin{itemize}
  \tightlist
  \item
    \obj{-∅} `\gl{ipfv}'\obj{yawë} `simult'
  \item
    \obj{-∅} `\gl{ipfv}' \obj{pe} `\gl{ess}' `when'
  \item
    \obj{-saj} `\gl{abs}.\gl{nmlz}'\obj{yawë} `simult'
  \item
    \obj{-tojpe} `purpose'
  \item
    (‑jpë)=tërë `after'
  \item
    on auxiliary: \obj{-ri} + \emph{po} `\gl{ctrf}'
  \end{itemize}
\item
  not attested:

  \begin{itemize}
  \tightlist
  \item
    \emph{se} `\gl{des}'
  \item
    \emph{-ajtawï} `if when'
  \end{itemize}
\end{itemize}

\section{\texorpdfstring{Number \label{sec:verbalnumber}}{Number }}

\begin{itemize}
\tightlist
\item
  \obj{-ri}=\obj{-kontomo}
\item
  \obj{-sapë}=\obj{-kontomo}
\item
  \obj{-saj}\obj{-se}=\obj{-jnë}
\item
  \obj{-se}=\obj{-jnë}=\obj{-kontomo}
\item
  \obj{-se}=\obj{-jnë}=\emph{pano}
  (\obj{-se}=\obj{-jnë}=kontom=\emph{pano}?)
\item
  \obj{-të}\obj{-kë} for the imperative
\item
  what about \emph{-i} `\gl{juss}'?
\end{itemize}

\section{Copula / Auxiliary}

{[}todo: paradigm{]} {[}todo: did any particles develop from inflected
forms? Man, wai, manai, etc{]} {[}todo: are there irregular pat/perfect
participles? nahkë, etc{]} {[}todo: ejnë may come from an inflected form
of the copula{]}

\begin{itemize}
\tightlist
\item
  there is (synchronically suppletive) stem allomorphy: \emph{chi} and
  \emph{wej}
\item
  \obj{marë} `\gl{rel}.\gl{inan}' and \obj{manïkï} `\gl{rel}.\gl{anim}'
\end{itemize}

\chapter{\texorpdfstring{Postpositions \label{postp}}{Postpositions }}

\section{Defining the category}

\begin{itemize}
\tightlist
\item
  monomorphemic vs bipartite (vs `stacked')
\end{itemize}

\section{\texorpdfstring{Inflectional morphology
\label{sec:postinfl}}{Inflectional morphology }}

Postpositions take the same inflectional prefixes as nouns
(\cref{sec:nounposssuf}). {[}todo: are there postpositions with third
person i-?{]}

\begin{table}
\caption{Person marking prefixes on postpositions}
\label{tab:postpprefixes}
\centering
\begin{tabular}{lll}
\toprule
         &                 \_V &                 \_C \\
\midrule
  \gl{1} &   \obj{u-} \obj{y-} &            \obj{u-} \\
  \gl{2} &  \obj{më-} \obj{y-} &           \obj{më-} \\
\gl{1+2} & \obj{ejnë} \obj{y-} & \obj{ejnë} \obj{y-} \\
  \gl{3} &            \obj{i-} &            \obj{t-} \\
      NP &            \obj{y-} &            \obj{y-} \\
\bottomrule
\end{tabular}

\end{table}

Also:

\begin{itemize}
\tightlist
\item
  \obj{-kontomo} `\gl{pl}'
\item
  \obj{ësë-} `\gl{detrz}'
\end{itemize}

\section{Locative Postpositions}

\begin{itemize}
\tightlist
\item
  clear bipartite Ground+Path
\item
  unproductive Bipartite X+Path?
\item
  other forms
\end{itemize}

\begin{table}
\caption{Locative postpositions}
\label{tab:locpost}
\centering
\begin{tabular}{lll}
\toprule
        &              \gl{all} &              \gl{loc} \\
\midrule
 inside & \obj{yaka} ‘\gl{all}’ & \obj{yawë} ‘\gl{loc}’ \\
aquatic &                     ? &                     ? \\
\bottomrule
\end{tabular}

\end{table}

{[}todo: how do these fit in?{]}

\begin{itemize}
\item
  \emph{poye} `above'
\item
  \emph{po} `locative'
\item
  \emph{yatë} `locative'
\item
  \emph{yapo} `negation'?
\item
  allative:
\end{itemize}

\ex \label{histpajirdi-186}
\begingl \glpreamble tichikimuru, peti warë patakasapë Yakucho pana //
\gla ti-chikimu-ru peti warë pataka-sapë Yakucho pana//
\glb \gl{3}-knee-\gl{poss} leg thus exit-\gl{perf}  \gl{all}//
\glft ‘‘su rodilla, su pierna, salió (llaga) hacia Ayacucho’ (personal knowledge
)//
\endgl
\xe

\section{Nonlocative Oblique Postpositions}

\begin{itemize}
\tightlist
\item
  \obj{pana} `\gl{dat}'
\item
  \obj{ke} `\gl{ins}'
\item
  \emph{wanai}
\end{itemize}

\section{Misc}

\begin{itemize}
\tightlist
\item
  \obj{chi} `\gl{cop}' combines with \obj{yawë} `\gl{loc}', sometimes
  spelled \emph{chi yawë}, sometimes \emph{chawë}.
\item
  syllable reduction
\item
  postpositions on bare verbs? (e.g.~\emph{wejtawë})
\end{itemize}

\chapter{\texorpdfstring{Particles, ideophones and interjections
\label{partideo}}{Particles, ideophones and interjections }}

\section{Particles}

Three kinds of particles elsewhere in the family:

\begin{enumerate}
\def\labelenumi{\arabic{enumi}.}
\tightlist
\item
  second position (modals, focus)
\item
  phrasal (focus)
\item
  clause boundary
\end{enumerate}

\begin{itemize}
\tightlist
\item
  prosodic effects?
\end{itemize}

\section{Ideophones}

\begin{itemize}
\tightlist
\item
  constructions with \obj{nwa} `thus'? (example with \emph{pïtï}
  `paint')
\end{itemize}

\section{Interjections}

\begin{itemize}
\tightlist
\item
  kind of particle?
\end{itemize}

\chapter{\texorpdfstring{Negation \label{negation}}{Negation }}

\begin{itemize}
\tightlist
\item
  probably relevant morphemes:

  \begin{itemize}
  \tightlist
  \item
    \obj{-ja} `\gl{neg}'
  \item
    \obj{-jra} `\gl{neg}'
  \item
    \obj{-jnari} `\gl{neg}'
  \item
    \obj{-jrama} `\gl{proh}'
  \item
    \obj{-kempïnirë} `\gl{ptcp}.\gl{nzr}.\gl{gno}:\gl{neg}'
  \item
    \obj{pïnirë} `\gl{neg}'
  \item
    \obj{pïrarë} `\gl{neg}.\gl{exist}'
  \item
    \obj{pïni} `\gl{neg}'
  \item
    \emph{‑yapo} `neg.purp'
  \end{itemize}
\end{itemize}

\chapter{Auxiliarized constructions}

claim: everything can take an auxiliary, except \obj{-kë} `\gl{imp}'

{[}todo: look at frequency and distributional possibilities for various
forms with auxiliaries{]} {[}todo: are there limits on what form of AUX
can occur?{]} {[}todo: conventionalized meanings of combinations?{]}
{[}todo: where is person marking? also alignment{]}

\section{Defining auxiliaries}

\section{Main clauses}

\begin{itemize}
\tightlist
\item
  multiple auxiliaries
\end{itemize}

\section{Subordinate clauses}

\begin{itemize}
\tightlist
\item
  \emph{chi=pëkë}
\item
  \emph{chi=yawë}/\emph{chawë}
\item
  \emph{chi-ripo}
\item
  \emph{wej-tojpe}
\end{itemize}

\chapter{\texorpdfstring{Phrases \label{phrases}}{Phrases }}

\section{\texorpdfstring{Noun phrases
\label{sec:nounphrases}}{Noun phrases }}

TBD

\chapter{\texorpdfstring{Nonverbal predications
\label{nonverbal}}{Nonverbal predications }}

\textcites[366]{gildea2018reconstructing} distinguishes two main formal
types of nonverbal predication in Cariban languages: juxtaposition and
copular constructions.

\begin{itemize}
\tightlist
\item
  juxtaposed NP + ADV or NP + LOC is found in Arara, Ikpeng, Ye'kwana,
  Wayana, and Apalaí
\item
  for PC:

  \begin{itemize}
  \tightlist
  \item
    Nsubj + Npred: nominal (juxtaposition) predication. Limited in
    functional domains.
  \item
    Nsubj + \gl{cop} + Adverbial (adverbs/postpositional phrases).
    Fairly unlimited.
  \end{itemize}
\item
  Innovations:

  \begin{itemize}
  \tightlist
  \item
    Nsubj + \gl{cop} + Npred (S\&M 2009)
  \item
    Nsubj + Adverbial
  \end{itemize}
\end{itemize}

\section{Observations}

\subsection{Patterns}

\begin{itemize}
\tightlist
\item
  ``existential particles'' (\emph{mëtë}, \emph{mïntë}, \emph{entë},
  \emph{ijtë})

  \begin{itemize}
  \tightlist
  \item
    etymologically locatives, used for existential function. they
    \textbf{can} co-occur with the existential negative \obj{pïrarë}
    \exref[]{ex-main-neg-part-pirare-cop-nsubj},
    \exref[]{ex-main-neg-part-pirare-nsubj}
  \item
    also occur in locative function
    \exref[]{loc-main-aff-part-cop-nsubj},
    \exref[]{loc-main-aff-part-nsubj}
  \item
    also combine with postpositional locatives \exref[]{histgrme-107},
    though more often not part of the clause?
    \exref[]{loc-sub-aff-advpred-nsubj-cop}, \exref[]{convfemgrme-157},
    \exref[]{convfemgrme-99}
  \end{itemize}
\item
  the copula\ldots{}

  \begin{itemize}
  \tightlist
  \item
    almost always occurs with adverbs

    \begin{itemize}
    \tightlist
    \item
      counterexample: \exref[]{perm-main-q-advpred-nsubj}
    \item
      not when negated \exref[]{temp-main-neg-nsubj-advpred-jra}
    \end{itemize}
  \item
    can occur with locatives and existential particles

    \begin{itemize}
    \tightlist
    \item
      fairly solid pattern of copula serving as a location for past
      marking existential function: compare mostly past
      \exref[]{ex-main-aff-part-cop-nsubj} with nonpast
      \exref[]{ex-main-aff-part-nsubj}
    \item
      however, no such pattern with locatives: copula
      \exref[]{loc-main-aff-part-cop-nsubj} and no copula
      \exref[]{loc-main-aff-part-nsubj} show no salient distribution
    \end{itemize}
  \item
    almost never occurs with nouns

    \begin{itemize}
    \tightlist
    \item
      only counterexample: \gl{np}\gl{pred} \gl{cop} with `sick'
      \exref[]{temp-main-q-npred-cop}; can occur without \gl{cop}, too
      \exref[]{temp-main-aff-npred-cop}
    \end{itemize}
  \item
    does not combine with \obj{pïrarë} `\gl{neg}.\gl{exist}'

    \begin{itemize}
    \tightlist
    \item
      one counterexample with \gl{part}\gl{pred} + \obj{pïrarë} +
      \gl{cop} + \gl{np}\gl{subj} is with a concessive
      \exref[]{ex-main-neg-part-pirare-cop-nsubj} {[}todo: @spike: I
      forgot, what was your analysis for this particular example?{]}
    \item
      it does occur with \obj{pïnirë} `\gl{neg}', though
      \exref[]{loc-main-neg-nsubj-cop-pinire-part},
      \exref[]{temp-main-neg-advpred-cop-pinire-nsubj} {[}todo: @spike:
      what do you think of the analysis that the copular form is
      synchronically deverbal in all of these examples, licensing the
      nominal negator?{]}
    \end{itemize}
  \item
    is always present in subordinate clauses (\emph{chi-yawë} `when',
    \emph{chi-pëke} `because') {[}todo: @spike: would you analyze these
    subordinate clauses as having a copula? if yes, that would
    constitute a small but solid difference between main and subordinate
    clauses, right?{]}

    \begin{itemize}
    \tightlist
    \item
      even with nominal predicates:
      \exref[]{cat-sub-aff-npred-nsubj-cop},
      \exref[]{temp-sub-aff-npred-nsubj-cop}
    \item
      one example of negation, occurs on additional copula
      \exref[]{loc-sub-neg-locpred-cop-neg-nsubj}
    \item
      one example of \emph{chi chipokono} \exref[]{convamgu-101}
    \end{itemize}
  \end{itemize}
\item
  several negation strategies:

  \begin{itemize}
  \tightlist
  \item
    \obj{-jra} `\gl{neg}' for adverbs and on nouns in the existential
    \obj{pïrarë} \gl{np}\textsubscript{\gl{subj}}\obj{-jra} construction
    \exref[]{ex-main-neg-pirare-nsubj-jra}
  \item
    \obj{pïnirë} `\gl{neg}' for nominal predicates and ones with
    locative particles
  \item
    \obj{pïrarë} `\gl{neg}.\gl{exist}' for existential predicates

    \begin{itemize}
    \tightlist
    \item
      also for identification? \exref[]{id-main-neg-npred-pirare}
    \end{itemize}
  \item
    \obj{-ja} or \obj{-jnari} on the copula
  \end{itemize}
\item
  order is fairly flexible; potentially rigid
  \gl{adv}\textsubscript{\gl{pred}} \gl{np}\textsubscript{\gl{subj}}
  \gl{cop}, though negated counterexample \exref[]{histyarirdi-249}
\item
  unclear role of \obj{manïkï} `\gl{rel}.\gl{anim}' in \gl{np}\gl{pred}
  + \gl{np}\gl{subj} construction
\item
  construction with two copulas \emph{chi wejsapë}?
  \exref[]{convhistfamsjm-92}, \exref[]{convhistfamsjm-59},
  \exref[]{histgrme-17}, \exref[]{histgrme-107} {[}todo: ask natalia{]}
\end{itemize}

\subsection{Categorization issues}

\begin{itemize}
\tightlist
\item
  possessives vs properties can be hard to distinguish (`footed', etc.)
\item
  locative predicates are teeming with ``existential'' particles:
  largely went by `estar' vs `haber' in translation, with the exception
  of \exref[]{convcosnoind-48}
\item
  \exref[]{poss-main-neg-locpred-nsubj} {[}todo: @spike: was this the
  analysis you suggested?{]}
\end{itemize}

\chapter{\texorpdfstring{Simple verbal clauses
\label{simpleverb}}{Simple verbal clauses }}

{[}todo: examples of {[}non{]}declarative w/ {[}in{]}transitive verbs{]}

\begin{itemize}
\tightlist
\item
  order of arguments re: the verb (and each other)
\item
  case marking patterns
\item
  indexation
\item
  clausal particles
\end{itemize}

\section{Intransitive clauses}

\subsection{First person}

\ex  Preverbal pronoun  \\\label{convrisamaj-4}
\begingl \glpreamble wïrë yaruwarijra //
\gla wïrë yaruwa-ri-jra//
\glb \gl{1}\gl{pro} laugh-\gl{ipfv}-\gl{neg}//
\glft ‘I don’t laugh.’ (personal knowledge
)//
\endgl
\xe

\ex  Postverbal pronoun  \\\label{convfemgrme-298}
\begingl \glpreamble këyaja wïrë //
\gla këya-ja wïrë//
\glb be.interested-\gl{neg} \gl{1}\gl{sg}//
\glft ‘‘no sé’’ (personal knowledge
)//
\endgl
\xe

\ex  Prefix  \\\label{convrisamaj-28}
\begingl \glpreamble uyïwïj yawë usenejkari sukuri jwama //
\gla u-y-ïwïj-∅ yawë u-senejka-ri sukuri jwama//
\glb \gl{1}-\gl{lk}-house-\gl{pert} \gl{loc} \gl{1}-stay-\gl{ipfv} silently ***//
\glft ‘I silently stay in my house.’ (personal knowledge
)//
\endgl
\xe

\ex  Preverbal pronoun and prefix  \\\label{ctorat-23}
\begingl \glpreamble aaa usukuru morone ta, wïrë usujta ta ne //
\gla aaa u-suku-ru morone ta wïrë u-sujta-∅ ta ne//
\glb ah \gl{1}-urine-\gl{pert} hurting like \gl{1}\gl{pro} \gl{1}-urinate-\gl{ipfv} like \gl{ints}//
\glft ‘My urine hurts, I will urinate.’ (personal knowledge
)//
\endgl
\xe

\begin{itemize}
\tightlist
\item
  not attested: prefixed verb followed by pronoun
\end{itemize}

\subsection{Second person}

\ex  Prefix  \\\label{convamgu-7}
\begingl \glpreamble tototomo pata yaka mëtëja //
\gla toto-tomo pata-Ø yaka më-të-ja//
\glb non.indian-\gl{pl} town-\gl{poss} into \gl{2}-go-\gl{neg}//
\glft ‘tu no (puedes) ir al pueblo de los criollos’ (personal knowledge
)//
\endgl
\xe

\begin{itemize}
\tightlist
\item
  not attested: pronoun
\end{itemize}

\subsection{Third person}

\ex  Zero  \\\label{ctorat-6}
\begingl \glpreamble wïnïjse //
\gla wïnïj-se//
\glb sleep-\gl{pst}//
\glft ‘He slept.’ (personal knowledge
)//
\endgl
\xe

\ex  Preverbal NP  \\\label{ctorat-16}
\begingl \glpreamble irëjpë tëwï waijtatomo nwajtëri //
\gla irëjpë tëwï waijta-tomo nwajtë-ri//
\glb then \gl{3}\gl{pro} mouse-\gl{pl} dance-\gl{ipfv}//
\glft ‘Then the mice were dancing.’ (personal knowledge
)//
\endgl
\xe

\ex  Postverbal NP  \\\label{histgrme-83}
\begingl \glpreamble wepïrï makë //
\gla wepï-rï makë//
\glb come-\gl{ipfv} mother:\gl{voc}//
\glft ‘‘viene mi mamá’’ (personal knowledge
)//
\endgl
\xe

\ex  Preverbal pronoun  \\\label{desccasagrme-28}
\begingl \glpreamble tëwï nwajtënëpëkë //
\gla tëwï nwajtë-nëpëkë//
\glb \gl{3}\gl{sg} dance-\gl{intr}.\gl{prog}//
\glft ‘‘él... se la pasaba bailando’’ (personal knowledge
)//
\endgl
\xe

\ex  Postverbal pronoun  \\\label{convfemgrme-113}
\begingl \glpreamble ë'ë, tawara takï chapëtiri tëwï //
\gla ë'ë tawara takï chapëti-ri tëwï//
\glb yes thus \gl{cnfrm} scream-\gl{ipfv} \gl{3}\gl{sg}//
\glft ‘‘sí, también grita ese’’ (personal knowledge
)//
\endgl
\xe

\begin{itemize}
\tightlist
\item
  \obj{të} `go' has an irregular third person marker \obj{ij-} `\gl{3}'
\end{itemize}

\ex \label{ctorat-45}
\begingl \glpreamble waraijtokon maniki irëjpë ijtëse //
\gla waraijtokon maniki irëjpë ij-të-se//
\glb man \gl{rel}.\gl{anim} then \gl{3}-go-\gl{pst}//
\glft ‘Then the man went.’ (personal knowledge
)//
\endgl
\xe

\section{Transitive clauses}

There are several factors that play a role here:

\begin{enumerate}
\def\labelenumi{\arabic{enumi}.}
\tightlist
\item
  presence or absence of prefix
\item
  presence or absence of pronouns and NPs
\item
  order of free arguments
\item
  presence or absence of \obj{ya} `\gl{erg}'
\end{enumerate}

\subsection{Third on third}

\ex  preverbal pronoun with ya  \\\label{convcosnoind-132}
\begingl \glpreamble tëwï ya nepïjpë wejsapë //
\gla tëwï ya nepï-jpë wej-sapë//
\glb \gl{3}\gl{sg} \gl{erg} bring-\gl{pst} \gl{cop}-\gl{perf}//
\glft ‘‘él lo trajo’’ (personal knowledge
)//
\endgl
\xe

\chapter{\texorpdfstring{Questions \label{questions}}{Questions }}

TBD

\chapter{\texorpdfstring{Multiclausal
\label{multiclausal}}{Multiclausal }}

\begin{itemize}
\item
  argument of the matrix clause
\item
  adverbial adjunct
\item
  relative clause
\item
  differences \& similarities to simple verb clauses?
\item
  order of arguments re: the verb (and each other)
\item
  case marking patterns
\item
  indexation
\item
  clausal particles
\item
  +mapping between matrix and subordinate
\end{itemize}

\chapter{\texorpdfstring{Word order variation
\label{wordorder}}{Word order variation }}

\section{Transitive clauses}

\ex \label{histyarirdi-615}
\begingl \glpreamble wïrë inija tëwï //
\gla wïrë ini-ja tëwï//
\glb \gl{1}\gl{sg} see-\gl{neg} \gl{3}\gl{sg}//
\glft ‘‘yo no ví eso’’ (personal knowledge
)//
\endgl
\xe

\section{\texorpdfstring{Nonverbal predication
\label{nvp-order}}{Nonverbal predication }}

\chapter{\texorpdfstring{Pragmatically marked constructions
\label{pragmarked}}{Pragmatically marked constructions }}

\begin{itemize}
\tightlist
\item
  participant nominalizations for pseudo-clefts
\end{itemize}

\chapter{\texorpdfstring{Detransitive voice
\label{voice}}{Detransitive voice }}

\begin{itemize}
\tightlist
\item
  functions of \gl{detrz}:

  \begin{itemize}
  \tightlist
  \item
    antipassive
  \item
    passive
  \item
    reflexive
  \item
    reciprocal
  \item
    anticausative
  \end{itemize}
\item
  other strategies for removing participant:

  \begin{itemize}
  \tightlist
  \item
    \emph{-se-mï} `gnomic'
  \item
    \obj{-në} `\gl{inf}'
  \end{itemize}
\item
  what is not used for voice?

  \begin{itemize}
  \tightlist
  \item
    \emph{-sapë}
  \item
    participle
  \end{itemize}
\end{itemize}

\printbibliography

\end{document}