\documentclass{article}
\usepackage{fontspec}
\setmainfont{Brill}
\usepackage[abbrevs=none,refmode=latex]{expex-acro}
\usepackage{booktabs}
\usepackage[style=authoryear]{biblatex}
\addbibresource{/home/florianm/Dropbox/research/cariban/yawarana/yaw_cldf/cldf/sources.bib}
\def\tightlist{}

\title{A digital sketch grammar of Yawarana}
\author{Florian Matter}

\usepackage{hyperref}
\usepackage[capitalise]{cleveref}
\begin{document}
\maketitle

\section{Verbs \label{sec:verbs}}

TODO: write introduction about verbs

\cref{sec:verb-template}

\subsection{Basic morphological template \label{sec:verb-template}}

The following table shows the morpological template of Yawarana verbs,
i.e.~the order in which bound morphemes can occur within a vebral word
form.

\begin{table}
\caption{Basic verb template}
\label{tab:verb_templ}
\centering
\begin{tabular}{lllll}
\toprule
  Prefix & Root &      Aspect &        Tense &     Number \\
\midrule
\obj{i-} &      & \obj{-pëti} &    \obj{-se} & \obj{-jnë} \\
         &      &             &   \obj{-jpë} &            \\
         &      &             & \obj{-tojpe} &            \\
\bottomrule
\end{tabular}

\end{table}

The first slot contains personal prefixes. The relative order of three
tense-aspect-mood suffixes can be seen in the following example:

\ex Yawarana \\
\label{ctorat-40}\begingl
\glpreamble  tipapëjsejne waijtajne //
\gla tipa-pëj-se-jne waijta-jne//
\glb go.group-PLUR-PST-PL mouse-PL//
\glft ‘las ratas se fueron’//  
\endgl 
\xe

More data is needed to firmly establish the relative order of all
suffixes.

\subsection{Person}

\ex Yawarana \\
\label{convrisamaj-28}\begingl
\glpreamble  uyïwïj yawë usenejkari sukuri jwama //
\gla u-y-ïwïj yawë u-s-enejka-ri sukuri jwama//
\glb 1-LK-house LOC 1-DETRZ-watch-NMLZ quietly ***//
\glft ‘yo me quedo en mi casa tranquila’//  
\endgl 
\xe

\ex Yawarana \\
\label{desccasmaj-008}\begingl
\glpreamble  sere mëkase //
\gla sere më-ka-se//
\glb cassava.bread 2-take.out-PTCP//
\glft ‘cuando uno va a arrancar yuca’//  
\endgl 
\xe

\ex Yawarana \\
\label{anfoperso-26}\begingl
\glpreamble  taturu makampra ya ana yoti //
\gla ta-tu-ru makampra ya ana y-oti//
\glb 3P-give-IPFV maternal.aunt:VOC OBL 1+3 LK-game.food//
\glft ‘mi tía lo da comida de nosotros’//  
\endgl 
\xe

\subsection{Aspect}

\subsubsection{The pluractional marker \obj{-pëti}.}

This aspect-marking morpheme is known from other Cariban languages
\parencite{mattiola2020pluractional}. This example illustrates its
\obj{-pëj} allomorph:

\ex Yawarana \\
\label{ctorat-40}\begingl
\glpreamble  tipapëjsejne waijtajne //
\gla tipa-pëj-se-jne waijta-jne//
\glb go.group-PLUR-PST-PL mouse-PL//
\glft ‘las ratas se fueron’//  
\endgl 
\xe

``Cuento de las ratas por AnFo'' And this example illustrates its
\obj{-pëti} allomorph:

\ex Yawarana \\
\label{ctorat-32}\begingl
\glpreamble  waraijtokomo ya inpëti ti //
\gla waraijtokomo ya in-pëti-∅ ti//
\glb man OBL see-PLUR-IPFV QUOT//
\glft ‘el hombre lo estaba viendo’//  
\endgl 
\xe

Yawarana

\subsection{Tense}

\subsubsection{\obj{-jpë}}

\obj{-jpë} is etymologically a nominalizer, but now it also functions as
a simple past marker:

\ex Yawarana \\
\label{anfoperso-02}\begingl
\glpreamble  ana këyetajpë, intipijkë ana chi yawë //
\gla ana këyeta-jpë intipijkë ana chi-∅ yawë//
\glb 1+3 grow.up-PST a.little 1+3 COP-NMLZ LOC//
\glft ‘nos criamos cuando estábamos chiquiticos nosotros’//  
\endgl 
\xe

It also occurs on nouns:

\ex Yawarana \\
\label{anfoperso-17}\begingl
\glpreamble  tawara ma ana këyetajpë, ana papa pan patajpë të //
\gla tawara ma ana këyeta-jpë ana papa pan pata-jpë të//
\glb too RESTR 1+3 grow.up-PST 1+3 father:VOC deceased town-PST LOC//
\glft ‘así nos criamos después de que se murió mi papá en su pueblo’//  
\endgl 
\xe

\subsubsection{The supine}

The suffix \obj{-se} can be used to express a co-called supine meaning,
i.e., purpose of motion. Intransitive verbs occur unprefixed, while
transitive verbs can carry person prefixes.

\ex Yawarana \\
\label{ctorat-03}\begingl
\glpreamble  enijpëtërë waraijtokomo ijtëse ti mïjna tëijpo wïnïjse //
\gla enijpëtërë waraijtokomo ij-të-se ti mïjna tëijpo wïnïj-se//
\glb one man 3-go-PST QUOT there far sleep-SUP//
\glft ‘un hombre se fué allá, a dormir lejos’//  
\endgl 
\xe

\section{Postpositions}

TODO: write

\section{Nouns}

TODO: write

\printbibliography

\end{document}