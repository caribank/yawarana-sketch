\documentclass{article}
\usepackage{fontspec}
\setmainfont{Brill}
\usepackage{glossingtool}
\usepackage{hyperref}
\usepackage{booktabs}
\begin{document}
Title: A digital sketch grammar of Yawarana

\section{Verbs}

This is an example for a simple past verb with \obj{-se}.

\ex<ctorat-42> Yawarana 
\begingl
\glpreamble tëwï ajpachi yaka wonse pïnika tëwï //
\gla tëwï ajpachi yaka won-se pïnika tëwï//
\glb 3SG weeds deer enter-PST PROB 3SG//
\glft ‘tal vez se metió en el monte’//  
\endgl 
\xe

\ex<anfoperso-18> Yawarana 
\begingl
\glpreamble papa pano sëmasaj yawë //
\gla papa pano sëma-saj yawë//
\glb father:VOC CONCL die-PFV LOC//
\glft ‘porque se murió mi papá’//  
\endgl 
\xe

\subsection{Basic morphological template}

\begin{table}
\caption{Basic verb template}
\label{verb_templ}
\centering
\begin{tabular}{lllll}
\toprule
  Prefix & Root &     Aspect &        Tense &      Number \\
\midrule
\obj{i-} &      & \obj{-pëtï} &  \obj{-se} & \obj{-jnë} \\
         &      &            & \obj{-jpë} &             \\
         &      &            &  \obj{-tojpe} &             \\
\bottomrule
\end{tabular}

\end{table}

\subsection{The pluractional marker \obj{-pëtï}}

This aspect-marking morpheme is known from other Cariban languages
\protect\hyperlink{source-mattiola2020pluractional}{Mattiola and Gildea
under review}. This example illustrates its allomorph:

\ex<ctorat-40> Yawarana 
\begingl
\glpreamble tipapëjsejne waijtajne //
\gla tipa-pëj-se-jne waijta-jne//
\glb go.group-PLUR-PST-PL mouse-PL//
\glft ‘las ratas se fueron’//  
\endgl 
\xe

\subsection{\obj{-jpë}}

\obj{-jpë} was originally a nominalizer, but now also functions as a
simple past marker:

\ex<anfoperso-02> Yawarana 
\begingl
\glpreamble ana këyetajpë, intipijkë ana chi yawë //
\gla ana këyeta-jpë intipijkë ana chi-∅ yawë//
\glb 1+3 grow.up-PST a.little 1+3 COP-IPFV LOC//
\glft ‘nos criamos cuando estábamos chiquiticos nosotros’//  
\endgl 
\xe

It also occurs on nouns:

\ex<anfoperso-17> Yawarana 
\begingl
\glpreamble tawara ma ana këyetajpë, ana papa pan patajpë të //
\gla tawara ma-∅ ana këyeta-jpë ana papa pan pata-jpë të-∅//
\glb too throw-IPFV 1+3 grow.up-PST 1+3 father:VOC late axe-PST go-IPFV//
\glft ‘así nos criamos después de que se murió mi papá en su pueblo’//  
\endgl 
\xe

Hellosdasda

\section{Nouns}

Someting about nouns.

\section{References}

\begin{itemize}
\tightlist
\item
  Mattiola, Simone and Gildea, Spike. under review. The pluractional
  marker -pödï of Akawaio (Cariban) and beyond.
\end{itemize}

% 
% \section{Verbs \href{sec:verbs}{label}}

This is an example for a simple past verb with \obj{-se}.

\ex<ctorat-42> Yawarana 
\begingl
\glpreamble tëwï ajpachi yaka wonse pïnika tëwï //
\gla tëwï ajpachi yaka won-se pïnika tëwï//
\glb 3SG weeds deer enter-PST PROB 3SG//
\glft ‘tal vez se metió en el monte’//  
\endgl 
\xe

\ex<anfoperso-18> Yawarana 
\begingl
\glpreamble papa pano sëmasaj yawë //
\gla papa pano sëma-saj yawë//
\glb father:VOC CONCL die-PFV LOC//
\glft ‘porque se murió mi papá’//  
\endgl 
\xe

\subsection{Basic morphological template}

\begin{table}
\caption{Basic verb template}
\label{verb_templ}
\centering
\begin{tabular}{lllll}
\toprule
  Prefix & Root &     Aspect &        Tense &      Number \\
\midrule
\obj{i-} &      & \obj{-pëtï} &  \obj{-se} & \obj{-jnë} \\
         &      &            & \obj{-jpë} &             \\
         &      &            &  \obj{-tojpe} &             \\
\bottomrule
\end{tabular}

\end{table}

\subsection{The pluractional marker \obj{-pëtï}}

This aspect-marking morpheme is known from other Cariban languages
\href{mattiola2020pluractional}{psrc}. This example illustrates its
\obj{-pëj} allomorph:

\ex<ctorat-40> Yawarana 
\begingl
\glpreamble tipapëjsejne waijtajne //
\gla tipa-pëj-se-jne waijta-jne//
\glb go.group-PLUR-PST-PL mouse-PL//
\glft ‘las ratas se fueron’//  
\endgl 
\xe

\subsection{\obj{-jpë}}

\obj{-jpë} was originally a nominalizer, but now also functions as a
simple past marker:

\ex<anfoperso-02> Yawarana 
\begingl
\glpreamble ana këyetajpë, intipijkë ana chi yawë //
\gla ana këyeta-jpë intipijkë ana chi-∅ yawë//
\glb 1+3 grow.up-PST a.little 1+3 COP-IPFV LOC//
\glft ‘nos criamos cuando estábamos chiquiticos nosotros’//  
\endgl 
\xe

It also occurs on nouns:

\ex<anfoperso-17> Yawarana 
\begingl
\glpreamble tawara ma ana këyetajpë, ana papa pan patajpë të //
\gla tawara ma-∅ ana këyeta-jpë ana papa pan pata-jpë të-∅//
\glb too throw-IPFV 1+3 grow.up-PST 1+3 father:VOC late axe-PST go-IPFV//
\glft ‘así nos criamos después de que se murió mi papá en su pueblo’//  
\endgl 
\xe
% 
% Hellosdasda
% 
% \section{Nouns}

Someting about nouns.
% 
% \input{possession.tex}
% 

\end{document}