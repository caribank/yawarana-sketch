\documentclass{memoir}
\setsecnumdepth{subsubsection}
\usepackage{tikz}
\usepackage{geometry}
\usepackage{xcolor}
\usepackage[some]{background}

\definecolor{titlepagecolor}{RGB}{208, 84, 0}

\backgroundsetup{
scale=1,
angle=0,
opacity=1,
contents={\begin{tikzpicture}[remember picture,overlay]
 \path [fill=titlepagecolor] (-0.5\paperwidth,5) rectangle (0.5\paperwidth,10);  
\end{tikzpicture}}
}

\font\hugefont="Brill" at 38pt

\usepackage{fontspec}
\setmainfont{Brill}
\usepackage[abbrevs=none,refmode=latex]{expex-acro}
\usepackage{booktabs}
\usepackage[style=authoryear]{biblatex}
\usepackage[textwidth=30mm]{todonotes}
\def\tightlist{}
\usepackage{longtable}
\usepackage{hyperref}
\usepackage[capitalise]{cleveref}

\lingset{everygla=\itshape, belowglpreambleskip=0ex, aboveglftskip=0ex}

\title{A digital sketch grammar of Yawarana}
\author{Florian Matter;Natalia Cáceres Arandia;Spike Gildea}

\newGlossingAbbrev{1}{first person}
\newGlossingAbbrev{2}{second person}
\newGlossingAbbrev{3}{third person}
\newGlossingAbbrev{1+2}{first person inclusive}
\newGlossingAbbrev{1+3}{first person exclusive}
\newGlossingAbbrev{a}{agent-like argument}
\newGlossingAbbrev{abs}{absolutive}
\newGlossingAbbrev{all}{allative}
\newGlossingAbbrev{anim}{animate}
\newGlossingAbbrev{circ}{circumstantive}
\newGlossingAbbrev{cncs}{concessive}
\newGlossingAbbrev{concl}{conclusive}
\newGlossingAbbrev{contrast}{contrastive}
\newGlossingAbbrev{cop}{copula}
\newGlossingAbbrev{dat}{dative}
\newGlossingAbbrev{dem}{demonstrative}
\newGlossingAbbrev{des}{desiderative}
\newGlossingAbbrev{detrz}{detransivizer}
\newGlossingAbbrev{dim}{diminutive}
\newGlossingAbbrev{dist}{distal}
\newGlossingAbbrev{emp}{emphatic}
\newGlossingAbbrev{erg}{ergative}
\newGlossingAbbrev{ess}{essive}
\newGlossingAbbrev{fut}{future}
\newGlossingAbbrev{hsy}{hearsay evidentiality}
\newGlossingAbbrev{imn}{imminent}
\newGlossingAbbrev{imp}{imperative}
\newGlossingAbbrev{inan}{inanimate}
\newGlossingAbbrev{inf}{infinitive}
\newGlossingAbbrev{ins}{instrumental}
\newGlossingAbbrev{intr}{intransitive}
\newGlossingAbbrev{ints}{intensifier}
\newGlossingAbbrev{ipfv}{imperfective}
\newGlossingAbbrev{lk}{linker}
\newGlossingAbbrev{loc}{locative}
\newGlossingAbbrev{med}{medial}
\newGlossingAbbrev{mot}{motion}
\newGlossingAbbrev{motimp}{motion imperative}
\newGlossingAbbrev{neg}{negation}
\newGlossingAbbrev{nmlz}{nominalizer}
\newGlossingAbbrev{npert}{unpossessed}
\newGlossingAbbrev{nposs}{nonpossessed}
\newGlossingAbbrev{p}{patient-like argument}
\newGlossingAbbrev{pert}{pertensive}
\newGlossingAbbrev{pfv}{perfective}
\newGlossingAbbrev{pl}{plural}
\newGlossingAbbrev{plur}{pluractional}
\newGlossingAbbrev{poss}{possession}
\newGlossingAbbrev{priv}{privative}
\newGlossingAbbrev{pro}{pronoun}
\newGlossingAbbrev{prob}{probabilitive}
\newGlossingAbbrev{prog}{progressive}
\newGlossingAbbrev{proh}{prohibitive}
\newGlossingAbbrev{prox}{proximal}
\newGlossingAbbrev{pst}{past}
\newGlossingAbbrev{quot}{quotative}
\newGlossingAbbrev{rst}{restrictive}
\newGlossingAbbrev{s}{intransitive argument}
\newGlossingAbbrev{sg}{singular}
\newGlossingAbbrev{tr}{transitive}
\newGlossingAbbrev{vbz}{verbalizer}

\begin{document}

\begin{titlingpage}
\BgThispage
\newgeometry{left=1cm,right=1cm}
\vspace*{2cm}
\centering
\textcolor{white}{ \hugefont A digital sketch grammar of Yawarana }
\vspace*{3cm}\par
\noindent
{
\raggedleft
\begin{minipage}{0.90\linewidth}
    \begin{flushright}
        
{\Huge Florian Matter }\\[\baselineskip]

{\Huge Natalia Cáceres Arandia }\\[\baselineskip]

{\Huge Spike Gildea }\\[\baselineskip]

    \end{flushright}
\end{minipage} \hspace{15pt}
}
\centering
\vfill
\rule{0.4\textwidth}{0.4pt}\\
{\Huge 2023 \\ \large pylingdocs }
\end{titlingpage}


\tableofcontents

\chapter{\texorpdfstring{Introduction \label{intro}}{Introduction }}

\section{\texorpdfstring{The Yawarana people and their language
\label{sec:people}}{The Yawarana people and their language }}

\section{\texorpdfstring{Location, historical records
\label{sec:context}}{Location, historical records }}

\section{\texorpdfstring{Current life
\label{sec:currentlife}}{Current life }}

\section{\texorpdfstring{Sociolinguistic vitality
\label{sec:vitality}}{Sociolinguistic vitality }}

\section{\texorpdfstring{Previous studies on the Yawarana language
\label{sec:previous}}{Previous studies on the Yawarana language }}

\section{\texorpdfstring{This project
\label{sec:thisproject}}{This project }}

\chapter{\texorpdfstring{Phonetics and phonology
\label{phono}}{Phonetics and phonology }}

\section{\texorpdfstring{Segmental phonetics and phonemes
\label{sec:segmental}}{Segmental phonetics and phonemes }}

The consonant phonemes of Yawarana are shown in \cref{tab:consonants},
and the vowels in \cref{tab:vowels}.

\begin{table}
\caption{Consonant phonemes}
\label{tab:consonants}
\centering
\begin{tabular}{llllll}
\toprule
          & bilabial & alveolar & palatal & velar & glottal \\
\midrule
occlusive &     /p/  &     /t/  &  /t͡ʃ/  &   /k/ &         \\
    nasal &     /m/  &     /n/  &    /ɲ/  &       &         \\
fricative &          &     /s/  &         &       &    /h/  \\
   liquid &          &     /r/  &         &       &         \\
    glide &     /w/  &          &     /j/ &       &         \\
\bottomrule
\end{tabular}

\end{table}

\begin{table}
\caption{Vowel phonemes}
\label{tab:vowels}
\centering
\begin{tabular}{llll}
\toprule
      & front & central & back \\
\midrule
close &  /i/  &    /ɨ/  & /u/  \\
  mid &  /e/  &    /ə/  & /o/  \\
 open &       &    /a/  &      \\
\bottomrule
\end{tabular}

\end{table}

\subsection{\texorpdfstring{Consonants
\label{sec:consonants}}{Consonants }}

\subsection{\texorpdfstring{Vowels \label{sec:vowels}}{Vowels }}

\section{\texorpdfstring{Morphophonological Processes
\label{sec:morphophono}}{Morphophonological Processes }}

\subsection{\texorpdfstring{Syllable Reduction
\label{sec:sylred}}{Syllable Reduction }}

\subsection{\texorpdfstring{Vowel harmony?
\label{sec:vowelharm}}{Vowel harmony? }}

\section{\texorpdfstring{Prosody \label{sec:prosody}}{Prosody }}

\subsection{\texorpdfstring{Lexical stress
\label{sec:stress}}{Lexical stress }}

\subsection{\texorpdfstring{Intonational Phrases
\label{sec:intphrases}}{Intonational Phrases }}

\subsection{\texorpdfstring{Intonational Melodies
\label{sec:intmelodies}}{Intonational Melodies }}

\section{\texorpdfstring{Historical Considerations
\label{sec:histphono}}{Historical Considerations }}

\chapter{\texorpdfstring{Distinguishing parts of speech in Yawarana
\label{POS}}{Distinguishing parts of speech in Yawarana }}

\chapter{\texorpdfstring{Nouns \label{nouns}}{Nouns }}

\section{\texorpdfstring{Pronouns \label{sec:pronouns}}{Pronouns }}

The personal pronouns of Yawarana are shown in \cref{tab:pronouns}. It
shows the usual Cariban inclusive/exclusive (\gl{1+2} and \gl{1+3})
distinction. Note that the plural marker \obj{-kontomo} appears to
usually be restricted to verbs, while \emph{-santomo} is found with
third person pronouns and demonstratives.

\begin{table}
\caption{Pronouns}
\label{tab:pronouns}
\centering
\begin{tabular}{lll}
\toprule
         &    \gl{sg} &           \gl{pl} \\
\midrule
  \gl{1} & \obj{wïrë} &                   \\
\gl{1+2} &            &        \obj{ejnë} \\
\gl{1+3} &            &         \obj{ana} \\
  \gl{2} & \obj{mërë} &  \obj{monkontomo} \\
  \gl{3} & \obj{tëwï} & \obj{tëwïsantomo} \\
\bottomrule
\end{tabular}

\end{table}

Short forms of the first and second person pronouns can occur as
proclitics attaching to nouns to indicate possessor
(\cref{sec:nominalperson}), attached to verbs to indicate subject or
object (described in \cref{verbinfl}), or attached to postpositions to
indicate object of the postposition (described in \cref{sec:postinfl}):

\ex Yawarana \\
\label{convrisamaj-28}    \begingl
    \glpreamble  uyïwïj yawë usenejkari sukuri jwama //
    \gla u-y-ïwïj yawë u-senejka-ri suku-ri jwama//
    \glb \gl{1}-\gl{lk}-house \gl{loc} \gl{1}-remain-\gl{ipfv} urine-\gl{pert} ***//
        \glft ‘I silently stay in my house.’//  
    \endgl 
\xe

\ex Yawarana \\
\label{desccasmaj-025}    \begingl
    \glpreamble  mënai wëjkase chijpë wararë //
    \gla më-nai-∅ wëjkase chi-jpë wararë//
    \glb \gl{2}-do-\gl{ipfv} *** \gl{cop}-\gl{pst} ***//
        \glft ‘se cayó tu cosa’//  
    \endgl 
\xe

\ex Yawarana \\
\label{convrisamaj-02}    \begingl
    \glpreamble  mëyaruwari, mëpëkëpene //
    \gla më-yaruwa-ri më-pëkëpene//
    \glb \gl{2}-laugh-\gl{ipfv} \gl{2}-alone//
        \glft ‘You just laugh.’//  
    \endgl 
\xe

\ex Yawarana \\
\label{ctoaragrme-07}    \begingl
    \glpreamble  moyochi //
    \gla moyochi//
    \glb ***//
        \glft ‘la araña’//  
    \endgl 
\xe

An open question is whether \obj{ta-} on verbs is a reduction of
\obj{tëwï}.

The third person demonstrative pronouns or articles are shown in
\cref{tab:pronouns3}.

\begin{table}
\caption{Demonstrative pronouns / articles}
\label{tab:pronouns3}
\centering
\begin{tabular}{lllll}
\toprule
              & \multicolumn{2}{l}{\gl{anim}} & \multicolumn{2}{l}{\gl{inan}} \\
\midrule
              &     \gl{sg} &           \gl{pl} &                 \gl{sg} &                              \gl{pl} \\
    \gl{prox} &  \obj{kërë} & \obj{kërësantomo} &               \obj{eni} &                         \obj{enijne} \\
medial? near? & \obj{michi} &                   & \obj{misi} / \obj{mërë} & \obj{michisantomo} / \obj{michitomo} \\
    \gl{dist} & \obj{mëjkï} & \obj{mëkïsantomo} &             \obj{mëjnï} &                       \obj{mëjnijne} \\
\bottomrule
\end{tabular}

\end{table}

None of the demonstrative pronouns have shortened, phonologically bound
counterparts.

\begin{itemize}
\tightlist
\item
  Nominal Interrogative pronouns:

  \begin{itemize}
  \tightlist
  \item
    \obj{anïkï} `who?'
  \item
    \obj{ati} `what?'
  \item
    \emph{ëjkë} `which? inan'
  \end{itemize}
\end{itemize}

\section{\texorpdfstring{Nominal inflection
\label{sec:nouninfl}}{Nominal inflection }}

Nouns in Yawarana may bear suffixes for possession
(\cref{sec:nounposssuf}) and number (\cref{sec:nominalnumber}), and
possessed nouns may bear a third person prefix, indexing a third person
possessor, or a first or second proclitic (a reduced form of the free
pronoun), indexing a first or second person possessor
(\cref{sec:nominalperson}).

\subsection{\texorpdfstring{Suffixes for possessed and non-possessed
nouns
\label{sec:nounposssuf}}{Suffixes for possessed and non-possessed nouns }}

In the possession construction in Yawarana, the possessor noun occurs
immediately preceding the possessed noun, which is the head of the
possession phrase. Alternatively, the possessor can appear as a bound
pronominal clitic (first \& second person) or a prefix (third person) on
the possessed noun. The possessor noun is never marked (for instance,
with genitive case), but the possessed noun (the head) is often marked
by a lexically specified `possessed' suffix, either \obj{-ru}
`\gl{pert}' or \obj{-ti} `pos'. Unpossessed nouns generally are
unmarked, but some 15 nouns bear the suffix \obj{-të} `\gl{npert}' when
they appear without a possessor. Examples
\exref[][unsuffixednouns]{onlypossessed} illustrate the possible
patterns of markedness for nouns when possessed and unpossessed. In the
first category, which contains the vast majority of nouns in our corpus,
the unpossessed noun is unmarked, but when possessed the suffix -ri
`pos' occurs \exref[]{onlypossessed}. A handful of nouns is marked with
-ri/-ti `pos' when possessed and with -të `npos' when not possessed
\exref[]{diffpossessed}. Another handful is unmarked when possessed and
marked with -të when not possessed \exref[]{suffunpossessed}. The fourth
logical possibility, in which the noun bears no marker of possession (or
non-possession) whether possessed or unpossessed, contains very few
members (only one attested so far); in this case, the difference is
marked only by the presence or absence of a possessive prefix or
free-form possessor \exref[]{unsuffixednouns}.

\ex\label{onlypossessed} Nouns that take a suffix only when possessed:

\begin{tabular}[t]{llll}

 \emph{akajra-ri} &          ‘X’s bow’ & \emph{akajra} &          ‘bow’ \\

\emph{y-amaka-ri} &        ‘X’s yucca’ &  \emph{amaka} &        ‘yucca’ \\
 \emph{y-ántë-ri} &     ‘X’s fishhook’ &   \emph{antë} &     ‘fishhook’ \\
\emph{y-ateri-ri} & ‘X’s garden/field’ &  \emph{ateri} & ‘garden/field’ \\
    \emph{ënu-ru} &          ‘X’s eye’ &    \emph{ënu} &          ‘eye’ \\
  \emph{y-ëpi-ri} &     ‘X’s medicine’ &    \emph{ëpi} &     ‘medicine’ \\

\end{tabular}
 \xe

\ex\label{diffpossessed} Nouns that take one suffix when possessed and
another when unpossessed

\begin{tabular}[t]{llll}

   \emph{yë-ri} & ‘X’s tooth’ &   \emph{yë-të} &                                ‘tooth’ \\

 \emph{pata-ri} & ‘X’s place’ & \emph{pata-të} & ‘(part of name) San Juan de Manapiare’ \\
\emph{y-ese-ti} & ‘X’s name’  &  \emph{ese-të} &                                 ‘name’ \\
\emph{y-ase-tï} & ‘X’s cord’  &  \emph{ase-të} &                                 ‘cord’ \\

\end{tabular}
 \xe

\ex\label{suffunpossessed} Nouns that take a suffix only when
unpossessed:

\begin{tabular}[t]{llll}

  \emph{yëjpë} &  ‘X’s bone’ &                \emph{yëjpë-të} &  ‘bone’ \\

   \emph{petï} & ‘X’s thigh’ & \emph{petï-të} / \emph{pej-të} & ‘thigh’ \\
\emph{y-aponi} & ‘X’s stool’ &                 \emph{apon-të} & ‘stool’ \\

\end{tabular}
 \xe

\ex\label{unsuffixednouns} Nouns that never take a suffix, whether
possessed or unpossessed:

\begin{tabular}[t]{llll}

\emph{i-jmëy} & 'his egg’ & \emph{ëjmëy} & 'egg’ \\

\end{tabular}
 \xe

\subsection{\texorpdfstring{Number suffixes
\label{sec:nominalnumber}}{Number suffixes }}

There are two plural suffixes that can occur on nouns, apparently freely
interchangeable. What conditions the choice of suffix is not clear as of
yet.

\ex Yawarana \\
\label{ctorat-17}    \begingl
    \glpreamble  waijtatomo ëjwenakase //
    \gla waijta-tomo ëj-wenaka-se//
    \glb mouse-\gl{pl} \gl{detrz}-vomit-\gl{pst}//
        \glft ‘The mice vomited.’//  
    \endgl 
\xe

\ex Yawarana \\
\label{ctorat-40}    \begingl
    \glpreamble  tipapëjsejne waijtajne //
    \gla tipa-pëj-se-jne waijta-jne//
    \glb go.in.group-\gl{plur}-\gl{pst}-\gl{pl} mouse-\gl{pl}//
        \glft ‘the mice went away.’//  
    \endgl 
\xe

\subsection{\texorpdfstring{Argument prefixes
\label{sec:nominalperson}}{Argument prefixes }}

Person prefixes on nouns are conditioned by the initial segment
(\cref{tab:possprefixes}). C-initial nouns take third person \obj{i-},
and first and second person are bare \obj{u-} and \obj{më-}. On
V-initial nouns, third person is marked by \obj{t-}, and first and
second person combine with the linker \obj{y-}. Some examples are shown
in \exref[][lastex]{ctorat-23}.

\begin{table}
\caption{Possessive prefixes on nouns}
\label{tab:possprefixes}
\centering
\begin{tabular}{lll}
\toprule
       &       \_C &               \_V \\
\midrule
\gl{1} &  \obj{u-} &  \obj{u-}\obj{y-} \\
\gl{2} & \obj{më-} & \obj{më-}\obj{y-} \\
\gl{3} &  \obj{i-} &          \obj{t-} \\
\bottomrule
\end{tabular}

\end{table}

\ex Yawarana \\
\label{ctorat-23}    \begingl
    \glpreamble  aaa usukuru morone ta wïrë usujta ta ne //
    \gla aaa u-suku-ru morone ta wïrë u-suj-ta-∅ ta ne//
    \glb *** \gl{1}-urine-\gl{pert} hurting like \gl{1}\gl{pro} \gl{1}-urine-\gl{vbz}-\gl{ipfv} like \gl{ints}//
        \glft ‘My urine hurts, I will urinate.’//  
    \endgl 
\xe

\ex Yawarana \\
\label{convrisamaj-28}    \begingl
    \glpreamble  uyïwïj yawë usenejkari sukuri jwama //
    \gla u-y-ïwïj yawë u-senejka-ri suku-ri jwama//
    \glb \gl{1}-\gl{lk}-house \gl{loc} \gl{1}-remain-\gl{ipfv} urine-\gl{pert} ***//
        \glft ‘I silently stay in my house.’//  
    \endgl 
\xe

\ex Yawarana \\
\label{desccasmaj-025}    \begingl
    \glpreamble  mënai wëjkase chijpë wararë //
    \gla më-nai-∅ wëjkase chi-jpë wararë//
    \glb \gl{2}-do-\gl{ipfv} *** \gl{cop}-\gl{pst} ***//
        \glft ‘se cayó tu cosa’//  
    \endgl 
\xe

\ex Yawarana \\
\label{ctorat-46}    \begingl
    \glpreamble  tïwïj yaka waraijtokomo manikijpë //
    \gla t-ïwïj yaka waraijtokomo maniki-jpë//
    \glb \gl{3}-house \gl{all} man ?-\gl{pst}//
        \glft ‘He went to his house.’//  
    \endgl 
\xe

\ex Yawarana \\
\label{lastex}    \begingl
    \glpreamble  pïrarë ti iwenaru wejsapë //
    \gla pïrarë ti i-wena-ru wej-sapë//
    \glb nothing \gl{hsy} \gl{3}-vomit-\gl{pert} \gl{cop}-\gl{pfv}//
        \glft ‘Their vomit was not there.’//  
    \endgl 
\xe

The linker also occurs with (pro-)nominal possessors:

\ex Yawarana \\
\label{desccasmaj-131}    \begingl
    \glpreamble  ejnë yemekunu //
    \gla ejnë yemekunu//
    \glb \gl{1+2}\gl{pro} ***//
        \glft ‘la mano de uno’//  
    \endgl 
\xe

There are some nouns (kinship terms?) that take an apparently older old
second person \obj{a-} (\cref{tab:oldpossprefixes}).

\begin{table}
\caption{Archaic possessive prefixes on nouns}
\label{tab:oldpossprefixes}
\centering
\begin{tabular}{lll}
\toprule
       &      \_C &              \_V \\
\midrule
\gl{1} & \obj{u-} & \obj{u-}\obj{y-} \\
\gl{2} & \obj{a-} & \obj{a-}\obj{y-} \\
\gl{3} & \obj{i-} &         \obj{t-} \\
\bottomrule
\end{tabular}

\end{table}

\subsubsection{irregularly inflected nouns:}

\begin{itemize}
\tightlist
\item
  `father':

  \begin{itemize}
  \tightlist
  \item
    1 \emph{papa}
  \item
    2 \emph{ëmë} / \emph{omo} / \emph{ëmo} (?)
  \item
    3 \emph{imu}
  \item
    NP \emph{yïmï}
  \end{itemize}
\end{itemize}

\section{\texorpdfstring{Nominal Derivational Morphology
\label{sec:nounderiv}}{Nominal Derivational Morphology }}

\begin{itemize}
\tightlist
\item
  V → N

  \begin{itemize}
  \tightlist
  \item
    \obj{-ri} `action \gl{nmlz}'
  \item
    \obj{-jpë}

    \begin{itemize}
    \tightlist
    \item
      `\gl{pst}.\gl{abs}.\gl{nmlz}'
    \item
      `\gl{pst}.ACT.\gl{nmlz}?'
    \end{itemize}
  \item
    ?\obj{-në} `\gl{inf} / generic action nominalizer'

    \begin{itemize}
    \tightlist
    \item
      Not only on intransitive verbs? see \emph{wanumanë} `gossip, lie'
      and \emph{wajtënë} `dance'
    \end{itemize}
  \item
    \obj{-ni} `\gl{a}.\gl{nmlz}'
  \item
    \obj{n-} `\gl{p}.\gl{nmlz}'

    \begin{itemize}
    \tightlist
    \item
      \obj{n-}V\obj{-ri} `nonpast?'
    \item
      ?? \obj{n-}V\obj{-jpë} `past?'
    \end{itemize}
  \item
    \obj{-sapë} `\gl{abs}.\gl{nmlz}' (contrast with ‑jpë )
  \item
    \obj{-topo} `\gl{circ}.\gl{nmlz}'
  \item
    \obj{‑pïnï} `\gl{priv}.\gl{nmlz}' ?
  \end{itemize}
\item
  Adv → N

  \begin{itemize}
  \tightlist
  \item
    \obj{-mï} `\gl{nmlz}'
  \end{itemize}
\item
  Postp → N

  \begin{itemize}
  \tightlist
  \item
    \obj{-ano} `\gl{nmlz}'
  \end{itemize}
\item
  What about \obj{-jpë} on AD forms? Does it derive a noun?
\end{itemize}

\chapter{\texorpdfstring{Verbal inflection
\label{verbinfl}}{Verbal inflection }}

\section{\texorpdfstring{Person prefixes
\label{sec:verbperson}}{Person prefixes }}

Verbs are inflected for person with a set of prefixes, shown in
\cref{tab:verbprefixes}. First and second person prefixes show
accusative alignment, expressing \gl{s} and \gl{p}. Intransitive verbs
are not overtly inflected for third person, while transitive verbs show
an optional \obj{ta-} in \gl{3}\textgreater{}\gl{3} scenarios. An
exception to this is the verb \obj{të} `to go', which shows an
idiosyncratic prefix \obj{ij-}.

\begin{table}
\caption{Person marking prefixes on verbs}
\label{tab:verbprefixes}
\centering
\begin{tabular}{lll}
\toprule
       & \gl{intr} &   \gl{tr} \\
\midrule
\gl{1} &  \obj{u-} &  \obj{u-} \\
\gl{2} & \obj{më-} & \obj{më-} \\
\gl{3} &         ∅ & \obj{ta-} \\
\bottomrule
\end{tabular}

\end{table}

\begin{itemize}
\tightlist
\item
  \obj{ta-} `\gl{3}\gl{p}' attested on one V in the pan‑Cariban
  ``progressive'' construction w/ 2nd person \gl{a}
\item
  Ø‑ `\gl{3}\gl{p}' with transitive verbs with \gl{1}\gl{a} or
  \gl{2}\gl{a}; also sometimes \gl{3}\gl{a}
\item
  one example of (\obj{më-}) `\gl{2}\gl{a}' on imperative verb
\item
  Note that all transitive verbs are consonant‑initial, whether
  etymologically or not because \obj{y-} `\gl{lk}' is added to all
  vowel‑initial roots

  \begin{itemize}
  \tightlist
  \item
    the \emph{y‑} disappears when preceded by the detransitivizer
  \end{itemize}
\end{itemize}

\section{Non‑personal inflectional prefixation}

\begin{itemize}
\tightlist
\item
  *\emph{t‑V‑se} is no more --- the t‑ is gone (except with
  \emph{t-ënë-se} `eat' and \emph{t-eni-se} `drink')
\item
  Any divergent forms with the negative?

  \begin{itemize}
  \tightlist
  \item
    \emph{i‑} with intransitive negatives?
  \item
    \emph{an‑} with transitive negatives?
  \end{itemize}
\end{itemize}

\section{\texorpdfstring{Tense‑aspect‑mood‑polarity suffixes
\label{sec:tam}}{Tense‑aspect‑mood‑polarity suffixes }}

Verbs in main clauses are inflected for TAMP with a set of suffixes,
shown in \cref{tab:verbtam}. They are discussed in
\crefrange{sec:riipfv}{sec:sareimn}.

\begin{table}
\caption{Verbal TAM suffixes}
\label{tab:verbtam}
\centering
\begin{tabular}{ll}
\toprule
       Suffix &            Function \\
\midrule
    \obj{-ri} &        imperfective \\
   \obj{-jpë} &                past \\
    \obj{-se} &             past 2? \\
  \obj{-sapë} &         perfective? \\
  \obj{-sarë} &     imminent future \\
\obj{-tëpëkë} & \gl{prog}.\gl{intr} \\
   \obj{pëkë} &   \gl{prog}.\gl{tr} \\
\bottomrule
\end{tabular}

\end{table}

\begin{itemize}
\item
  \obj{-ja} `\gl{neg}'
\item
  \emph{‑jrama} `\gl{proh}'
\item
  \emph{‑tojpano} `\gl{fut}'
\item
  \obj{-se}=\obj{pano} `\gl{pst}=\gl{concl}'
\item
  \obj{-saj}=\obj{pano} `\gl{pfv}=\gl{concl}'
\item
  imperatives:

  \begin{itemize}
  \tightlist
  \item
    \obj{-kë} / ‑\emph{të}\obj{-kë} `\gl{imp} / \gl{imp}.\gl{pl}'
  \item
    \obj{-ta} / \obj{-ta}\emph{ntë}\obj{-kë} `\gl{imp}.\gl{mot} /
    \gl{imp}.\gl{mot}.\gl{pl}'
  \end{itemize}
\end{itemize}

\subsection{\texorpdfstring{\obj{-ri} \label{sec:riipfv}}{ }}

\begin{itemize}
\tightlist
\item
  allomorphy:

  \begin{itemize}
  \tightlist
  \item
    \obj{-∅}, phonetic loss
  \item
    \obj{-ru}, assimilation
  \item
    what about \obj{-rï}? Looks like the original one\ldots{}
  \end{itemize}
\item
  diachrony: from the nominalizer \obj{-ri}
\item
  combines with \obj{-jra}:
\end{itemize}

\ex Yawarana \\
\label{convrisamaj-04}    \begingl
    \glpreamble  wïrë yaruwarijra\#\#\# //
    \gla wïrë yaruwa-ri-jra//
    \glb \gl{1}\gl{pro} laugh-\gl{ipfv}-\gl{neg}//
        \glft ‘I don’t laugh.’//  
    \endgl 
\xe

\subsubsection{Semantics}

\begin{itemize}
\tightlist
\item
  not specified for tense, just imperfective aspect:

  \begin{itemize}
  \tightlist
  \item
    past \exref[]{ctorat-16}
  \item
    future \exref[]{convrisamaj-06}
  \item
    gnomic/present? \exref[]{gnomicri}
  \end{itemize}
\end{itemize}

\ex Yawarana \\
\label{ctorat-16}    \begingl
    \glpreamble  irëjpë tëwï waijtatomo nwajtëri //
    \gla irëjpë tëwï waijta-tomo nwajtë-ri//
    \glb then \gl{3}\gl{pro} mouse-\gl{pl} dance-\gl{ipfv}//
        \glft ‘Then the mice were dancing.’//  
    \endgl 
\xe

\ex Yawarana \\
\label{convrisamaj-06}    \begingl
    \glpreamble ¿ kwase ejnë yaruwari? //
    \gla kwase ejnë yaruwa-ri//
    \glb how \gl{1+2}\gl{pro} laugh-\gl{ipfv}//
        \glft ‘How will we laugh?’//  
    \endgl 
\xe

\pex\label{gnomicri}    \a Yawarana\\
    \label{convrisamaj-04}        \begingl
        \glpreamble  wïrë yaruwarijra\#\#\# //
        \gla wïrë yaruwa-ri-jra//
        \glb \gl{1}\gl{pro} laugh-\gl{ipfv}-\gl{neg}//
            \glft ‘I don’t laugh.’//  
        \endgl 
    \a Yawarana\\
    \label{convrisamaj-28}        \begingl
        \glpreamble  uyïwïj yawë usenejkari sukuri jwama //
        \gla u-y-ïwïj yawë u-senejka-ri suku-ri jwama//
        \glb \gl{1}-\gl{lk}-house \gl{loc} \gl{1}-remain-\gl{ipfv} urine-\gl{pert} ***//
            \glft ‘I silently stay in my house.’//  
        \endgl 
\xe

\subsection{\texorpdfstring{\obj{-jpë}}{}}

\begin{itemize}
\tightlist
\item
  allomorphy: none?
\item
  diachrony: from nominalizer \obj{-jpë}
\end{itemize}

\subsection{\texorpdfstring{\obj{-se}}{}}

\begin{itemize}
\tightlist
\item
  allomorphy: \obj{-se}/\obj{-che}
\item
  diachrony: from participle \obj{-se}
\end{itemize}

\subsection{\texorpdfstring{\obj{-sapë}}{}}

\begin{itemize}
\tightlist
\item
  diachrony: from nominalizer \obj{-sapë}
\item
  distribution: only occurs on the copula?
\item
  allomorphy: \obj{-sapë} and \obj{-saj}
\item
  negation: with \obj{-ja} on lexical verb
  \exref[][ctoaragrme-40]{ctoaragrme-38}
\end{itemize}

\ex Yawarana \\
\label{ctoaragrme-38}    \begingl
    \glpreamble  irë wejtane mujyampe patakaja wejsapë //
    \gla irë wej-tane mujyampe patakaja wej-sapë//
    \glb \gl{dem} \gl{cop}-\gl{cncs} *** *** \gl{cop}-\gl{pfv}//
        \glft ‘a pesar de eso no salió embarazada’//  
    \endgl 
\xe

\ex Yawarana \\
\label{ctoaragrme-39}    \begingl
    \glpreamble  apatakaja pïnïka wejsapë //
    \gla apatakaja pïnïka wej-sapë//
    \glb *** \gl{prob} \gl{cop}-\gl{pfv}//
        \glft ‘tal vez no salió (embarazada)’//  
    \endgl 
\xe

\ex Yawarana \\
\label{ctoaragrme-40}    \begingl
    \glpreamble  tayakïjtëja pïnika wejsapë //
    \gla tayakïjtëja pïnika wej-sapë//
    \glb *** \gl{prob} \gl{cop}-\gl{pfv}//
        \glft ‘tal vez no se acostó con ella’//  
    \endgl 
\xe

\begin{itemize}
\tightlist
\item
  what about \exref[]{ctorat-19}? is that existential negation?
\end{itemize}

\ex Yawarana \\
\label{ctorat-19}    \begingl
    \glpreamble  pïrarë ti iwenaru wejsapë //
    \gla pïrarë ti i-wena-ru wej-sapë//
    \glb nothing \gl{hsy} \gl{3}-vomit-\gl{pert} \gl{cop}-\gl{pfv}//
        \glft ‘Their vomit was not there.’//  
    \endgl 
\xe

\subsection{\texorpdfstring{\obj{-sarë} \label{sec:sareimn}}{ }}

\begin{itemize}
\tightlist
\item
  once a converb, now `imminent future'
\end{itemize}

\ex Yawarana \\
\label{ctorat-25}    \begingl
    \glpreamble  irëjpë ta ti ta konopo wejsarë konopo wejsarë //
    \gla irëjpë ta-∅ ti ta konopo wej-sarë konopo wej-sarë//
    \glb then say-\gl{ipfv} \gl{hsy} like rain \gl{cop}-\gl{imn} rain \gl{cop}-\gl{imn}//
        \glft ‘Then they said: “it’s raining, it’s raining”.’//  
    \endgl 
\xe

\ex Yawarana \\
\label{ctoaragrme-25}    \begingl
    \glpreamble  moyochi tasarë, moyochi chipokono kojpaye pïnika warotari //
    \gla moyochi ta-sarë moyochi chipokono kojpaye pïnika warotari//
    \glb *** say-\gl{imn} *** *** *** \gl{prob} ***//
        \glft ‘le dicen araña, tal vez porque la araña trabaja de noche’//  
    \endgl 
\xe

\section{Subordinate Clause markers}

\begin{itemize}
\tightlist
\item
  Nominalizations
\item
  Adverbial Clauses

  \begin{itemize}
  \tightlist
  \item
    \obj{-se} `supine'
  \item
    \obj{-tojpe} `purpose'
  \item
    (‑jpë)=tërë `after'
  \item
    \obj{-tane} `concessive'
  \item
    \obj{-sarë} `converb'
  \item
    \obj{yawë} `simult'
  \item
    \emph{‑yapo} `neg.purp'
  \item
    others?
  \end{itemize}
\end{itemize}

What about desiderative \obj{-po}?

\section{Number}

\begin{itemize}
\tightlist
\item
  \emph{‑rï=kontomo}
\item
  \emph{‑se=jne=kontomo}
\item
  \emph{‑se=jne=pano} (\emph{‑se=jne=kontom=pano}?)
\end{itemize}

\section{Copula / Auxiliary}

\begin{itemize}
\tightlist
\item
  there is stem allomorphy: \obj{chi}, \obj{wej}
\end{itemize}

\chapter{\texorpdfstring{Verbal roots and stems
\label{derbderiv}}{Verbal roots and stems }}

\section{Deriving verbs}

\begin{itemize}
\tightlist
\item
  denominal verbalizers: \obj{-ta}, \obj{-jtë}?
\item
  detransitivizers: \obj{s-}, \obj{ëj-}
\end{itemize}

\chapter{\texorpdfstring{Adverbs \label{adverbs}}{Adverbs }}

\section{Inflection}

\begin{itemize}
\tightlist
\item
\end{itemize}

\section{\texorpdfstring{Simple adverbs
\label{sec:simpleadv}}{Simple adverbs }}

\section{\texorpdfstring{Derived adverbs
\label{sec:derivedadv}}{Derived adverbs }}

\obj{-tojpe} can be inflected:

\ex Yawarana \\
\label{convestsjm-086}    \textit{ijtëtojpe }\\
        ‘para que fuera’ \xe

\ex Yawarana \\
\label{histpedgrme-163}    \textit{tayëntojpe }\\
        ‘para que tome’ \xe

\ex Yawarana \\
\label{histyarirdi-0875}    \textit{uyepematojpe pïrarë wïrë inawë }\\
        ‘yo no tengo para pagar’ \xe

\chapter{\texorpdfstring{Postpositions \label{postp}}{Postpositions }}

\section{Defining the category}

\section{\texorpdfstring{Inflectional morphology
\label{sec:postinfl}}{Inflectional morphology }}

Postpositions take the same inflectional prefixes as nouns
(\cref{sec:nounposssuf}).

\begin{tabular}[t]{ll}

       \\

\gl{1} &     \obj{u-} \\
\gl{2} &    \obj{më-} \\
\gl{3} & \obj{i-/t-}? \\

\end{tabular}

\section{Locative Postpositions}

\begin{itemize}
\tightlist
\item
  clear bipartite Ground+Path
\item
  unproductive Bipartite X+Path?
\item
  other forms
\end{itemize}

\begin{table}
\caption{Locative postpositions}
\label{tab:locpost}
\centering
\begin{tabular}{lll}
\toprule
        &   \gl{all} &   \gl{loc} \\
\midrule
 inside & \obj{yaka} & \obj{yawë} \\
aquatic &          ? &          ? \\
\bottomrule
\end{tabular}

\end{table}

\begin{itemize}
\item
  \emph{poye} `above'
\item
  \emph{po} `locative'
\item
  \emph{yatë} `locative'
\item
  \emph{yapo} `negation'?
\item
  allative:
\end{itemize}

\ex Yawarana \\
\label{histpajirdi-186}    \begingl
    \glpreamble  tichikimuru, peti warë patakasapë Yakucho pana //
    \gla tichikimuru peti warë patakasapë yakucho pana//
    \glb *** *** thus *** *** \gl{dat}//
        \glft ‘su rodilla, su pierna, salió (llaga) hacia Ayacucho’//  
    \endgl 
\xe

\section{Nonlocative Oblique Postpositions}

\begin{itemize}
\tightlist
\item
  \obj{ya} `\gl{erg}'
\item
  \emph{ke} `\gl{ins}'
\item
  \emph{wanai}
\end{itemize}

\section{Propositional Postpositions}

\begin{itemize}
\tightlist
\item
  =se `\gl{des}'
\end{itemize}

\section{Misc}

\begin{itemize}
\tightlist
\item
  copular \obj{chi} combines with \obj{yawë}, sometimes spelled
  \emph{chi yawë}, sometimes \emph{chawë}.
\end{itemize}

\chapter{\texorpdfstring{Particles and Ideophones
\label{partideo}}{Particles and Ideophones }}

\chapter{\texorpdfstring{Negation \label{negation}}{Negation }}

\begin{itemize}
\tightlist
\item
  probably relevant morphemes:

  \begin{itemize}
  \item
    \obj{-ja}
  \item
    \obj{-jra}
  \item
    \obj{-jnari}
  \item
    \obj{-kempïnirë}
  \item
    \obj{pïnirë}
  \item
    \obj{pïrarë}
  \end{itemize}
\end{itemize}

\chapter{For testing and demonstration purposes}

\section{Some unparsable forms with derivational morphology}

\begin{itemize}
\tightlist
\item
  combination of verb with \obj{-kempïnirë} results in what? always used
  as predicate
\end{itemize}

\ex Yawarana \\
\label{convrisamaj-07}    \begingl
    \glpreamble  wïrë yaruwakempïnirë, mëkïsantomo a-ja-ja tajtane //
    \gla wïrë yaruwakempïnirë mëkïsantomo ajaja taj-tane//
    \glb \gl{1}\gl{pro} *** \gl{dist}.\gl{anim}.\gl{pl} hahaha say-\gl{cncs}//
        \glft ‘I don’t laugh, but they are saying “hahaha”.’//  
    \endgl 
\xe

\begin{itemize}
\tightlist
\item
  deverbal from \obj{tunamï} to an adverb, right?
\end{itemize}

\ex Yawarana \\
\label{convrisamaj-09}    \begingl
    \glpreamble  tëwï neke ne, tajne yakarama pokono nwarë tajne iri mïntë, tunampe //
    \gla tëwï neke ne ta-jne yakarama-∅ poko-no nwarë ta-jne i-ri mïntë tunam-pe//
    \glb \gl{3}\gl{pro} \gl{contrast} \gl{ints} \gl{3}-\gl{pl} tell-\gl{ipfv} on.surface-\gl{nmlz} thus \gl{3}-\gl{pl} do-\gl{ipfv} there.\gl{loc} hide-\gl{ess}//
        \glft ‘Yes indeed, they tell what they are doing there on the down-low.’//  
    \endgl 
\xe

\begin{itemize}
\tightlist
\item
  two more \obj{pe}:
\end{itemize}

\ex Yawarana \\
\label{anfoperso-40}    \begingl
    \glpreamble  yatampe ana tëse, pipi tawara rë waraijtokompe ijtëse //
    \gla yatampe ana të-se pipi tawara rë waraijtokompe ij-të-se//
    \glb *** \gl{1+3}\gl{pro} go-\gl{pst} younger.brother.of.woman too \gl{emp} *** \gl{3}-go-\gl{pst}//
        \glft ‘We became boys, my brother also became a man.’//  
    \endgl 
\xe

\begin{itemize}
\tightlist
\item
  deverbal nominalizer:
\end{itemize}

\ex Yawarana \\
\label{convrisamaj-13}    \begingl
    \glpreamble  pïrarë, seneja ejnë yarikatopo //
    \gla pïrarë sene-∅-ja ejnë yarika-topo//
    \glb nothing see.self-\gl{ipfv}-\gl{neg} \gl{1+2}\gl{pro} laugh-\gl{circ}.\gl{nmlz}//
        \glft ‘Nothing, there’s nothing for us to laugh.’//  
    \endgl 
\xe

\begin{itemize}
\tightlist
\item
  \obj{nope} + \obj{-ano}:
\end{itemize}

\ex Yawarana \\
\label{anfoperso-45}    \begingl
    \glpreamble  nopano wejsaj ta waraijtokomo //
    \gla nopano wej-saj ta waraijtokomo//
    \glb *** \gl{cop}-\gl{pfv} like man//
        \glft ‘He was a good man.’//  
    \endgl 
\xe

\begin{itemize}
\tightlist
\item
  probably special handling needed (copula + \obj{pëkë} + ? `because of
  that')
\end{itemize}

\ex Yawarana \\
\label{convrisamaj-29}    \begingl
    \glpreamble  nope seneja chipokono ejnë pana nope seneja chipokono ejnë yaruwatopo pïrärë //
    \gla nope sene-∅-ja chipokono ejnë pana nope sene-∅-ja chipokono ejnë yaruwa-topo pïrärë//
    \glb good see.self-\gl{ipfv}-\gl{neg} *** \gl{1+2}\gl{pro} \gl{dat} good see.self-\gl{ipfv}-\gl{neg} *** \gl{1+2}\gl{pro} laugh-\gl{circ}.\gl{nmlz} ***//
        \glft ‘We don’t see anything good, so we don’t laugh.’//  
    \endgl 
\xe

\begin{itemize}
\tightlist
\item
  how productive are verbalizers?
\end{itemize}

\ex Yawarana \\
\label{anfoperso-22}    \begingl
    \glpreamble  tëwï ya papa kampra pana tase ta, kaimotata //
    \gla tëwï ya papa kampra pana ta-se ta kaimo-ta-ta//
    \glb \gl{3}\gl{pro} \gl{erg} father big \gl{dat} say-\gl{pst} like game-\gl{vbz}-\gl{motimp}//
        \glft ‘She said to my uncle: “go hunt!”.’//  
    \endgl 
\xe

\begin{itemize}
\tightlist
\item
  what does \obj{-tane} do?
\end{itemize}

\ex Yawarana \\
\label{anfoperso-52}    \begingl
    \glpreamble  makë yakërë ma wejtane, tëijpë rë pïjkë ana yïwïtï //
    \gla makë y-akërë ma wej-tane tëijpë rë pïjkë ana y-ïwïtï//
    \glb mother \gl{lk}-with \gl{rst} \gl{cop}-\gl{cncs} far \gl{emp} \gl{dim} \gl{1+3}\gl{pro} \gl{lk}-house//
        \glft ‘Although I was with my mother, our house was a little further.’//  
    \endgl 
\xe

\begin{itemize}
\tightlist
\item
  MCMM has \emph{narë} as emphatic particle\ldots{} is this a
  adverbialization?
\end{itemize}

\ex Yawarana \\
\label{convrisamaj-47}    \begingl
    \glpreamble  aniki narëpe yakërë ejnë përemekïrï, ejnë pëkëpene, wanene //
    \gla aniki narëpe yakërë ejnë përemekï-rï ejnë pëkëpene wanene//
    \glb who *** *** \gl{1+2}\gl{pro} talk-\gl{ipfv} \gl{1+2}\gl{pro} alone aunt//
        \glft ‘Who are we gonna talk to? We’re alone, aunt.’//  
    \endgl 
\xe

\section{Inline linked entities}

\subsection{Single}

\begin{enumerate}
\def\labelenumi{\arabic{enumi}.}
\tightlist
\item
  morph: \obj{-jne}\footnote{This is what a footnote looks like.}
\item
  morpheme: \obj{-jnë}
\item
  wordform: (asamo-0-ri-jraneg)
\item
  text: ``Conversación sobre la risa entre GrMe y ElPe''
\end{enumerate}

\subsection{Multiple}

\begin{enumerate}
\def\labelenumi{\arabic{enumi}.}
\tightlist
\item
  morph: \obj{-jne}, and \obj{-i}
\item
  morpheme: \obj{-jnë}, and \obj{-ru}
\item
  wordform: (asamo-0-ri-jraneg,asamo-0-ri-zero)
\item
  text:
\end{enumerate}

\section{Examples}

\subsection{Interlinear}

single:

\ex Yawarana \\
\label{ctorat-34}    \begingl
    \glpreamble  ëkëtë mërë ëkï //
    \gla ëkëtë mërë ëkï//
    \glb where \gl{med}.\gl{inan} manioc.beer//
        \glft ‘Where is the chicha?’//  
    \endgl 
\xe

multiple:

\pex\label{multiigt}    \a Yawarana\\
    \label{ctorat-35}        \begingl
        \glpreamble  ëkï ta rë pïrarë wenarujpë ta rë pïrarë //
        \gla ëkï ta rë pïrarë wena-ru-jpë ta rë pïrarë//
        \glb manioc.beer like \gl{emp} nothing vomit-\gl{pert}-\gl{pst} like \gl{emp} nothing//
            \glft ‘The chicha was gone and so was the vomit.’//  
        \endgl 
    \a Yawarana\\
    \label{ctorat-36}        \begingl
        \glpreamble  ta ti wejsaj ti tëwï //
        \gla ta-∅ ti wej-saj ti tëwï//
        \glb say-\gl{ipfv} \gl{hsy} \gl{cop}-\gl{pfv} \gl{hsy} \gl{3}\gl{pro}//
            \glft ‘...he said.’//  
        \endgl 
\xe

\subsection{Other}

single:

\ex\label{test1}\begin{tabular}[t]{ll}

\obj{yaka} &   root \\

 \obj{-se} & suffix \\

\end{tabular}
 \xe

double:

\pex\label{multiparttest} \a\label{test2} Hello \obj{-se} `PST'
\a\label{test3} World

\emph{and} more \xe

You can even mix them:

\pex\label{multiparttest2} \a\label{test4} Some random text in
combination with an example from the corpus.

\a Yawarana \\
\label{ctorat-20}    \begingl
    \glpreamble  wenaru pïrarë //
    \gla wena-ru pïrarë//
    \glb vomit-\gl{pert} nothing//
        \glft ‘There was no vomit.’//  
    \endgl 

\xe

\subsection{Example references}

\exref[]{ctorat-34}

\exref[]{multiigt} or \exref[]{ctorat-36} or even \exref[a-b]{multiigt}

\exref[]{test1}

\exref[]{multiparttest}

\exref[][multiparttest]{test1}

\printbibliography

\end{document}