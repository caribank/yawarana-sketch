\documentclass{article}
\usepackage{fontspec}
\setmainfont{Brill}
\usepackage{glossingtool}
\usepackage{biblatex}
\addbibresource{../../yaw_cldf/etc/car.bib}
\DeclareAcronym{inm}{short=inm,long=?,short-format=\scshape}
\DeclareAcronym{plur}{short=plur,long=pluractional,short-format=\scshape}
\DeclareAcronym{emph}{short=emp,long=emphatic,short-format=\scshape}
\DeclareAcronym{vbz.tr}{short=vbz.tr,long=transitive verbalizer,short-format=\scshape}
\DeclareAcronym{p}{short=P,long=patientive transitive argument,short-format=\scshape}
\DeclareAcronym{contrast}{short=contrast,long=contrastive,short-format=\scshape}
\DeclareAcronym{restr}{short=restr,long=restrictive,short-format=\scshape}
\DeclareAcronym{tmp:dem}{short=tmp:dem,long=???,short-format=\scshape}
\DeclareAcronym{rel.inan}{short=rel.inan,long=inanimate relativizer,short-format=\scshape}
\DeclareAcronym{an.rel}{short=rel.anim,long=animate relativizer,short-format=\scshape}
\DeclareAcronym{in:md}{short=?,long=?,short-format=\scshape}
\usepackage{booktabs}
\usepackage{hyperref}
\begin{document}

\hypertarget{verbs}{%
\section{\texorpdfstring{Verbs \label{verbs}}{Verbs }}\label{verbs}}

This is an example for a simple past verb with \obj{-se}.

\ex<ctorat-42>  
\begingl
\glpreamble tëwï ajpachi yaka wonse pïnika tëwï //
\gla tëwï ajpachi yaka won-se pïnika tëwï//
\glb 3SG weeds into enter-PST PROB 3SG//
\glft ‘tal vez se metió en el monte’//  
\endgl 
\xe

\hypertarget{basic-morphological-template}{%
\subsection{Basic morphological
template}\label{basic-morphological-template}}

\begin{table}
\caption{Basic verb template}
\label{verb_templ}
\centering
\begin{tabular}{lllll}
\toprule
  Prefix & Root &     Aspect &        Tense &      Number \\
\midrule
\obj{i-} &      & \obj{-pëtï} &  \obj{-se} & \obj{-jnë} \\
         &      &            & \obj{-jtë} &             \\
\bottomrule
\end{tabular}

\end{table}

\hypertarget{the-pluractional-marker}{%
\subsection{\texorpdfstring{The pluractional marker
\obj{-pëtï}}{The pluractional marker }}\label{the-pluractional-marker}}

This aspect-marking morpheme is known from other Cariban languages
\parencite{mattiola2020pluractional}. This example illustrates its
\obj{-pëj} allomorph:

\ex<ctorat-40>  
\begingl
\glpreamble tipapëjsejne waijtajne //
\gla tipa-pëj-se-jne waijta-jne//
\glb go.group-PLUR-PST-PL mouse-PL//
\glft ‘las ratas se fueron’//  
\endgl 
\xe

\hypertarget{nouns}{%
\section{Nouns}\label{nouns}}

Someting about nouns. \printbibliography


\end{document}